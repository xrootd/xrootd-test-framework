% Generated by Sphinx.
\def\sphinxdocclass{report}
\documentclass[a4paper,11pt,openany]{sphinxmanual}
\usepackage[utf8]{inputenc}
\DeclareUnicodeCharacter{00A0}{\nobreakspace}
\usepackage[T1]{fontenc}
\usepackage[english]{babel}
\usepackage{times}
\usepackage[Bjarne]{fncychap}
\usepackage{longtable}
\usepackage{sphinx}
\usepackage{multirow}


\title{xrootd-tests Documentation}
\date{October 23, 2012}
\release{0.1}
\author{CERN, Justin Lewis Salmon}
\newcommand{\sphinxlogo}{}
\renewcommand{\releasename}{Release}
\makeindex

\makeatletter
\def\PYG@reset{\let\PYG@it=\relax \let\PYG@bf=\relax%
    \let\PYG@ul=\relax \let\PYG@tc=\relax%
    \let\PYG@bc=\relax \let\PYG@ff=\relax}
\def\PYG@tok#1{\csname PYG@tok@#1\endcsname}
\def\PYG@toks#1+{\ifx\relax#1\empty\else%
    \PYG@tok{#1}\expandafter\PYG@toks\fi}
\def\PYG@do#1{\PYG@bc{\PYG@tc{\PYG@ul{%
    \PYG@it{\PYG@bf{\PYG@ff{#1}}}}}}}
\def\PYG#1#2{\PYG@reset\PYG@toks#1+\relax+\PYG@do{#2}}

\expandafter\def\csname PYG@tok@gd\endcsname{\def\PYG@tc##1{\textcolor[rgb]{0.63,0.00,0.00}{##1}}}
\expandafter\def\csname PYG@tok@gu\endcsname{\let\PYG@bf=\textbf\def\PYG@tc##1{\textcolor[rgb]{0.50,0.00,0.50}{##1}}}
\expandafter\def\csname PYG@tok@gt\endcsname{\def\PYG@tc##1{\textcolor[rgb]{0.00,0.25,0.82}{##1}}}
\expandafter\def\csname PYG@tok@gs\endcsname{\let\PYG@bf=\textbf}
\expandafter\def\csname PYG@tok@gr\endcsname{\def\PYG@tc##1{\textcolor[rgb]{1.00,0.00,0.00}{##1}}}
\expandafter\def\csname PYG@tok@cm\endcsname{\let\PYG@it=\textit\def\PYG@tc##1{\textcolor[rgb]{0.25,0.50,0.56}{##1}}}
\expandafter\def\csname PYG@tok@vg\endcsname{\def\PYG@tc##1{\textcolor[rgb]{0.73,0.38,0.84}{##1}}}
\expandafter\def\csname PYG@tok@m\endcsname{\def\PYG@tc##1{\textcolor[rgb]{0.13,0.50,0.31}{##1}}}
\expandafter\def\csname PYG@tok@mh\endcsname{\def\PYG@tc##1{\textcolor[rgb]{0.13,0.50,0.31}{##1}}}
\expandafter\def\csname PYG@tok@cs\endcsname{\def\PYG@tc##1{\textcolor[rgb]{0.25,0.50,0.56}{##1}}\def\PYG@bc##1{\setlength{\fboxsep}{0pt}\colorbox[rgb]{1.00,0.94,0.94}{\strut ##1}}}
\expandafter\def\csname PYG@tok@ge\endcsname{\let\PYG@it=\textit}
\expandafter\def\csname PYG@tok@vc\endcsname{\def\PYG@tc##1{\textcolor[rgb]{0.73,0.38,0.84}{##1}}}
\expandafter\def\csname PYG@tok@il\endcsname{\def\PYG@tc##1{\textcolor[rgb]{0.13,0.50,0.31}{##1}}}
\expandafter\def\csname PYG@tok@go\endcsname{\def\PYG@tc##1{\textcolor[rgb]{0.19,0.19,0.19}{##1}}}
\expandafter\def\csname PYG@tok@cp\endcsname{\def\PYG@tc##1{\textcolor[rgb]{0.00,0.44,0.13}{##1}}}
\expandafter\def\csname PYG@tok@gi\endcsname{\def\PYG@tc##1{\textcolor[rgb]{0.00,0.63,0.00}{##1}}}
\expandafter\def\csname PYG@tok@gh\endcsname{\let\PYG@bf=\textbf\def\PYG@tc##1{\textcolor[rgb]{0.00,0.00,0.50}{##1}}}
\expandafter\def\csname PYG@tok@ni\endcsname{\let\PYG@bf=\textbf\def\PYG@tc##1{\textcolor[rgb]{0.84,0.33,0.22}{##1}}}
\expandafter\def\csname PYG@tok@nl\endcsname{\let\PYG@bf=\textbf\def\PYG@tc##1{\textcolor[rgb]{0.00,0.13,0.44}{##1}}}
\expandafter\def\csname PYG@tok@nn\endcsname{\let\PYG@bf=\textbf\def\PYG@tc##1{\textcolor[rgb]{0.05,0.52,0.71}{##1}}}
\expandafter\def\csname PYG@tok@no\endcsname{\def\PYG@tc##1{\textcolor[rgb]{0.38,0.68,0.84}{##1}}}
\expandafter\def\csname PYG@tok@na\endcsname{\def\PYG@tc##1{\textcolor[rgb]{0.25,0.44,0.63}{##1}}}
\expandafter\def\csname PYG@tok@nb\endcsname{\def\PYG@tc##1{\textcolor[rgb]{0.00,0.44,0.13}{##1}}}
\expandafter\def\csname PYG@tok@nc\endcsname{\let\PYG@bf=\textbf\def\PYG@tc##1{\textcolor[rgb]{0.05,0.52,0.71}{##1}}}
\expandafter\def\csname PYG@tok@nd\endcsname{\let\PYG@bf=\textbf\def\PYG@tc##1{\textcolor[rgb]{0.33,0.33,0.33}{##1}}}
\expandafter\def\csname PYG@tok@ne\endcsname{\def\PYG@tc##1{\textcolor[rgb]{0.00,0.44,0.13}{##1}}}
\expandafter\def\csname PYG@tok@nf\endcsname{\def\PYG@tc##1{\textcolor[rgb]{0.02,0.16,0.49}{##1}}}
\expandafter\def\csname PYG@tok@si\endcsname{\let\PYG@it=\textit\def\PYG@tc##1{\textcolor[rgb]{0.44,0.63,0.82}{##1}}}
\expandafter\def\csname PYG@tok@s2\endcsname{\def\PYG@tc##1{\textcolor[rgb]{0.25,0.44,0.63}{##1}}}
\expandafter\def\csname PYG@tok@vi\endcsname{\def\PYG@tc##1{\textcolor[rgb]{0.73,0.38,0.84}{##1}}}
\expandafter\def\csname PYG@tok@nt\endcsname{\let\PYG@bf=\textbf\def\PYG@tc##1{\textcolor[rgb]{0.02,0.16,0.45}{##1}}}
\expandafter\def\csname PYG@tok@nv\endcsname{\def\PYG@tc##1{\textcolor[rgb]{0.73,0.38,0.84}{##1}}}
\expandafter\def\csname PYG@tok@s1\endcsname{\def\PYG@tc##1{\textcolor[rgb]{0.25,0.44,0.63}{##1}}}
\expandafter\def\csname PYG@tok@gp\endcsname{\let\PYG@bf=\textbf\def\PYG@tc##1{\textcolor[rgb]{0.78,0.36,0.04}{##1}}}
\expandafter\def\csname PYG@tok@sh\endcsname{\def\PYG@tc##1{\textcolor[rgb]{0.25,0.44,0.63}{##1}}}
\expandafter\def\csname PYG@tok@ow\endcsname{\let\PYG@bf=\textbf\def\PYG@tc##1{\textcolor[rgb]{0.00,0.44,0.13}{##1}}}
\expandafter\def\csname PYG@tok@sx\endcsname{\def\PYG@tc##1{\textcolor[rgb]{0.78,0.36,0.04}{##1}}}
\expandafter\def\csname PYG@tok@bp\endcsname{\def\PYG@tc##1{\textcolor[rgb]{0.00,0.44,0.13}{##1}}}
\expandafter\def\csname PYG@tok@c1\endcsname{\let\PYG@it=\textit\def\PYG@tc##1{\textcolor[rgb]{0.25,0.50,0.56}{##1}}}
\expandafter\def\csname PYG@tok@kc\endcsname{\let\PYG@bf=\textbf\def\PYG@tc##1{\textcolor[rgb]{0.00,0.44,0.13}{##1}}}
\expandafter\def\csname PYG@tok@c\endcsname{\let\PYG@it=\textit\def\PYG@tc##1{\textcolor[rgb]{0.25,0.50,0.56}{##1}}}
\expandafter\def\csname PYG@tok@mf\endcsname{\def\PYG@tc##1{\textcolor[rgb]{0.13,0.50,0.31}{##1}}}
\expandafter\def\csname PYG@tok@err\endcsname{\def\PYG@bc##1{\setlength{\fboxsep}{0pt}\fcolorbox[rgb]{1.00,0.00,0.00}{1,1,1}{\strut ##1}}}
\expandafter\def\csname PYG@tok@kd\endcsname{\let\PYG@bf=\textbf\def\PYG@tc##1{\textcolor[rgb]{0.00,0.44,0.13}{##1}}}
\expandafter\def\csname PYG@tok@ss\endcsname{\def\PYG@tc##1{\textcolor[rgb]{0.32,0.47,0.09}{##1}}}
\expandafter\def\csname PYG@tok@sr\endcsname{\def\PYG@tc##1{\textcolor[rgb]{0.14,0.33,0.53}{##1}}}
\expandafter\def\csname PYG@tok@mo\endcsname{\def\PYG@tc##1{\textcolor[rgb]{0.13,0.50,0.31}{##1}}}
\expandafter\def\csname PYG@tok@mi\endcsname{\def\PYG@tc##1{\textcolor[rgb]{0.13,0.50,0.31}{##1}}}
\expandafter\def\csname PYG@tok@kn\endcsname{\let\PYG@bf=\textbf\def\PYG@tc##1{\textcolor[rgb]{0.00,0.44,0.13}{##1}}}
\expandafter\def\csname PYG@tok@o\endcsname{\def\PYG@tc##1{\textcolor[rgb]{0.40,0.40,0.40}{##1}}}
\expandafter\def\csname PYG@tok@kr\endcsname{\let\PYG@bf=\textbf\def\PYG@tc##1{\textcolor[rgb]{0.00,0.44,0.13}{##1}}}
\expandafter\def\csname PYG@tok@s\endcsname{\def\PYG@tc##1{\textcolor[rgb]{0.25,0.44,0.63}{##1}}}
\expandafter\def\csname PYG@tok@kp\endcsname{\def\PYG@tc##1{\textcolor[rgb]{0.00,0.44,0.13}{##1}}}
\expandafter\def\csname PYG@tok@w\endcsname{\def\PYG@tc##1{\textcolor[rgb]{0.73,0.73,0.73}{##1}}}
\expandafter\def\csname PYG@tok@kt\endcsname{\def\PYG@tc##1{\textcolor[rgb]{0.56,0.13,0.00}{##1}}}
\expandafter\def\csname PYG@tok@sc\endcsname{\def\PYG@tc##1{\textcolor[rgb]{0.25,0.44,0.63}{##1}}}
\expandafter\def\csname PYG@tok@sb\endcsname{\def\PYG@tc##1{\textcolor[rgb]{0.25,0.44,0.63}{##1}}}
\expandafter\def\csname PYG@tok@k\endcsname{\let\PYG@bf=\textbf\def\PYG@tc##1{\textcolor[rgb]{0.00,0.44,0.13}{##1}}}
\expandafter\def\csname PYG@tok@se\endcsname{\let\PYG@bf=\textbf\def\PYG@tc##1{\textcolor[rgb]{0.25,0.44,0.63}{##1}}}
\expandafter\def\csname PYG@tok@sd\endcsname{\let\PYG@it=\textit\def\PYG@tc##1{\textcolor[rgb]{0.25,0.44,0.63}{##1}}}

\def\PYGZbs{\char`\\}
\def\PYGZus{\char`\_}
\def\PYGZob{\char`\{}
\def\PYGZcb{\char`\}}
\def\PYGZca{\char`\^}
\def\PYGZam{\char`\&}
\def\PYGZlt{\char`\<}
\def\PYGZgt{\char`\>}
\def\PYGZsh{\char`\#}
\def\PYGZpc{\char`\%}
\def\PYGZdl{\char`\$}
\def\PYGZti{\char`\~}
% for compatibility with earlier versions
\def\PYGZat{@}
\def\PYGZlb{[}
\def\PYGZrb{]}
\makeatother

\begin{document}

\maketitle
\tableofcontents
\phantomsection\label{index::doc}



\chapter{Contents}
\label{index:contents}\label{index:xrdtest-framework-release-documentation}

\section{General Overview}
\label{general-overview:general-overview}\label{general-overview::doc}
The XrdTest Framework is comprised of 3 main components:
\begin{itemize}
\item {} 
{\hyperref[general-overview:master]{\emph{Master}}}

\item {} 
{\hyperref[general-overview:hypervisor]{\emph{Hypervisor}}}

\item {} 
{\hyperref[general-overview:slave]{\emph{Slave}}}

\end{itemize}

Each of which is explained in more detail below.


\subsection{Master}
\label{general-overview:master}\label{general-overview:id1}
Module file: \code{XrdTestMaster.py}

The master is the user entry point for the testing framework. The service is
configured via a \emph{ini}-style configuration file (see {\hyperref[config-master::doc]{\emph{Master configuration}}}).

It includes a web interface showing the current statistics of the service,
as well as the status of the tests that are being run and have been run in the
past (see {\hyperref[web-interface::doc]{\emph{Using the XrdTest Web Interface}}}).

It accepts connections from slave and hypervisor daemons and dispatches commands
to them. The master is responsible for running,  orchestrating and synchronizing
test suites.

Quick summary:
\begin{itemize}
\item {} 
User entry point to the framework

\item {} 
Supervises and synchronizes all system activities

\item {} 
Accepts connections from slaves and hypervisors and dispatches commands to
them

\item {} 
Runs as a system service (daemon), configured via batch of configuration files

\end{itemize}


\subsection{Hypervisor}
\label{general-overview:hypervisor}\label{general-overview:id2}
Module file: \code{XrdTestHypervisor.py}

The hypervisor receives cluster configurations from the master and starts/stops
/configures the virtual machines which make up the cluster accordingly,
including configuring the virtual network with which the slaves use to
communicate. It uses \code{qemu} with the \code{kvm} kernel module in Linux
and the \code{libvirt} virtualization API as a layer to communicate with
\code{qemu}.

Quick summary:
\begin{itemize}
\item {} 
Component to manage clusters of virtual machines on demand of the master

\item {} 
It is run as a system service (daemon), configured via configuration file
(see {\hyperref[config-hypervisor::doc]{\emph{Hypervisor configuration}}})

\item {} 
Starts/stops/configures virtual machines

\item {} 
Uses \code{libvirt} for managing virtual machines

\end{itemize}


\subsection{Slave}
\label{general-overview:slave}\label{general-overview:id3}
Module file: \code{XrdTestSlave.py}

The slave component is installed on virtual or physical machines, and runs the
actual tests. In the first iteration it will receive a bunch of shell scripts
from the master and run them. Slaves connect to the master automatically, made
possible by libvirt's use of dnsmasq.

Quick summary:
\begin{itemize}
\item {} 
The component which actually runs tests

\item {} 
May be running on virtual or physical machines

\item {} 
Runs as a system service (daemon), configured via configuration file (see
{\hyperref[config-slave::doc]{\emph{Slave configuration}}})

\item {} 
Receives test cases from the master and runs them synchronously with other
slaves

\end{itemize}


\subsection{Running as a system service (daemon)}
\label{general-overview:running-as-a-system-service-daemon}
If the application was installed from an RPM, it is automatically added to the
system services (via \code{chkconfig}), thus it will be started automatically
during system boot. It can also be started manually from the command line as
follows:

\begin{Verbatim}[commandchars=\\\{\}]
service COMPONENT\_NAME start
\end{Verbatim}

where COMPONENT\_NAME is accordingly:
\begin{itemize}
\item {} 
\code{xrdtest-master}

\item {} 
\code{xrdtest-slave}

\item {} 
\code{xrdtest-hypervisor}

\end{itemize}


\subsection{Running in debug mode}
\label{general-overview:running-in-debug-mode}
\begin{notice}{note}{Note:}
The default location for configuration files is
\code{/etc/XrdTest/\textless{}CONFIG\_FILE\_NAME\textgreater{}.conf}.
\end{notice}

To start each component (master, hypervisor or slave) in debug mode, run the
shell command below:

\begin{Verbatim}[commandchars=\\\{\}]
python /usr/sbin/XrdTestMaster.py -c XrdTestMaster.conf
\end{Verbatim}

One can start a hypervisor or a slave by replacing \code{XrdTestMaster} with
\code{XrdTestHypervisor} or \code{XrdTestSlave} accordingly.

When running in debug mode, the component will print log messages to \code{stdout},
rather than writing them to the log file.


\subsection{Running in background mode}
\label{general-overview:running-in-background-mode}
To start a component in background mode (as a daemon), add the \code{-b} option
to the shell command. It will then store log and PID files in their proper
directories, as specified in the configuration file. For example:

\begin{Verbatim}[commandchars=\\\{\}]
\PYG{c}{\PYGZsh{} python /usr/sbin/XrdTestMaster.py -d -c XrdTestMaster.conf}
\end{Verbatim}


\section{Installation}
\label{installation:installation}\label{installation::doc}
The framework is available in RPM packages for Linux SLC6. Each application
requires at least Python \code{2.4}. It comprises the following four main packages
(the libraries required by them are also listed below):
\begin{itemize}
\item {} 
\textbf{xrdtest-lib}

Dependencies: None

\item {} 
\textbf{xrdtest-master}
\begin{description}
\item[{Dependencies:}] \leavevmode\begin{itemize}
\item {} 
\code{python-apscheduler-2.0.3}

\item {} 
\code{python-cheetah-2.4.1}

\item {} 
\code{python-cherrypy-3.1.2}

\item {} 
\code{python-inotify-0.9.1}

\item {} 
\code{python-uuid-1.30}

\item {} 
\code{pyOpenSSL-0.10-2}

\item {} 
\code{python-ssl-1.15}

\end{itemize}

\end{description}

\item {} 
\textbf{xrdtest-hypervisor}
\begin{description}
\item[{Dependencies:}] \leavevmode\begin{itemize}
\item {} 
\code{python-ssl-1.15}

\item {} 
\code{libvirt-python-0.9.3}

\item {} 
\code{libvirt-0.9.10}

\end{itemize}

\end{description}

\item {} 
\textbf{xrdtest-slave}
\begin{description}
\item[{Dependencies:}] \leavevmode\begin{itemize}
\item {} 
\code{python-ssl-1.15}

\end{itemize}

\end{description}

\end{itemize}

To install each component, you must follow typical RPM installation instructions.


\section{Master configuration}
\label{config-master:master-configuration}\label{config-master::doc}
This section describes how to configure the XrdTest Master framework component,
including how to set up a repository to hold test suite and cluster definitions,
web interface options, logging options and security configurations.

\begin{notice}{note}{Note:}
The default location for configuration files is
\code{/etc/XrdTest/\textless{}CONFIG\_FILE\_NAME\textgreater{}.conf}.
\end{notice}

The master configuration file uses the \emph{ini}-style format of the python
\code{ConfigParser} module. There are multiple sections, each of which is explained
separately below. First, the configuration directive will be given, followed by
an explanation.


\subsection{Configuration sections}
\label{config-master:configuration-sections}

\subsubsection{\texttt{{[}general{]}}}
\label{config-master:general}
\begin{Verbatim}[commandchars=\\\{\}]
\PYG{n}{test}\PYG{o}{-}\PYG{n}{repos}\PYG{o}{=}\PYG{n}{remote}\PYG{p}{,}\PYG{n}{local}
\end{Verbatim}

A list of repositories to use, each of which must have a corresponding
{[}test-repo-\textless{}reponame\textgreater{}{]} section below. As an example, we use two test suites: one
local (\code{test-repo-local}), and one in a remote \code{git} repository
(\code{test-repo-remote}).

\begin{Verbatim}[commandchars=\\\{\}]
suite\_sessions\_file=/var/log/XrdTest/suite\_history.bin
\end{Verbatim}

The path to the file which stores previous test suite history.


\subsubsection{\texttt{{[}test-repo-remote{]}}}
\label{config-master:test-repo-remote}
The section for the first of our two example repositories. This repository is a
remote \code{git} repository. Currently, the framework supports localfs and \code{git}
repositories only. It is planned to include \code{svn} support in the future.

\begin{notice}{note}{Note:}
You need passwordless access to the repository for this
to work (such as key-based SSH, Kerberos, or a HTTP URL). Password based
authentication will not work, as synchronization of the remote repository
happens automatically at certain time intervals.
\end{notice}

\begin{Verbatim}[commandchars=\\\{\}]
\PYG{c}{\PYGZsh{} Example settings for a remote git repository.}
\PYG{n+nb}{type}\PYG{o}{=}\PYG{n}{git}

\PYG{c}{\PYGZsh{} Path to the remote repository. Accepts any valid Git URL.}
\PYG{n}{remote\PYGZus{}repo}\PYG{o}{=}\PYG{n}{jsalmon}\PYG{n+nd}{@xrootd.cern.ch}\PYG{p}{:}\PYG{o}{/}\PYG{n}{var}\PYG{o}{/}\PYG{n}{repos}\PYG{o}{/}\PYG{n}{xrootd}\PYG{o}{-}\PYG{n}{testsuite}\PYG{o}{.}\PYG{n}{git}

\PYG{c}{\PYGZsh{} Which local/remote branches to use.}
\PYG{n}{remote\PYGZus{}branch}\PYG{o}{=}\PYG{n}{origin}\PYG{o}{/}\PYG{n}{master}
\PYG{n}{local\PYGZus{}branch}\PYG{o}{=}\PYG{n}{master}

\PYG{c}{\PYGZsh{} Path where the remote repo will be checked out locally.}
\PYG{n}{local\PYGZus{}path}\PYG{o}{=}\PYG{o}{/}\PYG{n}{var}\PYG{o}{/}\PYG{n}{tmp}\PYG{o}{/}\PYG{n}{xrootd}\PYG{o}{-}\PYG{n}{testsuite}

\PYG{c}{\PYGZsh{} Paths to the local checkouts of cluster and test suite definitions.}
\PYG{n}{cluster\PYGZus{}defs\PYGZus{}path}\PYG{o}{=}\PYG{n}{clusters}
\PYG{n}{suite\PYGZus{}defs\PYGZus{}path}\PYG{o}{=}\PYG{n}{test}\PYG{o}{-}\PYG{n}{suites}
\end{Verbatim}

Each directive should be fairly self-explanatory. The \code{remote\_repo} directive
\textbf{accepts any valid git URL}.

It is necessary to provide a
path where the remote repository will be checked out, as the system in fact
clones the remote repository to this local path, does \code{fetch}/\code{diff}
periodically, then does \code{pull} if there are changes in the remote repo.

It is also necessary to point to the directories which hold cluster and test
suite definitions \textbf{inside the local checkout directory}. This is in case you
want to change the naming conventions to better suit your environment.


\subsubsection{\texttt{{[}test-repo-local{]}}}
\label{config-master:test-repo-local}
The section for the second example repository. This repository is located in the
local filesystem, and is much simpler to configure than a remote one.

\begin{Verbatim}[commandchars=\\\{\}]
\PYG{c}{\PYGZsh{} Example settings for a local repository of cluster/test suite definitions.}
\PYG{n+nb}{type}\PYG{o}{=}\PYG{n}{localfs}

\PYG{n}{local\PYGZus{}path}\PYG{o}{=}\PYG{o}{/}\PYG{n}{var}\PYG{o}{/}\PYG{n}{repos}\PYG{o}{/}\PYG{n}{xrootd}\PYG{o}{-}\PYG{n}{testsuite}
\PYG{n}{cluster\PYGZus{}defs\PYGZus{}path}\PYG{o}{=}\PYG{n}{clusters}
\PYG{n}{suite\PYGZus{}defs\PYGZus{}path}\PYG{o}{=}\PYG{n}{test}\PYG{o}{-}\PYG{n}{suites}
\end{Verbatim}

You need to point to the top directory, and the subdirectories which hold cluster
and test suite definitions.


\subsubsection{\texttt{{[}server{]}}}
\label{config-master:server}
\begin{Verbatim}[commandchars=\\\{\}]
\PYG{c}{\PYGZsh{} Password to authenticate hypervisors.}
\PYG{n}{connection\PYGZus{}passwd}\PYG{o}{=}\PYG{n}{some\PYGZus{}password}

\PYG{c}{\PYGZsh{} The IP and port the master will listen on.}
\PYG{n}{ip}\PYG{o}{=}\PYG{l+m+mf}{0.0}\PYG{o}{.}\PYG{l+m+mf}{0.0}
\PYG{n}{port}\PYG{o}{=}\PYG{l+m+mi}{20000}
\end{Verbatim}


\subsubsection{\texttt{{[}webserver{]}}}
\label{config-master:webserver}
\begin{Verbatim}[commandchars=\\\{\}]
\PYG{c}{\PYGZsh{} Absolute path to webpage files (defaults to /usr/share/XrdTest/webpage).}
\PYG{c}{\PYGZsh{} Uncomment and add your path to change the web root.}
\PYG{n}{webpage\PYGZus{}dir}\PYG{o}{=}\PYG{o}{/}\PYG{n}{usr}\PYG{o}{/}\PYG{n}{share}\PYG{o}{/}\PYG{n}{XrdTest}\PYG{o}{/}\PYG{n}{webpage}

\PYG{c}{\PYGZsh{} Protocol to use for the web server. Defaults to HTTP.}
\PYG{n}{protocol}\PYG{o}{=}\PYG{n}{https}

\PYG{c}{\PYGZsh{} The port to access the web interface on. Defaults to 8080 for HTTP and 8443}
\PYG{c}{\PYGZsh{} for HTTPS.}
\PYG{n}{port}\PYG{o}{=}\PYG{l+m+mi}{8443}

\PYG{c}{\PYGZsh{} The password that allows running test suites via the webpage (defaults to none)}
\PYG{c}{\PYGZsh{} suite\PYGZus{}run\PYGZus{}pass=somepass}
\end{Verbatim}


\subsubsection{\texttt{{[}scheduler{]}}}
\label{config-master:scheduler}
\begin{Verbatim}[commandchars=\\\{\}]
\PYG{c}{\PYGZsh{} If set to 0, the scheduler will not run, strangely enough.}
\PYG{n}{enabled}\PYG{o}{=}\PYG{l+m+mi}{1}
\end{Verbatim}


\subsubsection{\texttt{{[}security{]}}}
\label{config-master:security}
\begin{Verbatim}[commandchars=\\\{\}]
\PYG{c}{\PYGZsh{} Location of the master's SSL certificate and private key. Will be generated}
\PYG{c}{\PYGZsh{} automatically at install time. Don't change these.}
\PYG{n}{certfile}\PYG{o}{=}\PYG{o}{/}\PYG{n}{etc}\PYG{o}{/}\PYG{n}{XrdTest}\PYG{o}{/}\PYG{n}{certs}\PYG{o}{/}\PYG{n}{mastercert}\PYG{o}{.}\PYG{n}{pem}
\PYG{n}{keyfile}\PYG{o}{=}\PYG{o}{/}\PYG{n}{etc}\PYG{o}{/}\PYG{n}{XrdTest}\PYG{o}{/}\PYG{n}{certs}\PYG{o}{/}\PYG{n}{masterkey}\PYG{o}{.}\PYG{n}{pem}

\PYG{c}{\PYGZsh{} Location of the key/certificate which the master will use to become it's own}
\PYG{c}{\PYGZsh{} CA (for signing CSRs from slaves which need to use GSI).}
\PYG{n}{ca\PYGZus{}certfile}\PYG{o}{=}\PYG{o}{/}\PYG{n}{etc}\PYG{o}{/}\PYG{n}{XrdTest}\PYG{o}{/}\PYG{n}{certs}\PYG{o}{/}\PYG{n}{cacert}\PYG{o}{.}\PYG{n}{pem}
\PYG{n}{ca\PYGZus{}keyfile}\PYG{o}{=}\PYG{o}{/}\PYG{n}{etc}\PYG{o}{/}\PYG{n}{XrdTest}\PYG{o}{/}\PYG{n}{certs}\PYG{o}{/}\PYG{n}{cakey}\PYG{o}{.}\PYG{n}{pem}
\end{Verbatim}


\subsubsection{\texttt{{[}daemon{]}}}
\label{config-master:daemon}
\begin{Verbatim}[commandchars=\\\{\}]
\PYG{c}{\PYGZsh{} Path to PID file if being run in daemon mode.}
\PYG{n}{pid\PYGZus{}file\PYGZus{}path}\PYG{o}{=}\PYG{o}{/}\PYG{n}{var}\PYG{o}{/}\PYG{n}{run}\PYG{o}{/}\PYG{n}{XrdTestMaster}\PYG{o}{.}\PYG{n}{pid}

\PYG{c}{\PYGZsh{} Path the the master's log file.}
\PYG{n}{log\PYGZus{}file\PYGZus{}path}\PYG{o}{=}\PYG{o}{/}\PYG{n}{var}\PYG{o}{/}\PYG{n}{log}\PYG{o}{/}\PYG{n}{XrdTest}\PYG{o}{/}\PYG{n}{XrdTestMaster}\PYG{o}{.}\PYG{n}{log}

\PYG{c}{\PYGZsh{} Amount of information to log. Constants from standard python logging module.}
\PYG{c}{\PYGZsh{} Defaults to INFO. Possible values: NOTSET (off), ERROR (only errors), WARN}
\PYG{c}{\PYGZsh{} (warnings and above), INFO (most logs), DEBUG (everything)}
\PYG{n}{log\PYGZus{}level}\PYG{o}{=}\PYG{n}{DEBUG}
\end{Verbatim}


\subsection{Other considerations}
\label{config-master:other-considerations}\begin{itemize}
\item {} 
Firewall (tcp on port 10000)

\end{itemize}


\section{Hypervisor configuration}
\label{config-hypervisor:hypervisor-configuration}\label{config-hypervisor::doc}

\subsection{Configuration sections}
\label{config-hypervisor:configuration-sections}

\subsubsection{\texttt{{[}test\_master{]}}}
\label{config-hypervisor:test-master}
\begin{Verbatim}[commandchars=\\\{\}]
\PYG{c}{\PYGZsh{} IP and port of the XrdTest Master.}
\PYG{n}{ip}\PYG{o}{=}\PYG{n}{somehost}\PYG{o}{.}\PYG{n}{somedomain}\PYG{o}{.}\PYG{n}{com}
\PYG{n}{port}\PYG{o}{=}\PYG{l+m+mi}{20000}

\PYG{c}{\PYGZsh{} Password to authenticate with the master.}
\PYG{n}{connection\PYGZus{}passwd}\PYG{o}{=}\PYG{n}{some\PYGZus{}passwd}
\end{Verbatim}


\subsubsection{\texttt{{[}virtual\_machines{]}}}
\label{config-hypervisor:virtual-machines}
\begin{Verbatim}[commandchars=\\\{\}]
\PYG{c}{\PYGZsh{} Path to the KVM executable.}
\PYG{c}{\PYGZsh{} emulator\PYGZus{}path=/usr/bin/kvm}
\PYG{n}{emulator\PYGZus{}path}\PYG{o}{=}\PYG{o}{/}\PYG{n}{usr}\PYG{o}{/}\PYG{n}{libexec}\PYG{o}{/}\PYG{n}{qemu}\PYG{o}{-}\PYG{n}{kvm}

\PYG{c}{\PYGZsh{} Name of the libvirt storage pool in which slave boot images will be placed.}
\PYG{c}{\PYGZsh{} You must configure this storage pool yourself, and place any boot images as}
\PYG{c}{\PYGZsh{} libvirt storage volumes into the pool. This pool can be anywhere (NAS, NFS}
\PYG{c}{\PYGZsh{} etc), as long as it is visible as a libvirt storage pool on this hypervisor.}
\PYG{n}{storage\PYGZus{}pool}\PYG{o}{=}\PYG{n}{XrdTest}
\end{Verbatim}


\subsubsection{\texttt{{[}security{]}}}
\label{config-hypervisor:security}
\begin{Verbatim}[commandchars=\\\{\}]
\PYG{c}{\PYGZsh{} Paths to SSL certificates and keys for the hypervisor.}
\PYG{n}{certfile}\PYG{o}{=}\PYG{o}{/}\PYG{n}{etc}\PYG{o}{/}\PYG{n}{XrdTest}\PYG{o}{/}\PYG{n}{certs}\PYG{o}{/}\PYG{n}{hypervisorcert}\PYG{o}{.}\PYG{n}{pem}
\PYG{n}{keyfile}\PYG{o}{=}\PYG{o}{/}\PYG{n}{etc}\PYG{o}{/}\PYG{n}{XrdTest}\PYG{o}{/}\PYG{n}{certs}\PYG{o}{/}\PYG{n}{hypervisorkey}\PYG{o}{.}\PYG{n}{pem}
\end{Verbatim}


\subsubsection{\texttt{{[}daemon{]}}}
\label{config-hypervisor:daemon}
\begin{Verbatim}[commandchars=\\\{\}]
\PYG{c}{\PYGZsh{} Path to the PID file for the hypervisor when running as daemon.}
\PYG{n}{pid\PYGZus{}file\PYGZus{}path}\PYG{o}{=}\PYG{o}{/}\PYG{n}{var}\PYG{o}{/}\PYG{n}{run}\PYG{o}{/}\PYG{n}{XrdTestHypervisor}\PYG{o}{.}\PYG{n}{pid}

\PYG{c}{\PYGZsh{} Where the hypervisor writes its logs}
\PYG{n}{log\PYGZus{}file\PYGZus{}path}\PYG{o}{=}\PYG{o}{/}\PYG{n}{var}\PYG{o}{/}\PYG{n}{log}\PYG{o}{/}\PYG{n}{XrdTest}\PYG{o}{/}\PYG{n}{XrdTestHypervisor}\PYG{o}{.}\PYG{n}{log}

\PYG{c}{\PYGZsh{} Amount of information to log. Constants from standard python logging module.}
\PYG{c}{\PYGZsh{} Defaults to INFO. Possible values: NOTSET (off), ERROR (only errors), WARN}
\PYG{c}{\PYGZsh{} (warnings and above), INFO (most logs), DEBUG (everything)}
\PYG{n}{log\PYGZus{}level}\PYG{o}{=}\PYG{n}{INFO}
\end{Verbatim}


\subsection{Other considerations}
\label{config-hypervisor:other-considerations}\begin{itemize}
\item {} 
Available memory, storage pool size

\end{itemize}


\section{Slave configuration}
\label{config-slave:slave-configuration}\label{config-slave::doc}

\subsection{Configuration sections}
\label{config-slave:configuration-sections}

\subsubsection{\texttt{{[}test\_master{]}}}
\label{config-slave:test-master}
\begin{Verbatim}[commandchars=\\\{\}]
\PYG{c}{\PYGZsh{} IP and port of the XrdTest Master. Slaves can set this to master.xrd.test,}
\PYG{c}{\PYGZsh{} as the virtual network will have a DNS entry which will resolve back to the}
\PYG{c}{\PYGZsh{} actual master IP.}
\PYG{n}{ip}\PYG{o}{=}\PYG{n}{master}\PYG{o}{.}\PYG{n}{xrd}\PYG{o}{.}\PYG{n}{test}
\PYG{n}{port}\PYG{o}{=}\PYG{l+m+mi}{20000}
\end{Verbatim}


\subsubsection{\texttt{{[}security{]}}}
\label{config-slave:security}
\begin{Verbatim}[commandchars=\\\{\}]
\PYG{c}{\PYGZsh{} Paths to SSL certificates and keys for the slave.}
\PYG{n}{certfile}\PYG{o}{=}\PYG{o}{/}\PYG{n}{etc}\PYG{o}{/}\PYG{n}{XrdTest}\PYG{o}{/}\PYG{n}{certs}\PYG{o}{/}\PYG{n}{slavecert}\PYG{o}{.}\PYG{n}{pem}
\PYG{n}{keyfile}\PYG{o}{=}\PYG{o}{/}\PYG{n}{etc}\PYG{o}{/}\PYG{n}{XrdTest}\PYG{o}{/}\PYG{n}{certs}\PYG{o}{/}\PYG{n}{slavekey}\PYG{o}{.}\PYG{n}{pem}
\end{Verbatim}


\subsubsection{\texttt{{[}daemon{]}}}
\label{config-slave:daemon}
\begin{Verbatim}[commandchars=\\\{\}]
\PYG{c}{\PYGZsh{} Path to the PID file for the slave when running as daemon.}
\PYG{n}{pid\PYGZus{}file\PYGZus{}path}\PYG{o}{=}\PYG{o}{/}\PYG{n}{var}\PYG{o}{/}\PYG{n}{run}\PYG{o}{/}\PYG{n}{XrdTestSlave}\PYG{o}{.}\PYG{n}{pid}

\PYG{c}{\PYGZsh{} Where the slave writes its logs}
\PYG{n}{log\PYGZus{}file\PYGZus{}path}\PYG{o}{=}\PYG{o}{/}\PYG{n}{var}\PYG{o}{/}\PYG{n}{log}\PYG{o}{/}\PYG{n}{XrdTest}\PYG{o}{/}\PYG{n}{XrdTestSlave}\PYG{o}{.}\PYG{n}{log}

\PYG{c}{\PYGZsh{} Amount of information to log. Constants from standard python logging module.}
\PYG{c}{\PYGZsh{} Defaults to INFO. Possible values: NOTSET (off), ERROR (only errors), WARN}
\PYG{c}{\PYGZsh{} (warnings and above), INFO (most logs), DEBUG (everything)}
\PYG{n}{log\PYGZus{}level}\PYG{o}{=}\PYG{n}{DEBUG}
\end{Verbatim}


\subsection{Other considerations}
\label{config-slave:other-considerations}\begin{itemize}
\item {} 
Slave image config (network, size, OS, root password etc)

\end{itemize}


\section{Writing test suites}
\label{testsuites::doc}\label{testsuites:writing-test-suites}
This page describes how to write test suites for the XrdTest Framework. For
details on how to set up a repository to hold your test suites, see
{\hyperref[config-master::doc]{\emph{Master configuration}}}.

\begin{notice}{note}{Note:}
Full examples can be found in the \code{examples} directory.
\end{notice}


\subsection{Structure of a test suite}
\label{testsuites:structure-of-a-test-suite}
Test suites have a specific structure which must be adhered to,
explained below:
\begin{itemize}
\item {} 
Each test suite resides in its own directory.

\item {} 
Each test suite has a definition file, which uses Python syntax. It is loaded
dynamically as a Python function at runtime, so it must be syntactically
correct. It must have a specific name (see {\hyperref[testsuites:def-file]{\emph{The definition file}}} below).

\item {} 
Each test suite has a mandatory global \textbf{initialization} script, which is
used to set up the (xrootd) environment ready for running test cases (see
{\hyperref[testsuites:scripts]{\emph{Writing initialization/run/finalization scripts}}} below).

\item {} 
Test suites can optionally have a global \textbf{finalization} script, generally
used to perform cleanup tasks, such as removing files from data servers,
removing authentication credentials etc. (see {\hyperref[testsuites:scripts]{\emph{Writing initialization/run/finalization scripts}}} below).

\item {} 
Each test suite has a subdirectory called \textbf{tc} which holds the set of
test cases for this suite.
\begin{itemize}
\item {} 
Each test case resides in its own subdirectory. The name of the directory
defines the name of the test case.

\item {} 
Each test case has a mandatory \textbf{initialization} script

\item {} 
Each test case has a mandatory \textbf{run} script.

\item {} 
Test cases can optionally have a \textbf{finalization} script.

\end{itemize}

\end{itemize}


\subsubsection{The definition file}
\label{testsuites:def-file}\label{testsuites:the-definition-file}
The test suite definition file must be in the root directory of the test suite
directory, and it must have the same name as the folder, with a \code{.py} extension.

A test suite is defined inside a function named \code{getTestSuite()} which takes
no parameters. Here is an example of the beginning of a definition file:

\begin{Verbatim}[commandchars=\\\{\}]
\PYG{k+kn}{from} \PYG{n+nn}{XrdTest.TestUtils} \PYG{k+kn}{import} \PYG{n}{TestSuite}

\PYG{k}{def} \PYG{n+nf}{getTestSuite}\PYG{p}{(}\PYG{p}{)}\PYG{p}{:}

    \PYG{n}{ts} \PYG{o}{=} \PYG{n}{TestSuite}\PYG{p}{(}\PYG{p}{)}
    \PYG{n}{ts}\PYG{o}{.}\PYG{n}{name} \PYG{o}{=} \PYG{l+s}{"}\PYG{l+s}{ts\PYGZus{}002\PYGZus{}frm}\PYG{l+s}{"}
\end{Verbatim}

The \code{ts.name} attribute is the unique name of the test suite. It must match
the name of the file exactly (minus the \emph{.py} extension) and also match the
directory name in which this test suite resides.

The test suite name is arbitrary, but in the CERN \code{xrootd-testsuite}
repository we have a naming convention of \code{ts\_\textless{}numerical id\textgreater{}\_\textless{}shorthand
description\textgreater{}}. For example, the suite which tests GSI functionality is named
\textbf{ts\_006\_gsi}. You are of course free to choose your own naming conventions,
however.


\paragraph{Defining required clusters}
\label{testsuites:defining-required-clusters}
To define the cluster(s) which this test suite requires, include a line like
this:

\begin{Verbatim}[commandchars=\\\{\}]
\PYG{n}{ts}\PYG{o}{.}\PYG{n}{clusters} \PYG{o}{=} \PYG{p}{[}\PYG{l+s}{'}\PYG{l+s}{cluster\PYGZus{}002\PYGZus{}frm}\PYG{l+s}{'}\PYG{p}{]}
\end{Verbatim}

This line is mandatory. Currently, there is only support for one cluster per
test suite. It is planned to have the ability to run multiple clusters on
multiple hypervisors in the future. For information on how to define a cluster,
see {\hyperref[clusters::doc]{\emph{Writing cluster definitions}}}.

There is also the ability to specify a subset of the machines in a cluster,
with a line like this:

\begin{Verbatim}[commandchars=\\\{\}]
\PYG{n}{ts}\PYG{o}{.}\PYG{n}{machines} \PYG{o}{=} \PYG{p}{[}\PYG{l+s}{'}\PYG{l+s}{frm1}\PYG{l+s}{'}\PYG{p}{,} \PYG{l+s}{'}\PYG{l+s}{frm2}\PYG{l+s}{'}\PYG{p}{,} \PYG{l+s}{'}\PYG{l+s}{ds1}\PYG{l+s}{'}\PYG{p}{,} \PYG{l+s}{'}\PYG{l+s}{ds2}\PYG{l+s}{'}\PYG{p}{,} \PYG{l+s}{'}\PYG{l+s}{client1}\PYG{l+s}{'}\PYG{p}{]}
\end{Verbatim}

There haven't been any use cases where this has been needed yet, but the
functionality exists if one comes along. This line os not mandatory.


\paragraph{Scheduling test suites}
\label{testsuites:scheduling-test-suites}
To schedule the test suite to be run at particular intervals (cron-style), you
must include a line like this:

\begin{Verbatim}[commandchars=\\\{\}]
\PYG{n}{ts}\PYG{o}{.}\PYG{n}{schedule} \PYG{o}{=} \PYG{n+nb}{dict}\PYG{p}{(}\PYG{n}{second}\PYG{o}{=}\PYG{l+s}{'}\PYG{l+s}{30}\PYG{l+s}{'}\PYG{p}{,} \PYG{n}{minute}\PYG{o}{=}\PYG{l+s}{'}\PYG{l+s}{08}\PYG{l+s}{'}\PYG{p}{,} \PYG{n}{hour}\PYG{o}{=}\PYG{l+s}{'}\PYG{l+s}{*}\PYG{l+s}{'}\PYG{p}{,} \PYG{n}{day}\PYG{o}{=}\PYG{l+s}{'}\PYG{l+s}{*}\PYG{l+s}{'}\PYG{p}{,} \PYG{n}{month}\PYG{o}{=}\PYG{l+s}{'}\PYG{l+s}{1}\PYG{l+s}{'}\PYG{p}{)}
\end{Verbatim}

This line is mandatory.


\paragraph{Defining what is run}
\label{testsuites:defining-what-is-run}
To define which test cases will be run in this suite, include a line similar
to this:

\begin{Verbatim}[commandchars=\\\{\}]
\PYG{n}{ts}\PYG{o}{.}\PYG{n}{tests} \PYG{o}{=} \PYG{p}{[}\PYG{l+s}{'}\PYG{l+s}{copy\PYGZus{}to}\PYG{l+s}{'}\PYG{p}{,} \PYG{l+s}{'}\PYG{l+s}{copy\PYGZus{}from}\PYG{l+s}{'}\PYG{p}{]}
\end{Verbatim}

This line is mandatory. If a test case exists in the \textbf{tc} directory, but is
not included in the line in the definition file, it will not be run.

To point to the suite initialization script, include a line like this:

\begin{Verbatim}[commandchars=\\\{\}]
\PYG{n}{ts}\PYG{o}{.}\PYG{n}{initialize} \PYG{o}{=} \PYG{l+s}{"}\PYG{l+s}{file://suite\PYGZus{}init.sh}\PYG{l+s}{"}
\end{Verbatim}

This line can be a relative file URL (as above), an absolute file URL, or a
HTTP URL. The initialization script is mandatory.

To point to the suite finalization script, include a line like this:

\begin{Verbatim}[commandchars=\\\{\}]
\PYG{n}{ts}\PYG{o}{.}\PYG{n}{finalize} \PYG{o}{=} \PYG{l+s}{"}\PYG{l+s}{file://suite\PYGZus{}finalize.sh}\PYG{l+s}{"}
\end{Verbatim}

The finalization script is not mandatory. It can be used for general cleaning
up after all test cases have been run.


\paragraph{Including log files}
\label{testsuites:including-log-files}
The framework has functionality for retrieving arbitrary log files from each
slave at each stage of the test suite. To use this feature, include a line
like this:

\begin{Verbatim}[commandchars=\\\{\}]
\PYG{n}{ts}\PYG{o}{.}\PYG{n}{logs} \PYG{o}{=} \PYG{p}{[}\PYG{l+s}{'}\PYG{l+s}{/var/log/xrootd/@slavename@/xrootd.log}\PYG{l+s}{'}\PYG{p}{,}
           \PYG{l+s}{'}\PYG{l+s}{/var/log/xrootd/@slavename@/cmsd.log}\PYG{l+s}{'}\PYG{p}{,}
           \PYG{l+s}{'}\PYG{l+s}{/var/log/XrdTest/XrdTestSlave.log}\PYG{l+s}{'}\PYG{p}{]}
\end{Verbatim}

You should provide the path to any log files which will be useful to inspect.
It is possible to use the @slavename@ tag in the log file path (See
{\hyperref[testsuites:tagging]{\emph{The @tag@ system}}} for an explanation of the @slavename@ and other tags). It can
be useful to include the slave log (XrdTestSlave.log) for debugging purposes.


\paragraph{Getting email alerts}
\label{testsuites:getting-email-alerts}
It is possible to give an arbitrary list of email addresses, each of which can
be notified of the outcome of a test suite run, to a specified level of verbosity.

The list of email addresses is defined with a line like the following:

\begin{Verbatim}[commandchars=\\\{\}]
\PYG{n}{ts}\PYG{o}{.}\PYG{n}{alert\PYGZus{}emails} \PYG{o}{=} \PYG{p}{[}\PYG{l+s}{'}\PYG{l+s}{jsalmon@cern.ch}\PYG{l+s}{'}\PYG{p}{,} \PYG{l+s}{'}\PYG{l+s}{foo@bar.com}\PYG{l+s}{'}\PYG{p}{]}
\end{Verbatim}

The amount of email alerts to be sent is configured with policy lines like the
following:

\begin{Verbatim}[commandchars=\\\{\}]
\PYG{n}{ts}\PYG{o}{.}\PYG{n}{alert\PYGZus{}success} \PYG{o}{=} \PYG{l+s}{'}\PYG{l+s}{SUITE}\PYG{l+s}{'}
\PYG{n}{ts}\PYG{o}{.}\PYG{n}{alert\PYGZus{}failure} \PYG{o}{=} \PYG{l+s}{'}\PYG{l+s}{CASE}\PYG{l+s}{'}
\end{Verbatim}

There is a separate policy for failure notifications and success notifications
for flexibility. The possible options for both policies are:
\begin{itemize}
\item {} 
\code{SUITE} - Send an email about the final state of the entire test suite
(success or failure).

\item {} 
\code{CASE} - Send an email about the final state of each individual test case
(success or failure). Implied SUITE.

\item {} 
\code{NONE} - Don't send any emails.

\end{itemize}

The default options are generally OK, i.e. \code{CASE} for failure alerts (as you
want to know if the test suite failed and also which individual test cases failed)
and \code{SUITE} for success alerts (you don't care if each test case succeeds, only
that the whole suite succeeds). You might want to put \code{NONE} for the success
policy if you really only care about failures.


\subsection{Writing initialization/run/finalization scripts}
\label{testsuites:writing-initialization-run-finalization-scripts}\label{testsuites:scripts}
As mentioned earlier, each test suite has a mandatory global initialization
script, an optional global finalization script, and a set if initialization/
run/finalization scripts for each test case.

The framework has been designed in this way, so that actions can be synchronized
between participants (slaves) in the cluster. For example, if a slave completes
its global initialization script, it will wait for all other slaves to complete
theirs before moving on to the next stage. Similarly, a slave will not begin the
run stage of a test case until it and all other slaves have completed the test
case initialization stage. The XrdTest Master is actually responsible for
orchestrating this activity.

\textbf{It is important to note that} should the global initialization script fail
on any slave for any reason, then the \textbf{entire test suite} will be considered
as failed, and no test cases will be run. A command that returns a non-zero
exit code is considered as a failure, unless specifically stated otherwise by
using the \code{assert\_fail} function (see {\hyperref[testsuites:functions]{\emph{Available functions}}} below).

If a \textbf{test case} initialization script fails, the suite will continue to run.
The same is true for the remaining stages of the suite.

Also note that you do not need to worry about \emph{stdout} and \emph{stderr}. Anything
that is printed to \emph{stderr} will be redirected to \emph{stdout}. This is due to
both ease of use, and to problems with Python's \code{subprocess} module and the
way it handles \emph{stderr}.

The framework provides some features to make the scripts more flexible,explained
below.


\subsubsection{The \texttt{@tag@} system}
\label{testsuites:the-tag-system}\label{testsuites:tagging}
There are some special keywords which can be used inside any test suite script.
These keywords, or \emph{tags}, have a descriptive name enclosed with \code{@} symbols.
Each tag within a script will be replaced with an appropriate real value at
runtime, based upon which slave is currently running the script, the cluster
configuration, and the parameters with which the master is to be contacted.

The currently available tags are as follows:
\begin{itemize}
\item {} 
\code{@slavename@} - The FQDN of the current slave running the script. This allows
one to write a single script containing if/else blocks to determine which piece
of code the current slave will run, based upon its name.

\item {} 
\code{@port@} - The port on which the master should be contacted (defined in the
master configuration file, see {\hyperref[config-master::doc]{\emph{Master configuration}}}).

\item {} 
\code{@proto@} - The protocol by which the master should be contacted (defined in
the master configuration file, see {\hyperref[config-master::doc]{\emph{Master configuration}}}).

\item {} 
\code{@diskmounts@} - Gets resolved to the appropriate disk mount command(s) for
the current slave. Disks are always mounted as \code{ext4} using \code{user\_xattr}.

\end{itemize}

It is planned to allow user extensions to the tagging system sometime in the
future, so that arbitrary tags can be used inside scripts for even greater
flexibility.


\subsubsection{Available functions}
\label{testsuites:functions}\label{testsuites:available-functions}
There is a small library of functions (located in \emph{/etc/XrdTest/utils}) that can
be used by default in test scripts. To use these functions, simply source the
file inside the script like this:

\begin{Verbatim}[commandchars=\\\{\}]
\PYG{c}{\PYGZsh{}!/bin/bash}
\PYG{n}{source} \PYG{o}{/}\PYG{n}{etc}\PYG{o}{/}\PYG{n}{XrdTest}\PYG{o}{/}\PYG{n}{utils}\PYG{o}{/}\PYG{n}{functions}\PYG{o}{.}\PYG{n}{sh}
\end{Verbatim}

A brief description of the currently available functions:
\begin{itemize}
\item {} 
\code{assert\_fail} - a function to assert the non-zero exit code of a function.
Used for testing invalid use cases and verifiying that they fail as they should.
For example:

\begin{Verbatim}[commandchars=\\\{\}]
assert\_fail rm non\_existent
\end{Verbatim}

will return zero and not cause the script to exit (as would have happened if
the \code{assert\_fail} command were not used).

\item {} 
\code{log} - Used for timestamping and printing single-line commands or progress
messages. For example:

\begin{Verbatim}[commandchars=\\\{\}]
log "Initializing test suite on slave @slavename@"
\end{Verbatim}

will print a timestamped line in the session log which looks like this:

\begin{Verbatim}[commandchars=\\\{\}]
[10:49:20] Initializing test suite on slave manager1.xrd.test
\end{Verbatim}

\item {} 
\code{stamp} - Used for timestamping and printing entire command outputs. For
example:

\begin{Verbatim}[commandchars=\\\{\}]
stamp ls -al /data
\end{Verbatim}

will produce output like this:

\begin{Verbatim}[commandchars=\\\{\}]
[09:57:37]  total 51208
[09:57:37]  drwxr-xr-x.  2 daemon daemon     4096 Oct 22 09:57 .
[09:57:37]  dr-xr-xr-x. 25 root   root       4096 Oct 22 09:52 ..
[09:57:37]  -rw-r--r--.  1 root   root   52428800 Oct 22 09:57 some\_file
\end{Verbatim}

\end{itemize}


\section{Writing cluster definitions}
\label{clusters::doc}\label{clusters:writing-cluster-definitions}
\begin{Verbatim}[commandchars=\\\{\}]
\PYG{k+kn}{from} \PYG{n+nn}{XrdTest.ClusterUtils} \PYG{k+kn}{import} \PYG{n}{Cluster}\PYG{p}{,} \PYG{n}{Network}\PYG{p}{,} \PYG{n}{Host}\PYG{p}{,} \PYG{n}{Disk}

\PYG{k}{def} \PYG{n+nf}{getCluster}\PYG{p}{(}\PYG{p}{)}\PYG{p}{:}
    \PYG{n}{cluster} \PYG{o}{=} \PYG{n}{Cluster}\PYG{p}{(}\PYG{p}{)}
    \PYG{c}{\PYGZsh{}---------------------------------------------------------------------------}
    \PYG{c}{\PYGZsh{} Global names}
    \PYG{c}{\PYGZsh{}---------------------------------------------------------------------------}
    \PYG{n}{cluster}\PYG{o}{.}\PYG{n}{name} \PYG{o}{=} \PYG{l+s}{'}\PYG{l+s}{cluster\PYGZus{}example}\PYG{l+s}{'}
    \PYG{n}{network\PYGZus{}name} \PYG{o}{=} \PYG{n}{cluster}\PYG{o}{.}\PYG{n}{name} \PYG{o}{+} \PYG{l+s}{'}\PYG{l+s}{\PYGZus{}net}\PYG{l+s}{'}

    \PYG{c}{\PYGZsh{}---------------------------------------------------------------------------}
    \PYG{c}{\PYGZsh{} Cluster defaults}
    \PYG{c}{\PYGZsh{}}
    \PYG{c}{\PYGZsh{} The bootImage parameter is relative to some libvirt-managed storage pool.}
    \PYG{c}{\PYGZsh{}---------------------------------------------------------------------------}
    \PYG{n}{cluster}\PYG{o}{.}\PYG{n}{defaultHost}\PYG{o}{.}\PYG{n}{bootImage} \PYG{o}{=} \PYG{l+s}{'}\PYG{l+s}{slc6\PYGZus{}testslave\PYGZus{}ref.img}\PYG{l+s}{'}
    \PYG{n}{cluster}\PYG{o}{.}\PYG{n}{defaultHost}\PYG{o}{.}\PYG{n}{cacheBootImage} \PYG{o}{=} \PYG{n+nb+bp}{True}
    \PYG{n}{cluster}\PYG{o}{.}\PYG{n}{defaultHost}\PYG{o}{.}\PYG{n}{arch} \PYG{o}{=} \PYG{l+s}{'}\PYG{l+s}{x86\PYGZus{}64}\PYG{l+s}{'}
    \PYG{n}{cluster}\PYG{o}{.}\PYG{n}{defaultHost}\PYG{o}{.}\PYG{n}{ramSize} \PYG{o}{=} \PYG{l+s}{'}\PYG{l+s}{1048576}\PYG{l+s}{'}
    \PYG{n}{cluster}\PYG{o}{.}\PYG{n}{defaultHost}\PYG{o}{.}\PYG{n}{net} \PYG{o}{=} \PYG{n}{network\PYGZus{}name}

    \PYG{c}{\PYGZsh{}---------------------------------------------------------------------------}
    \PYG{c}{\PYGZsh{} Host definitions}
    \PYG{c}{\PYGZsh{}---------------------------------------------------------------------------}
    \PYG{n}{metamanager1} \PYG{o}{=} \PYG{n}{Host}\PYG{p}{(}\PYG{l+s}{'}\PYG{l+s}{metamanager1.xrd.test}\PYG{l+s}{'}\PYG{p}{,} \PYG{l+s}{'}\PYG{l+s}{192.168.127.3}\PYG{l+s}{'}\PYG{p}{,} \PYG{l+s}{"}\PYG{l+s}{52:54:00:65:44:65}\PYG{l+s}{"}\PYG{p}{)}
    \PYG{n}{manager1} \PYG{o}{=} \PYG{n}{Host}\PYG{p}{(}\PYG{l+s}{'}\PYG{l+s}{manager1.xrd.test}\PYG{l+s}{'}\PYG{p}{,} \PYG{l+s}{'}\PYG{l+s}{192.168.127.4}\PYG{l+s}{'}\PYG{p}{,} \PYG{l+s}{"}\PYG{l+s}{52:54:00:65:44:66}\PYG{l+s}{"}\PYG{p}{)}
    \PYG{n}{manager2} \PYG{o}{=} \PYG{n}{Host}\PYG{p}{(}\PYG{l+s}{'}\PYG{l+s}{manager2.xrd.test}\PYG{l+s}{'}\PYG{p}{,} \PYG{l+s}{'}\PYG{l+s}{192.168.127.5}\PYG{l+s}{'}\PYG{p}{,} \PYG{l+s}{"}\PYG{l+s}{52:54:00:65:44:67}\PYG{l+s}{"}\PYG{p}{)}
    \PYG{n}{ds1} \PYG{o}{=} \PYG{n}{Host}\PYG{p}{(}\PYG{l+s}{'}\PYG{l+s}{ds1.xrd.test}\PYG{l+s}{'}\PYG{p}{,} \PYG{l+s}{'}\PYG{l+s}{192.168.127.6}\PYG{l+s}{'}\PYG{p}{,} \PYG{l+s}{"}\PYG{l+s}{52:54:00:65:44:68}\PYG{l+s}{"}\PYG{p}{)}
    \PYG{n}{ds2} \PYG{o}{=} \PYG{n}{Host}\PYG{p}{(}\PYG{l+s}{'}\PYG{l+s}{ds2.xrd.test}\PYG{l+s}{'}\PYG{p}{,} \PYG{l+s}{'}\PYG{l+s}{192.168.127.7}\PYG{l+s}{'}\PYG{p}{,} \PYG{l+s}{"}\PYG{l+s}{52:54:00:65:44:69}\PYG{l+s}{"}\PYG{p}{)}
    \PYG{n}{ds3} \PYG{o}{=} \PYG{n}{Host}\PYG{p}{(}\PYG{l+s}{'}\PYG{l+s}{ds3.xrd.test}\PYG{l+s}{'}\PYG{p}{,} \PYG{l+s}{'}\PYG{l+s}{192.168.127.8}\PYG{l+s}{'}\PYG{p}{,} \PYG{l+s}{"}\PYG{l+s}{52:54:00:65:44:70}\PYG{l+s}{"}\PYG{p}{)}
    \PYG{n}{ds4} \PYG{o}{=} \PYG{n}{Host}\PYG{p}{(}\PYG{l+s}{'}\PYG{l+s}{ds4.xrd.test}\PYG{l+s}{'}\PYG{p}{,} \PYG{l+s}{'}\PYG{l+s}{192.168.127.9}\PYG{l+s}{'}\PYG{p}{,} \PYG{l+s}{"}\PYG{l+s}{52:54:00:65:44:71}\PYG{l+s}{"}\PYG{p}{)}
    \PYG{n}{client1} \PYG{o}{=} \PYG{n}{Host}\PYG{p}{(}\PYG{l+s}{'}\PYG{l+s}{client1.xrd.test}\PYG{l+s}{'}\PYG{p}{,} \PYG{l+s}{'}\PYG{l+s}{192.168.127.10}\PYG{l+s}{'}\PYG{p}{,} \PYG{l+s}{"}\PYG{l+s}{52:54:00:65:44:72}\PYG{l+s}{"}\PYG{p}{)}

    \PYG{c}{\PYGZsh{}---------------------------------------------------------------------------}
    \PYG{c}{\PYGZsh{} Additional host disk definitions}
    \PYG{c}{\PYGZsh{}}
    \PYG{c}{\PYGZsh{} As per the libvirt docs, the device name given here is not guaranteed to}
    \PYG{c}{\PYGZsh{} map to the same name in the guest OS. Incrementing the device name works}
    \PYG{c}{\PYGZsh{} (i.e. disk1 = vda, disk2 = vdb etc.).}
    \PYG{c}{\PYGZsh{}}
    \PYG{c}{\PYGZsh{} Disk sizes should be larger than 10GB for data server nodes, otherwise}
    \PYG{c}{\PYGZsh{} the node might not be selected by the cmsd.}
    \PYG{c}{\PYGZsh{}---------------------------------------------------------------------------}
    \PYG{n}{metamanager1}\PYG{o}{.}\PYG{n}{disks} \PYG{o}{=}  \PYG{p}{[}\PYG{n}{Disk}\PYG{p}{(}\PYG{l+s}{'}\PYG{l+s}{disk1}\PYG{l+s}{'}\PYG{p}{,} \PYG{l+s}{'}\PYG{l+s}{20G}\PYG{l+s}{'}\PYG{p}{,} \PYG{n}{device}\PYG{o}{=}\PYG{l+s}{'}\PYG{l+s}{vda}\PYG{l+s}{'}\PYG{p}{,} \PYG{n}{mountPoint}\PYG{o}{=}\PYG{l+s}{'}\PYG{l+s}{/data}\PYG{l+s}{'}\PYG{p}{)}\PYG{p}{]}
    \PYG{n}{manager1}\PYG{o}{.}\PYG{n}{disks} \PYG{o}{=}  \PYG{p}{[}\PYG{n}{Disk}\PYG{p}{(}\PYG{l+s}{'}\PYG{l+s}{disk1}\PYG{l+s}{'}\PYG{p}{,} \PYG{l+s}{'}\PYG{l+s}{20G}\PYG{l+s}{'}\PYG{p}{,} \PYG{n}{device}\PYG{o}{=}\PYG{l+s}{'}\PYG{l+s}{vda}\PYG{l+s}{'}\PYG{p}{,} \PYG{n}{mountPoint}\PYG{o}{=}\PYG{l+s}{'}\PYG{l+s}{/data}\PYG{l+s}{'}\PYG{p}{)}\PYG{p}{]}
    \PYG{n}{manager2}\PYG{o}{.}\PYG{n}{disks} \PYG{o}{=}  \PYG{p}{[}\PYG{n}{Disk}\PYG{p}{(}\PYG{l+s}{'}\PYG{l+s}{disk1}\PYG{l+s}{'}\PYG{p}{,} \PYG{l+s}{'}\PYG{l+s}{20G}\PYG{l+s}{'}\PYG{p}{,} \PYG{n}{device}\PYG{o}{=}\PYG{l+s}{'}\PYG{l+s}{vda}\PYG{l+s}{'}\PYG{p}{,} \PYG{n}{mountPoint}\PYG{o}{=}\PYG{l+s}{'}\PYG{l+s}{/data}\PYG{l+s}{'}\PYG{p}{)}\PYG{p}{]}
    \PYG{n}{ds1}\PYG{o}{.}\PYG{n}{disks} \PYG{o}{=}  \PYG{p}{[}\PYG{n}{Disk}\PYG{p}{(}\PYG{l+s}{'}\PYG{l+s}{disk1}\PYG{l+s}{'}\PYG{p}{,} \PYG{l+s}{'}\PYG{l+s}{20G}\PYG{l+s}{'}\PYG{p}{,} \PYG{n}{device}\PYG{o}{=}\PYG{l+s}{'}\PYG{l+s}{vda}\PYG{l+s}{'}\PYG{p}{,} \PYG{n}{mountPoint}\PYG{o}{=}\PYG{l+s}{'}\PYG{l+s}{/data}\PYG{l+s}{'}\PYG{p}{)}\PYG{p}{]}
    \PYG{n}{ds2}\PYG{o}{.}\PYG{n}{disks} \PYG{o}{=}  \PYG{p}{[}\PYG{n}{Disk}\PYG{p}{(}\PYG{l+s}{'}\PYG{l+s}{disk1}\PYG{l+s}{'}\PYG{p}{,} \PYG{l+s}{'}\PYG{l+s}{20G}\PYG{l+s}{'}\PYG{p}{,} \PYG{n}{device}\PYG{o}{=}\PYG{l+s}{'}\PYG{l+s}{vda}\PYG{l+s}{'}\PYG{p}{,} \PYG{n}{mountPoint}\PYG{o}{=}\PYG{l+s}{'}\PYG{l+s}{/data}\PYG{l+s}{'}\PYG{p}{)}\PYG{p}{]}
    \PYG{n}{ds3}\PYG{o}{.}\PYG{n}{disks} \PYG{o}{=}  \PYG{p}{[}\PYG{n}{Disk}\PYG{p}{(}\PYG{l+s}{'}\PYG{l+s}{disk1}\PYG{l+s}{'}\PYG{p}{,} \PYG{l+s}{'}\PYG{l+s}{20G}\PYG{l+s}{'}\PYG{p}{,} \PYG{n}{device}\PYG{o}{=}\PYG{l+s}{'}\PYG{l+s}{vda}\PYG{l+s}{'}\PYG{p}{,} \PYG{n}{mountPoint}\PYG{o}{=}\PYG{l+s}{'}\PYG{l+s}{/data}\PYG{l+s}{'}\PYG{p}{)}\PYG{p}{]}
    \PYG{n}{ds4}\PYG{o}{.}\PYG{n}{disks} \PYG{o}{=}  \PYG{p}{[}\PYG{n}{Disk}\PYG{p}{(}\PYG{l+s}{'}\PYG{l+s}{disk1}\PYG{l+s}{'}\PYG{p}{,} \PYG{l+s}{'}\PYG{l+s}{20G}\PYG{l+s}{'}\PYG{p}{,} \PYG{n}{device}\PYG{o}{=}\PYG{l+s}{'}\PYG{l+s}{vda}\PYG{l+s}{'}\PYG{p}{,} \PYG{n}{mountPoint}\PYG{o}{=}\PYG{l+s}{'}\PYG{l+s}{/data}\PYG{l+s}{'}\PYG{p}{)}\PYG{p}{]}
    \PYG{n}{client1}\PYG{o}{.}\PYG{n}{disks} \PYG{o}{=}  \PYG{p}{[}\PYG{n}{Disk}\PYG{p}{(}\PYG{l+s}{'}\PYG{l+s}{disk1}\PYG{l+s}{'}\PYG{p}{,} \PYG{l+s}{'}\PYG{l+s}{20G}\PYG{l+s}{'}\PYG{p}{,} \PYG{n}{device}\PYG{o}{=}\PYG{l+s}{'}\PYG{l+s}{vda}\PYG{l+s}{'}\PYG{p}{,} \PYG{n}{mountPoint}\PYG{o}{=}\PYG{l+s}{'}\PYG{l+s}{/data}\PYG{l+s}{'}\PYG{p}{)}\PYG{p}{]}

    \PYG{c}{\PYGZsh{}---------------------------------------------------------------------------}
    \PYG{c}{\PYGZsh{} Network definition}
    \PYG{c}{\PYGZsh{}---------------------------------------------------------------------------}
    \PYG{n}{net} \PYG{o}{=} \PYG{n}{Network}\PYG{p}{(}\PYG{p}{)}
    \PYG{n}{net}\PYG{o}{.}\PYG{n}{bridgeName} \PYG{o}{=} \PYG{l+s}{'}\PYG{l+s}{virbr\PYGZus{}example}\PYG{l+s}{'}
    \PYG{n}{net}\PYG{o}{.}\PYG{n}{name} \PYG{o}{=} \PYG{n}{network\PYGZus{}name}
    \PYG{n}{net}\PYG{o}{.}\PYG{n}{ip} \PYG{o}{=} \PYG{l+s}{'}\PYG{l+s}{192.168.127.1}\PYG{l+s}{'}
    \PYG{n}{net}\PYG{o}{.}\PYG{n}{netmask} \PYG{o}{=} \PYG{l+s}{'}\PYG{l+s}{255.255.255.0}\PYG{l+s}{'}
    \PYG{n}{net}\PYG{o}{.}\PYG{n}{DHCPRange} \PYG{o}{=} \PYG{p}{(}\PYG{l+s}{'}\PYG{l+s}{192.168.127.2}\PYG{l+s}{'}\PYG{p}{,} \PYG{l+s}{'}\PYG{l+s}{192.168.127.254}\PYG{l+s}{'}\PYG{p}{)}
    \PYG{c}{\PYGZsh{}---------------------------------------------------------------------------}
    \PYG{c}{\PYGZsh{} Optional load balancing configuration}
    \PYG{c}{\PYGZsh{}---------------------------------------------------------------------------}
    \PYG{c}{\PYGZsh{} The DNS alias to be used}
    \PYG{n}{net}\PYG{o}{.}\PYG{n}{lbAlias} \PYG{o}{=} \PYG{l+s}{'}\PYG{l+s}{lb.xrd.test}\PYG{l+s}{'}
    \PYG{c}{\PYGZsh{} The machines that will be load balanced (round-robin) under the alias}
    \PYG{n}{net}\PYG{o}{.}\PYG{n}{lbHosts} \PYG{o}{=} \PYG{p}{[}\PYG{n}{ds1}\PYG{p}{,} \PYG{n}{ds2}\PYG{p}{,} \PYG{n}{ds3}\PYG{p}{,} \PYG{n}{ds4}\PYG{p}{]}

    \PYG{c}{\PYGZsh{} Hosts to be included in the cluster}
    \PYG{n}{hosts} \PYG{o}{=} \PYG{p}{[}\PYG{n}{metamanager1}\PYG{p}{,} \PYG{n}{manager1}\PYG{p}{,} \PYG{n}{manager2}\PYG{p}{,} \PYG{n}{ds1}\PYG{p}{,} \PYG{n}{ds2}\PYG{p}{,} \PYG{n}{ds3}\PYG{p}{,} \PYG{n}{ds4}\PYG{p}{,} \PYG{n}{client1}\PYG{p}{]}

    \PYG{n}{net}\PYG{o}{.}\PYG{n}{addHosts}\PYG{p}{(}\PYG{n}{hosts}\PYG{p}{)}
    \PYG{n}{cluster}\PYG{o}{.}\PYG{n}{network} \PYG{o}{=} \PYG{n}{net}
    \PYG{n}{cluster}\PYG{o}{.}\PYG{n}{addHosts}\PYG{p}{(}\PYG{n}{hosts}\PYG{p}{)}
    \PYG{k}{return} \PYG{n}{cluster}
\end{Verbatim}


\section{Using the XrdTest Web Interface}
\label{web-interface:using-the-xrdtest-web-interface}\label{web-interface::doc}\begin{itemize}
\item {} 
Available method calls

\item {} 
Running test suites manually (password)

\item {} 
Activating test suites

\item {} 
CA

\item {} 
SSS distribution

\end{itemize}


\section{Source code reference manual}
\label{ref-manual::doc}\label{ref-manual:source-code-reference-manual}

\subsection{XrdTest Package}
\label{ref-manual/XrdTest:xrdtest-package}\label{ref-manual/XrdTest::doc}

\subsubsection{\texttt{XrdTest} Package}
\label{ref-manual/XrdTest:id1}\phantomsection\label{ref-manual/XrdTest:module-XrdTest}\index{XrdTest (module)}
Library files for the XrdTest Framework


\subsubsection{\texttt{ClusterManager} Module}
\label{ref-manual/XrdTest:clustermanager-module}\label{ref-manual/XrdTest:module-XrdTest.ClusterManager}\index{XrdTest.ClusterManager (module)}\phantomsection\label{ref-manual/XrdTest:module-ClusterManager}\index{ClusterManager (module)}\index{ClusterManager (class in XrdTest.ClusterManager)}

\begin{fulllineitems}
\phantomsection\label{ref-manual/XrdTest:XrdTest.ClusterManager.ClusterManager}\pysigline{\strong{class }\code{XrdTest.ClusterManager.}\bfcode{ClusterManager}}
Virtual machines clusters' manager
\index{attachDisk() (XrdTest.ClusterManager.ClusterManager method)}

\begin{fulllineitems}
\phantomsection\label{ref-manual/XrdTest:XrdTest.ClusterManager.ClusterManager.attachDisk}\pysiglinewithargsret{\bfcode{attachDisk}}{\emph{host}, \emph{diskName}, \emph{diskSize}, \emph{cache}, \emph{device}}{}
TODO:

\end{fulllineitems}

\index{attachDisks() (XrdTest.ClusterManager.ClusterManager method)}

\begin{fulllineitems}
\phantomsection\label{ref-manual/XrdTest:XrdTest.ClusterManager.ClusterManager.attachDisks}\pysiglinewithargsret{\bfcode{attachDisks}}{\emph{host}}{}
\end{fulllineitems}

\index{copyImg() (XrdTest.ClusterManager.ClusterManager method)}

\begin{fulllineitems}
\phantomsection\label{ref-manual/XrdTest:XrdTest.ClusterManager.ClusterManager.copyImg}\pysiglinewithargsret{\bfcode{copyImg}}{\emph{huName}, \emph{safeCounter=None}}{}
Method runnable in separate threads, to separate copying of source
operating system image to a temporary image.

@param huName: host.uname - host unique name
@param safeCounter: thread safe counter to signalize this run finished

\end{fulllineitems}

\index{createCluster() (XrdTest.ClusterManager.ClusterManager method)}

\begin{fulllineitems}
\phantomsection\label{ref-manual/XrdTest:XrdTest.ClusterManager.ClusterManager.createCluster}\pysiglinewithargsret{\bfcode{createCluster}}{\emph{cluster}}{}
Creates cluster: first network, then virtual machines - hosts.
If it get request to create machine (host) that already exists (with the
same name) - it removes it completely. The same story regards network.
@param cluster: cluster definition object

\end{fulllineitems}

\index{createNetwork() (XrdTest.ClusterManager.ClusterManager method)}

\begin{fulllineitems}
\phantomsection\label{ref-manual/XrdTest:XrdTest.ClusterManager.ClusterManager.createNetwork}\pysiglinewithargsret{\bfcode{createNetwork}}{\emph{networkObj}, \emph{clusterName}}{}
Creates and starts cluster's network. It utilizes defineNetwork
at first and doesn't create network definition if it already exists.
@param networkObj: Network object
@raise ClusterManagerException: when fails
@return: None

\end{fulllineitems}

\index{defineHost() (XrdTest.ClusterManager.ClusterManager method)}

\begin{fulllineitems}
\phantomsection\label{ref-manual/XrdTest:XrdTest.ClusterManager.ClusterManager.defineHost}\pysiglinewithargsret{\bfcode{defineHost}}{\emph{host}}{}
Defines virtual host in a cluster using given host object,
not starting it. Host with the given name may be defined once
in the system. Stores hosts objects in class property self.hosts.

@param host: ClusterManager.Host object
@raise ClusterManagerException: when fails
@return: host object from libvirt lib

\end{fulllineitems}

\index{defineNetwork() (XrdTest.ClusterManager.ClusterManager method)}

\begin{fulllineitems}
\phantomsection\label{ref-manual/XrdTest:XrdTest.ClusterManager.ClusterManager.defineNetwork}\pysiglinewithargsret{\bfcode{defineNetwork}}{\emph{netObj}}{}
Defines network object without starting it.
@param xml: libvirt XML cluster definition
@raise ClusterManagerException: when fails
@return: None

\end{fulllineitems}

\index{disconnect() (XrdTest.ClusterManager.ClusterManager method)}

\begin{fulllineitems}
\phantomsection\label{ref-manual/XrdTest:XrdTest.ClusterManager.ClusterManager.disconnect}\pysiglinewithargsret{\bfcode{disconnect}}{}{}
Undefines and removes all virtual machines and networks created
by this cluster manager and disconnects from libvirt manager.
@raise ClusterManagerException: when fails

\end{fulllineitems}

\index{findStoragePool() (XrdTest.ClusterManager.ClusterManager method)}

\begin{fulllineitems}
\phantomsection\label{ref-manual/XrdTest:XrdTest.ClusterManager.ClusterManager.findStoragePool}\pysiglinewithargsret{\bfcode{findStoragePool}}{\emph{poolname}}{}
Attempt to find a storage pool with the given name.

\end{fulllineitems}

\index{findStorageVolume() (XrdTest.ClusterManager.ClusterManager method)}

\begin{fulllineitems}
\phantomsection\label{ref-manual/XrdTest:XrdTest.ClusterManager.ClusterManager.findStorageVolume}\pysiglinewithargsret{\bfcode{findStorageVolume}}{\emph{poolname}, \emph{volumename}}{}
Attempt to find a storage volume (file) in the specified libvirt storage
pool. If the volume is not found, the default pool will be searched. Return
the full path to the volume.

\end{fulllineitems}

\index{getPoolPath() (XrdTest.ClusterManager.ClusterManager method)}

\begin{fulllineitems}
\phantomsection\label{ref-manual/XrdTest:XrdTest.ClusterManager.ClusterManager.getPoolPath}\pysiglinewithargsret{\bfcode{getPoolPath}}{\emph{poolXML}}{}
Parse the given storage pool XML description and return its
path.

\end{fulllineitems}

\index{removeCluster() (XrdTest.ClusterManager.ClusterManager method)}

\begin{fulllineitems}
\phantomsection\label{ref-manual/XrdTest:XrdTest.ClusterManager.ClusterManager.removeCluster}\pysiglinewithargsret{\bfcode{removeCluster}}{\emph{clusterName}}{}
\end{fulllineitems}

\index{removeDanglingHost() (XrdTest.ClusterManager.ClusterManager method)}

\begin{fulllineitems}
\phantomsection\label{ref-manual/XrdTest:XrdTest.ClusterManager.ClusterManager.removeDanglingHost}\pysiglinewithargsret{\bfcode{removeDanglingHost}}{\emph{hostObj}}{}
Remove already defined host, if it has name the same as hostObj.
@param hostObj:

\end{fulllineitems}

\index{removeDanglingNetwork() (XrdTest.ClusterManager.ClusterManager method)}

\begin{fulllineitems}
\phantomsection\label{ref-manual/XrdTest:XrdTest.ClusterManager.ClusterManager.removeDanglingNetwork}\pysiglinewithargsret{\bfcode{removeDanglingNetwork}}{\emph{netObj}}{}
Remove already defined network, if it has name the same as netObj.
@param hostObj:

\end{fulllineitems}

\index{removeHost() (XrdTest.ClusterManager.ClusterManager method)}

\begin{fulllineitems}
\phantomsection\label{ref-manual/XrdTest:XrdTest.ClusterManager.ClusterManager.removeHost}\pysiglinewithargsret{\bfcode{removeHost}}{\emph{hostUName}}{}
Can not be used inside loop iterating over hosts!
@param hostUName: host.uname host unique name

\end{fulllineitems}

\index{removeHosts() (XrdTest.ClusterManager.ClusterManager method)}

\begin{fulllineitems}
\phantomsection\label{ref-manual/XrdTest:XrdTest.ClusterManager.ClusterManager.removeHosts}\pysiglinewithargsret{\bfcode{removeHosts}}{\emph{hostsUnameList}}{}
Remove multiple hosts.
@param hostsUnameList: list of unames of defined hosts.

\end{fulllineitems}

\index{removeNetwork() (XrdTest.ClusterManager.ClusterManager method)}

\begin{fulllineitems}
\phantomsection\label{ref-manual/XrdTest:XrdTest.ClusterManager.ClusterManager.removeNetwork}\pysiglinewithargsret{\bfcode{removeNetwork}}{\emph{netUName}}{}
Can not be used inside loop iterating over networks!
@param hostName:

\end{fulllineitems}

\index{updateState() (XrdTest.ClusterManager.ClusterManager method)}

\begin{fulllineitems}
\phantomsection\label{ref-manual/XrdTest:XrdTest.ClusterManager.ClusterManager.updateState}\pysiglinewithargsret{\bfcode{updateState}}{\emph{state}, \emph{clusterName}}{}
Send a progress update message to the master.

\end{fulllineitems}

\index{virtconnect() (XrdTest.ClusterManager.ClusterManager method)}

\begin{fulllineitems}
\phantomsection\label{ref-manual/XrdTest:XrdTest.ClusterManager.ClusterManager.virtconnect}\pysiglinewithargsret{\bfcode{virtconnect}}{\emph{url='qemu:///system'}}{}
Creates and returns connection to virtual machines manager
@param url: connection url
@raise ClusterManagerException: when fails to connect
@return: None

\end{fulllineitems}


\end{fulllineitems}



\subsubsection{\texttt{ClusterUtils} Module}
\label{ref-manual/XrdTest:clusterutils-module}\label{ref-manual/XrdTest:module-XrdTest.ClusterUtils}\index{XrdTest.ClusterUtils (module)}\index{Cluster (class in XrdTest.ClusterUtils)}

\begin{fulllineitems}
\phantomsection\label{ref-manual/XrdTest:XrdTest.ClusterUtils.Cluster}\pysigline{\strong{class }\code{XrdTest.ClusterUtils.}\bfcode{Cluster}}
Bases: {\hyperref[ref-manual/XrdTest:XrdTest.Utils.Stateful]{\code{XrdTest.Utils.Stateful}}}
\index{S\_ACTIVE (XrdTest.ClusterUtils.Cluster attribute)}

\begin{fulllineitems}
\phantomsection\label{ref-manual/XrdTest:XrdTest.ClusterUtils.Cluster.S_ACTIVE}\pysigline{\bfcode{S\_ACTIVE}\strong{ = (8, `Cluster active.')}}
\end{fulllineitems}

\index{S\_ATTACHING\_DISKS (XrdTest.ClusterUtils.Cluster attribute)}

\begin{fulllineitems}
\phantomsection\label{ref-manual/XrdTest:XrdTest.ClusterUtils.Cluster.S_ATTACHING_DISKS}\pysigline{\bfcode{S\_ATTACHING\_DISKS}\strong{ = (6, `Attaching slave disks.')}}
\end{fulllineitems}

\index{S\_COPYING\_IMAGES (XrdTest.ClusterUtils.Cluster attribute)}

\begin{fulllineitems}
\phantomsection\label{ref-manual/XrdTest:XrdTest.ClusterUtils.Cluster.S_COPYING_IMAGES}\pysigline{\bfcode{S\_COPYING\_IMAGES}\strong{ = (5, `Copying slave images.')}}
\end{fulllineitems}

\index{S\_CREATING\_NETWORK (XrdTest.ClusterUtils.Cluster attribute)}

\begin{fulllineitems}
\phantomsection\label{ref-manual/XrdTest:XrdTest.ClusterUtils.Cluster.S_CREATING_NETWORK}\pysigline{\bfcode{S\_CREATING\_NETWORK}\strong{ = (3, `Creating network.')}}
\end{fulllineitems}

\index{S\_CREATING\_SLAVES (XrdTest.ClusterUtils.Cluster attribute)}

\begin{fulllineitems}
\phantomsection\label{ref-manual/XrdTest:XrdTest.ClusterUtils.Cluster.S_CREATING_SLAVES}\pysigline{\bfcode{S\_CREATING\_SLAVES}\strong{ = (4, `Creating slaves.')}}
\end{fulllineitems}

\index{S\_DEFINED (XrdTest.ClusterUtils.Cluster attribute)}

\begin{fulllineitems}
\phantomsection\label{ref-manual/XrdTest:XrdTest.ClusterUtils.Cluster.S_DEFINED}\pysigline{\bfcode{S\_DEFINED}\strong{ = (0, `Cluster defined correctly.')}}
\end{fulllineitems}

\index{S\_DEFINITION\_SENT (XrdTest.ClusterUtils.Cluster attribute)}

\begin{fulllineitems}
\phantomsection\label{ref-manual/XrdTest:XrdTest.ClusterUtils.Cluster.S_DEFINITION_SENT}\pysigline{\bfcode{S\_DEFINITION\_SENT}\strong{ = (1, `Cluster start command sent to hypervisor.')}}
\end{fulllineitems}

\index{S\_DESTROYING\_CLUSTER (XrdTest.ClusterUtils.Cluster attribute)}

\begin{fulllineitems}
\phantomsection\label{ref-manual/XrdTest:XrdTest.ClusterUtils.Cluster.S_DESTROYING_CLUSTER}\pysigline{\bfcode{S\_DESTROYING\_CLUSTER}\strong{ = (10, `Destroying cluster')}}
\end{fulllineitems}

\index{S\_ERROR (XrdTest.ClusterUtils.Cluster attribute)}

\begin{fulllineitems}
\phantomsection\label{ref-manual/XrdTest:XrdTest.ClusterUtils.Cluster.S_ERROR}\pysigline{\bfcode{S\_ERROR}\strong{ = (-4, `Cluster error')}}
\end{fulllineitems}

\index{S\_ERROR\_START (XrdTest.ClusterUtils.Cluster attribute)}

\begin{fulllineitems}
\phantomsection\label{ref-manual/XrdTest:XrdTest.ClusterUtils.Cluster.S_ERROR_START}\pysigline{\bfcode{S\_ERROR\_START}\strong{ = (-3, `Error at start:')}}
\end{fulllineitems}

\index{S\_ERROR\_STOP (XrdTest.ClusterUtils.Cluster attribute)}

\begin{fulllineitems}
\phantomsection\label{ref-manual/XrdTest:XrdTest.ClusterUtils.Cluster.S_ERROR_STOP}\pysigline{\bfcode{S\_ERROR\_STOP}\strong{ = (-2, `Error at stop:')}}
\end{fulllineitems}

\index{S\_STARTING\_CLUSTER (XrdTest.ClusterUtils.Cluster attribute)}

\begin{fulllineitems}
\phantomsection\label{ref-manual/XrdTest:XrdTest.ClusterUtils.Cluster.S_STARTING_CLUSTER}\pysigline{\bfcode{S\_STARTING\_CLUSTER}\strong{ = (2, `Starting cluster.')}}
\end{fulllineitems}

\index{S\_STOPCOMMAND\_SENT (XrdTest.ClusterUtils.Cluster attribute)}

\begin{fulllineitems}
\phantomsection\label{ref-manual/XrdTest:XrdTest.ClusterUtils.Cluster.S_STOPCOMMAND_SENT}\pysigline{\bfcode{S\_STOPCOMMAND\_SENT}\strong{ = (9, `Cluster stop command sent to hypervisor.')}}
\end{fulllineitems}

\index{S\_STOPPED (XrdTest.ClusterUtils.Cluster attribute)}

\begin{fulllineitems}
\phantomsection\label{ref-manual/XrdTest:XrdTest.ClusterUtils.Cluster.S_STOPPED}\pysigline{\bfcode{S\_STOPPED}\strong{ = (11, `Cluster stopped.')}}
Represents a cluster comprised of hosts connected through network.

\end{fulllineitems}

\index{S\_UNKNOWN (XrdTest.ClusterUtils.Cluster attribute)}

\begin{fulllineitems}
\phantomsection\label{ref-manual/XrdTest:XrdTest.ClusterUtils.Cluster.S_UNKNOWN}\pysigline{\bfcode{S\_UNKNOWN}\strong{ = (0, `Cluster state unknown.')}}
\end{fulllineitems}

\index{S\_UNKNOWN\_NOHYPERV (XrdTest.ClusterUtils.Cluster attribute)}

\begin{fulllineitems}
\phantomsection\label{ref-manual/XrdTest:XrdTest.ClusterUtils.Cluster.S_UNKNOWN_NOHYPERV}\pysigline{\bfcode{S\_UNKNOWN\_NOHYPERV}\strong{ = (-1, `Cluster state unknown: no hypervisor to run cluster.')}}
\end{fulllineitems}

\index{S\_WAITING\_SLAVES (XrdTest.ClusterUtils.Cluster attribute)}

\begin{fulllineitems}
\phantomsection\label{ref-manual/XrdTest:XrdTest.ClusterUtils.Cluster.S_WAITING_SLAVES}\pysigline{\bfcode{S\_WAITING\_SLAVES}\strong{ = (7, `Waiting for slaves to connect.')}}
\end{fulllineitems}

\index{addHost() (XrdTest.ClusterUtils.Cluster method)}

\begin{fulllineitems}
\phantomsection\label{ref-manual/XrdTest:XrdTest.ClusterUtils.Cluster.addHost}\pysiglinewithargsret{\bfcode{addHost}}{\emph{host}}{}
\end{fulllineitems}

\index{addHosts() (XrdTest.ClusterUtils.Cluster method)}

\begin{fulllineitems}
\phantomsection\label{ref-manual/XrdTest:XrdTest.ClusterUtils.Cluster.addHosts}\pysiglinewithargsret{\bfcode{addHosts}}{\emph{hosts}}{}
\end{fulllineitems}

\index{network (XrdTest.ClusterUtils.Cluster attribute)}

\begin{fulllineitems}
\phantomsection\label{ref-manual/XrdTest:XrdTest.ClusterUtils.Cluster.network}\pysigline{\bfcode{network}}
\end{fulllineitems}

\index{networkGet() (XrdTest.ClusterUtils.Cluster method)}

\begin{fulllineitems}
\phantomsection\label{ref-manual/XrdTest:XrdTest.ClusterUtils.Cluster.networkGet}\pysiglinewithargsret{\bfcode{networkGet}}{}{}
\end{fulllineitems}

\index{networkSet() (XrdTest.ClusterUtils.Cluster method)}

\begin{fulllineitems}
\phantomsection\label{ref-manual/XrdTest:XrdTest.ClusterUtils.Cluster.networkSet}\pysiglinewithargsret{\bfcode{networkSet}}{\emph{net}}{}
\end{fulllineitems}

\index{setEmulatorPath() (XrdTest.ClusterUtils.Cluster method)}

\begin{fulllineitems}
\phantomsection\label{ref-manual/XrdTest:XrdTest.ClusterUtils.Cluster.setEmulatorPath}\pysiglinewithargsret{\bfcode{setEmulatorPath}}{\emph{emulator\_path}}{}
\end{fulllineitems}

\index{validateAgainstSystem() (XrdTest.ClusterUtils.Cluster method)}

\begin{fulllineitems}
\phantomsection\label{ref-manual/XrdTest:XrdTest.ClusterUtils.Cluster.validateAgainstSystem}\pysiglinewithargsret{\bfcode{validateAgainstSystem}}{\emph{clusters}}{}
Check if cluster definition is correct with other
clusters defined in the system. This correctness is critical for
cluster definition to be added.
@param clusters:

\end{fulllineitems}

\index{validateDynamic() (XrdTest.ClusterUtils.Cluster method)}

\begin{fulllineitems}
\phantomsection\label{ref-manual/XrdTest:XrdTest.ClusterUtils.Cluster.validateDynamic}\pysiglinewithargsret{\bfcode{validateDynamic}}{}{}
Check if Cluster definition is semantically correct i.e. on the
hypervisor's machine e.g. if disk images really exists on
the machine it's to be planted.

\end{fulllineitems}

\index{validateStatic() (XrdTest.ClusterUtils.Cluster method)}

\begin{fulllineitems}
\phantomsection\label{ref-manual/XrdTest:XrdTest.ClusterUtils.Cluster.validateStatic}\pysiglinewithargsret{\bfcode{validateStatic}}{}{}
Check if Cluster definition is correct and sufficient
to create a cluster.

\end{fulllineitems}


\end{fulllineitems}

\index{ClusterManagerException}

\begin{fulllineitems}
\phantomsection\label{ref-manual/XrdTest:XrdTest.ClusterUtils.ClusterManagerException}\pysiglinewithargsret{\strong{exception }\code{XrdTest.ClusterUtils.}\bfcode{ClusterManagerException}}{\emph{desc}, \emph{typeFlag=1}}{}
Bases: \code{exceptions.Exception}

General Exception raised by module

\end{fulllineitems}

\index{Disk (class in XrdTest.ClusterUtils)}

\begin{fulllineitems}
\phantomsection\label{ref-manual/XrdTest:XrdTest.ClusterUtils.Disk}\pysiglinewithargsret{\strong{class }\code{XrdTest.ClusterUtils.}\bfcode{Disk}}{\emph{name}, \emph{size}, \emph{device='vda'}, \emph{mountPoint='/data'}, \emph{cache=True}}{}
Bases: \code{object}
\index{parseDiskSize() (XrdTest.ClusterUtils.Disk method)}

\begin{fulllineitems}
\phantomsection\label{ref-manual/XrdTest:XrdTest.ClusterUtils.Disk.parseDiskSize}\pysiglinewithargsret{\bfcode{parseDiskSize}}{\emph{size}}{}
Takes a size in readable format, e.g. 50G or 200M and returns the size in bytes.

If a purely numeric string is given, a byte value will be assumed.

\end{fulllineitems}


\end{fulllineitems}

\index{Host (class in XrdTest.ClusterUtils)}

\begin{fulllineitems}
\phantomsection\label{ref-manual/XrdTest:XrdTest.ClusterUtils.Host}\pysiglinewithargsret{\strong{class }\code{XrdTest.ClusterUtils.}\bfcode{Host}}{\emph{name='`}, \emph{ip='`}, \emph{net='`}, \emph{ramSize='`}, \emph{arch='`}, \emph{bootImage=None}, \emph{cacheBootImage=True}, \emph{emulatorPath='`}, \emph{uuid='`}}{}
Bases: \code{object}

Represents a virtual host which may be added to network
\index{randMac() (XrdTest.ClusterUtils.Host method)}

\begin{fulllineitems}
\phantomsection\label{ref-manual/XrdTest:XrdTest.ClusterUtils.Host.randMac}\pysiglinewithargsret{\bfcode{randMac}}{}{}
\end{fulllineitems}

\index{uname (XrdTest.ClusterUtils.Host attribute)}

\begin{fulllineitems}
\phantomsection\label{ref-manual/XrdTest:XrdTest.ClusterUtils.Host.uname}\pysigline{\bfcode{uname}}
Return unique name of the machine within cluster's namespace.

\end{fulllineitems}

\index{xmlDesc (XrdTest.ClusterUtils.Host attribute)}

\begin{fulllineitems}
\phantomsection\label{ref-manual/XrdTest:XrdTest.ClusterUtils.Host.xmlDesc}\pysigline{\bfcode{xmlDesc}}
\end{fulllineitems}

\index{xmlDomainPattern (XrdTest.ClusterUtils.Host attribute)}

\begin{fulllineitems}
\phantomsection\label{ref-manual/XrdTest:XrdTest.ClusterUtils.Host.xmlDomainPattern}\pysigline{\bfcode{xmlDomainPattern}\strong{ = ``\textbackslash{}n\textless{}domain type='kvm'\textgreater{}\textbackslash{}n  \textless{}name\textgreater{}\%(uname)s\textless{}/name\textgreater{}\textbackslash{}n  \textless{}uuid\textgreater{}\%(uuid)s\textless{}/uuid\textgreater{}\textbackslash{}n  \textless{}memory\textgreater{}\%(ramSize)s\textless{}/memory\textgreater{}\textbackslash{}n  \textless{}vcpu\textgreater{}1\textless{}/vcpu\textgreater{}\textbackslash{}n  \textless{}os\textgreater{}\textbackslash{}n    \textless{}type arch='\%(arch)s' machine='pc'\textgreater{}hvm\textless{}/type\textgreater{}\textbackslash{}n    \textless{}boot dev='hd'/\textgreater{}\textbackslash{}n  \textless{}/os\textgreater{}\textbackslash{}n  \textless{}features\textgreater{}\textbackslash{}n    \textless{}acpi/\textgreater{}\textbackslash{}n    \textless{}apic/\textgreater{}\textbackslash{}n    \textless{}pae/\textgreater{}\textbackslash{}n  \textless{}/features\textgreater{}\textbackslash{}n  \textless{}on\_poweroff\textgreater{}destroy\textless{}/on\_poweroff\textgreater{}\textbackslash{}n  \textless{}on\_reboot\textgreater{}restart\textless{}/on\_reboot\textgreater{}\textbackslash{}n  \textless{}on\_crash\textgreater{}restart\textless{}/on\_crash\textgreater{}\textbackslash{}n  \textless{}devices\textgreater{}\textbackslash{}n    \textless{}emulator\textgreater{}\%(emulatorPath)s\textless{}/emulator\textgreater{}\textbackslash{}n    \textless{}disk type='file' device='disk'\textgreater{}\textbackslash{}n      \textless{}driver name='qemu' type='raw'/\textgreater{}\textbackslash{}n      \textless{}source file='\%(runningDiskImage)s'/\textgreater{}\textbackslash{}n      \textless{}target dev='hda' bus='ide'/\textgreater{}\textbackslash{}n      \textless{}address type='drive' controller=`0' bus=`0' unit=`0'/\textgreater{}\textbackslash{}n    \textless{}/disk\textgreater{}\textbackslash{}n    \textless{}disk type='block' device='cdrom'\textgreater{}\textbackslash{}n      \textless{}driver name='qemu' type='raw'/\textgreater{}\textbackslash{}n      \textless{}target dev='hdc' bus='ide'/\textgreater{}\textbackslash{}n      \textless{}readonly/\textgreater{}\textbackslash{}n      \textless{}address type='drive' controller=`0' bus=`1' unit=`0'/\textgreater{}\textbackslash{}n    \textless{}/disk\textgreater{}\textbackslash{}n    \textless{}controller type='ide' index=`0'\textgreater{}\textbackslash{}n      \textless{}address type='pci' domain=`0x0000' bus=`0x00' slot=`0x01'\textbackslash{}n      function=`0x1'/\textgreater{}\textbackslash{}n    \textless{}/controller\textgreater{}\textbackslash{}n    \textless{}interface type='network'\textgreater{}\textbackslash{}n      \textless{}mac address='\%(mac)s'/\textgreater{}\textbackslash{}n      \textless{}source network='\%(net)s'/\textgreater{}\textbackslash{}n      \textless{}address type='pci' domain=`0x0000' bus=`0x00' slot=`0x03' function=`0x0'/\textgreater{}\textbackslash{}n      \textless{}model type='virtio'/\textgreater{}\textbackslash{}n    \textless{}/interface\textgreater{}\textbackslash{}n    \textless{}input type='mouse' bus='ps2'/\textgreater{}\textbackslash{}n    \textless{}!-- VIDEO SECTION - NORMALLY NOT NEEDED --\textgreater{}\textbackslash{}n    \textless{}graphics type='vnc' port=`5900' autoport='yes' keymap='en-us'/\textgreater{}\textbackslash{}n    \textless{}video\textgreater{}\textbackslash{}n    \textless{}model type='cirrus' vram=`9216' heads=`1' /\textgreater{}\textbackslash{}n    \textless{}address type='pci' domain=`0x0000' bus=`0x00' slot=`0x02' function=`0x0' /\textgreater{}\textbackslash{}n    \textless{}/video\textgreater{}\textbackslash{}n    \textless{}!-- END OF VIDEO SECTION --\textgreater{}\textbackslash{}n    \textless{}memballoon model='virtio'\textgreater{}\textbackslash{}n      \textless{}address type='pci' domain=`0x0000' bus=`0x00' slot=`0x04'\textbackslash{}n      function=`0x0'/\textgreater{}\textbackslash{}n    \textless{}/memballoon\textgreater{}\textbackslash{}n  \textless{}/devices\textgreater{}\textbackslash{}n\textless{}/domain\textgreater{}\textbackslash{}n\textbackslash{}n''}}
\end{fulllineitems}


\end{fulllineitems}

\index{Network (class in XrdTest.ClusterUtils)}

\begin{fulllineitems}
\phantomsection\label{ref-manual/XrdTest:XrdTest.ClusterUtils.Network}\pysigline{\strong{class }\code{XrdTest.ClusterUtils.}\bfcode{Network}}
Bases: \code{object}

Represents a virtual network
\index{addDHCPHost() (XrdTest.ClusterUtils.Network method)}

\begin{fulllineitems}
\phantomsection\label{ref-manual/XrdTest:XrdTest.ClusterUtils.Network.addDHCPHost}\pysiglinewithargsret{\bfcode{addDHCPHost}}{\emph{host}}{}
\end{fulllineitems}

\index{addDnsHost() (XrdTest.ClusterUtils.Network method)}

\begin{fulllineitems}
\phantomsection\label{ref-manual/XrdTest:XrdTest.ClusterUtils.Network.addDnsHost}\pysiglinewithargsret{\bfcode{addDnsHost}}{\emph{host}}{}
\end{fulllineitems}

\index{addHost() (XrdTest.ClusterUtils.Network method)}

\begin{fulllineitems}
\phantomsection\label{ref-manual/XrdTest:XrdTest.ClusterUtils.Network.addHost}\pysiglinewithargsret{\bfcode{addHost}}{\emph{host}}{}
Add host to network. First to DHCP and then to DNS.
@param host: tuple (MAC address, IP address, HOST fqdn)

\end{fulllineitems}

\index{addHosts() (XrdTest.ClusterUtils.Network method)}

\begin{fulllineitems}
\phantomsection\label{ref-manual/XrdTest:XrdTest.ClusterUtils.Network.addHosts}\pysiglinewithargsret{\bfcode{addHosts}}{\emph{hostsList}}{}
Add hosts to network.
@param param: hostsList

\end{fulllineitems}

\index{uname (XrdTest.ClusterUtils.Network attribute)}

\begin{fulllineitems}
\phantomsection\label{ref-manual/XrdTest:XrdTest.ClusterUtils.Network.uname}\pysigline{\bfcode{uname}}
Return unique name of the machine within cluster's namespace.

\end{fulllineitems}

\index{xmlDesc (XrdTest.ClusterUtils.Network attribute)}

\begin{fulllineitems}
\phantomsection\label{ref-manual/XrdTest:XrdTest.ClusterUtils.Network.xmlDesc}\pysigline{\bfcode{xmlDesc}}
\end{fulllineitems}

\index{xmlDescPattern (XrdTest.ClusterUtils.Network attribute)}

\begin{fulllineitems}
\phantomsection\label{ref-manual/XrdTest:XrdTest.ClusterUtils.Network.xmlDescPattern}\pysigline{\bfcode{xmlDescPattern}\strong{ = `\textbackslash{}n\textless{}network\textgreater{}\textbackslash{}n  \textless{}name\textgreater{}\%(name)s\textless{}/name\textgreater{}\textbackslash{}n  \textless{}dns\textgreater{}\textbackslash{}n      \textless{}txt name=''xrd.test'' value=''Welcome to xrd testing framework domain.'' /\textgreater{}\textbackslash{}n      \textless{}host ip=''\%(xrdTestMasterIP)s''\textgreater{}\textbackslash{}n          \textless{}hostname\textgreater{}master.xrd.test\textless{}/hostname\textgreater{}\textbackslash{}n      \textless{}/host\textgreater{}\textbackslash{}n      \%(dnshostsxml)s\textbackslash{}n  \textless{}/dns\textgreater{}\textbackslash{}n  \textless{}forward mode=''nat''/\textgreater{}\textbackslash{}n  \textless{}bridge name=''\%(bridgename)s'' /\textgreater{}\textbackslash{}n  \textless{}ip address=''\%(ip)s'' netmask=''\%(netmask)s''\textgreater{}\textbackslash{}n    \textless{}dhcp\textgreater{}\textbackslash{}n      \textless{}range start=''\%(rangestart)s'' end=''\%(rangeend)s'' /\textgreater{}\textbackslash{}n    \%(hostsxml)s\textbackslash{}n    \textless{}/dhcp\textgreater{}\textbackslash{}n  \textless{}/ip\textgreater{}\textbackslash{}n\textless{}/network\textgreater{}\textbackslash{}n'}}
\end{fulllineitems}

\index{xmlDnsHostPattern (XrdTest.ClusterUtils.Network attribute)}

\begin{fulllineitems}
\phantomsection\label{ref-manual/XrdTest:XrdTest.ClusterUtils.Network.xmlDnsHostPattern}\pysigline{\bfcode{xmlDnsHostPattern}\strong{ = `\textbackslash{}n      \textless{}host ip=''\%(ip)s''\textgreater{}\textbackslash{}n          \%(lbalias)s\textbackslash{}n          \textless{}hostname\textgreater{}\%(hostname)s\textless{}/hostname\textgreater{}\textbackslash{}n      \textless{}/host\textgreater{}\textbackslash{}n'}}
\end{fulllineitems}

\index{xmlHostPattern (XrdTest.ClusterUtils.Network attribute)}

\begin{fulllineitems}
\phantomsection\label{ref-manual/XrdTest:XrdTest.ClusterUtils.Network.xmlHostPattern}\pysigline{\bfcode{xmlHostPattern}\strong{ = `\textbackslash{}n      \textless{}host mac=''\%(mac)s'' name=''\%(name)s'' ip=''\%(ip)s'' /\textgreater{}\textbackslash{}n'}}
\end{fulllineitems}


\end{fulllineitems}

\index{extractClusterName() (in module XrdTest.ClusterUtils)}

\begin{fulllineitems}
\phantomsection\label{ref-manual/XrdTest:XrdTest.ClusterUtils.extractClusterName}\pysiglinewithargsret{\code{XrdTest.ClusterUtils.}\bfcode{extractClusterName}}{\emph{path}}{}
\end{fulllineitems}

\index{getFileContent() (in module XrdTest.ClusterUtils)}

\begin{fulllineitems}
\phantomsection\label{ref-manual/XrdTest:XrdTest.ClusterUtils.getFileContent}\pysiglinewithargsret{\code{XrdTest.ClusterUtils.}\bfcode{getFileContent}}{\emph{filePath}}{}
Read and return whole file content as a string
@param filePath:

\end{fulllineitems}

\index{loadClusterDef() (in module XrdTest.ClusterUtils)}

\begin{fulllineitems}
\phantomsection\label{ref-manual/XrdTest:XrdTest.ClusterUtils.loadClusterDef}\pysiglinewithargsret{\code{XrdTest.ClusterUtils.}\bfcode{loadClusterDef}}{\emph{fp}, \emph{clusters}, \emph{validateWithRest=True}}{}
\end{fulllineitems}

\index{loadClustersDefs() (in module XrdTest.ClusterUtils)}

\begin{fulllineitems}
\phantomsection\label{ref-manual/XrdTest:XrdTest.ClusterUtils.loadClustersDefs}\pysiglinewithargsret{\code{XrdTest.ClusterUtils.}\bfcode{loadClustersDefs}}{\emph{path}}{}
Loads cluster definitions from .py files stored in path directory
@param path: path for .py files, storing cluster definitions

\end{fulllineitems}



\subsubsection{\texttt{Daemon} Module}
\label{ref-manual/XrdTest:daemon-module}\label{ref-manual/XrdTest:module-XrdTest.Daemon}\index{XrdTest.Daemon (module)}\index{Daemon (class in XrdTest.Daemon)}

\begin{fulllineitems}
\phantomsection\label{ref-manual/XrdTest:XrdTest.Daemon.Daemon}\pysiglinewithargsret{\strong{class }\code{XrdTest.Daemon.}\bfcode{Daemon}}{\emph{progName}, \emph{pidFile}, \emph{logFile}}{}
Represents and manages running daemon of a given runnable object.
For initialization it requires object that inherits class Runnable.
\index{check() (XrdTest.Daemon.Daemon method)}

\begin{fulllineitems}
\phantomsection\label{ref-manual/XrdTest:XrdTest.Daemon.Daemon.check}\pysiglinewithargsret{\bfcode{check}}{\emph{pid=None}}{}
Checks if process with given pid is currently running. If no pid is
given, it tries to retrieve pid from the pidFile given in the
constructor of this class.

@param pid: pid of process to be checked
@return: pid (if process runs), None (otherwise)

\end{fulllineitems}

\index{reload() (XrdTest.Daemon.Daemon method)}

\begin{fulllineitems}
\phantomsection\label{ref-manual/XrdTest:XrdTest.Daemon.Daemon.reload}\pysiglinewithargsret{\bfcode{reload}}{\emph{pid=None}}{}
Reloads the daemon by sending SIGHUM

\end{fulllineitems}

\index{removePidFile() (XrdTest.Daemon.Daemon method)}

\begin{fulllineitems}
\phantomsection\label{ref-manual/XrdTest:XrdTest.Daemon.Daemon.removePidFile}\pysiglinewithargsret{\bfcode{removePidFile}}{}{}
Remove pid file if it exists

\end{fulllineitems}

\index{start() (XrdTest.Daemon.Daemon method)}

\begin{fulllineitems}
\phantomsection\label{ref-manual/XrdTest:XrdTest.Daemon.Daemon.start}\pysiglinewithargsret{\bfcode{start}}{\emph{runnable}}{}
Starts the daemon as a separate process.

@param runnable: instance of runnable object (inherits Runnable).

\end{fulllineitems}

\index{stop() (XrdTest.Daemon.Daemon method)}

\begin{fulllineitems}
\phantomsection\label{ref-manual/XrdTest:XrdTest.Daemon.Daemon.stop}\pysiglinewithargsret{\bfcode{stop}}{\emph{pid=None}}{}
Stop the deamon

\end{fulllineitems}


\end{fulllineitems}

\index{DaemonException}

\begin{fulllineitems}
\phantomsection\label{ref-manual/XrdTest:XrdTest.Daemon.DaemonException}\pysiglinewithargsret{\strong{exception }\code{XrdTest.Daemon.}\bfcode{DaemonException}}{\emph{desc}}{}
Bases: \code{exceptions.Exception}

General Exception raised by Daemon.

\end{fulllineitems}

\index{Runnable (class in XrdTest.Daemon)}

\begin{fulllineitems}
\phantomsection\label{ref-manual/XrdTest:XrdTest.Daemon.Runnable}\pysigline{\strong{class }\code{XrdTest.Daemon.}\bfcode{Runnable}}
Bases: \code{object}

Abstract basic class for object to be runned as a daemon.
@note: children class it should handle SIGUP signal or it will suspend
\index{run() (XrdTest.Daemon.Runnable method)}

\begin{fulllineitems}
\phantomsection\label{ref-manual/XrdTest:XrdTest.Daemon.Runnable.run}\pysiglinewithargsret{\bfcode{run}}{}{}
Main jobs of programme. Has to be implemented.

\end{fulllineitems}


\end{fulllineitems}



\subsubsection{\texttt{DirectoryWatch} Module}
\label{ref-manual/XrdTest:directorywatch-module}\label{ref-manual/XrdTest:module-XrdTest.DirectoryWatch}\index{XrdTest.DirectoryWatch (module)}\index{ClustersDefinitionsChangeHandler (class in XrdTest.DirectoryWatch)}

\begin{fulllineitems}
\phantomsection\label{ref-manual/XrdTest:XrdTest.DirectoryWatch.ClustersDefinitionsChangeHandler}\pysiglinewithargsret{\strong{class }\code{XrdTest.DirectoryWatch.}\bfcode{ClustersDefinitionsChangeHandler}}{\emph{pevent=None}, \emph{**kwargs}}{}
Bases: \code{pyinotify.ProcessEvent}

If cluster definition file changes - it runs.
\index{process\_default() (XrdTest.DirectoryWatch.ClustersDefinitionsChangeHandler method)}

\begin{fulllineitems}
\phantomsection\label{ref-manual/XrdTest:XrdTest.DirectoryWatch.ClustersDefinitionsChangeHandler.process_default}\pysiglinewithargsret{\bfcode{process\_default}}{\emph{event}}{}
Actual method that handle incoming dir change event.
@param event:

\end{fulllineitems}


\end{fulllineitems}

\index{DirectoryWatch (class in XrdTest.DirectoryWatch)}

\begin{fulllineitems}
\phantomsection\label{ref-manual/XrdTest:XrdTest.DirectoryWatch.DirectoryWatch}\pysiglinewithargsret{\strong{class }\code{XrdTest.DirectoryWatch.}\bfcode{DirectoryWatch}}{\emph{repo}, \emph{config}, \emph{callback}, \emph{watch\_type=None}}{}
Bases: \code{object}

Base class for monitoring directories and invoking callback on change.
Instantiation of this class defines which type of watch will happen (local
or remote)
\index{IN\_CREATE (XrdTest.DirectoryWatch.DirectoryWatch attribute)}

\begin{fulllineitems}
\phantomsection\label{ref-manual/XrdTest:XrdTest.DirectoryWatch.DirectoryWatch.IN_CREATE}\pysigline{\bfcode{IN\_CREATE}\strong{ = 256L}}
\end{fulllineitems}

\index{IN\_DELETE (XrdTest.DirectoryWatch.DirectoryWatch attribute)}

\begin{fulllineitems}
\phantomsection\label{ref-manual/XrdTest:XrdTest.DirectoryWatch.DirectoryWatch.IN_DELETE}\pysigline{\bfcode{IN\_DELETE}\strong{ = 512L}}
\end{fulllineitems}

\index{IN\_MODIFY (XrdTest.DirectoryWatch.DirectoryWatch attribute)}

\begin{fulllineitems}
\phantomsection\label{ref-manual/XrdTest:XrdTest.DirectoryWatch.DirectoryWatch.IN_MODIFY}\pysigline{\bfcode{IN\_MODIFY}\strong{ = 2L}}
\end{fulllineitems}

\index{IN\_MOVED (XrdTest.DirectoryWatch.DirectoryWatch attribute)}

\begin{fulllineitems}
\phantomsection\label{ref-manual/XrdTest:XrdTest.DirectoryWatch.DirectoryWatch.IN_MOVED}\pysigline{\bfcode{IN\_MOVED}\strong{ = 192L}}
\end{fulllineitems}

\index{mask (XrdTest.DirectoryWatch.DirectoryWatch attribute)}

\begin{fulllineitems}
\phantomsection\label{ref-manual/XrdTest:XrdTest.DirectoryWatch.DirectoryWatch.mask}\pysigline{\bfcode{mask}\strong{ = 962L}}
\end{fulllineitems}

\index{watch() (XrdTest.DirectoryWatch.DirectoryWatch method)}

\begin{fulllineitems}
\phantomsection\label{ref-manual/XrdTest:XrdTest.DirectoryWatch.DirectoryWatch.watch}\pysiglinewithargsret{\bfcode{watch}}{}{}
\end{fulllineitems}

\index{watch\_localfs() (XrdTest.DirectoryWatch.DirectoryWatch method)}

\begin{fulllineitems}
\phantomsection\label{ref-manual/XrdTest:XrdTest.DirectoryWatch.DirectoryWatch.watch_localfs}\pysiglinewithargsret{\bfcode{watch\_localfs}}{}{}
Monitor a local directory for changes.

\end{fulllineitems}

\index{watch\_remote\_git() (XrdTest.DirectoryWatch.DirectoryWatch method)}

\begin{fulllineitems}
\phantomsection\label{ref-manual/XrdTest:XrdTest.DirectoryWatch.DirectoryWatch.watch_remote_git}\pysiglinewithargsret{\bfcode{watch\_remote\_git}}{}{}
Monitor a remote git repository by polling at a set interval.

\end{fulllineitems}


\end{fulllineitems}

\index{SuiteDefinitionsChangeHandler (class in XrdTest.DirectoryWatch)}

\begin{fulllineitems}
\phantomsection\label{ref-manual/XrdTest:XrdTest.DirectoryWatch.SuiteDefinitionsChangeHandler}\pysiglinewithargsret{\strong{class }\code{XrdTest.DirectoryWatch.}\bfcode{SuiteDefinitionsChangeHandler}}{\emph{pevent=None}, \emph{**kwargs}}{}
Bases: \code{pyinotify.ProcessEvent}

If suite definition file changes it runs
\index{process\_default() (XrdTest.DirectoryWatch.SuiteDefinitionsChangeHandler method)}

\begin{fulllineitems}
\phantomsection\label{ref-manual/XrdTest:XrdTest.DirectoryWatch.SuiteDefinitionsChangeHandler.process_default}\pysiglinewithargsret{\bfcode{process\_default}}{\emph{event}}{}
Actual method that handle incoming dir change event.
@param event:

\end{fulllineitems}


\end{fulllineitems}



\subsubsection{\texttt{GitUtils} Module}
\label{ref-manual/XrdTest:module-XrdTest.GitUtils}\label{ref-manual/XrdTest:gitutils-module}\index{XrdTest.GitUtils (module)}\index{git\_clone() (in module XrdTest.GitUtils)}

\begin{fulllineitems}
\phantomsection\label{ref-manual/XrdTest:XrdTest.GitUtils.git_clone}\pysiglinewithargsret{\code{XrdTest.GitUtils.}\bfcode{git\_clone}}{\emph{remote\_repo}, \emph{local\_repo}, \emph{cwd}}{}
Clone a remote repository into a new local directory. Must have key-based
authentication set up for this to work.

TODO: handle exceptions for no key-based auth

@param remote\_repo: the repository repo on the remote host. 
@param local\_repo: the local repo in which to clone the new repo.
@param cwd: the working directory in which to execute.

\end{fulllineitems}

\index{git\_diff() (in module XrdTest.GitUtils)}

\begin{fulllineitems}
\phantomsection\label{ref-manual/XrdTest:XrdTest.GitUtils.git_diff}\pysiglinewithargsret{\code{XrdTest.GitUtils.}\bfcode{git\_diff}}{\emph{local\_branch}, \emph{remote\_branch}, \emph{cwd}}{}
Perform a diff operation between a local and remote repository.

@param local\_branch: the local repository branch name.
@param remote\_branch: the remote branch name.
@param cwd: the working directory in which to execute.

\end{fulllineitems}

\index{git\_fetch() (in module XrdTest.GitUtils)}

\begin{fulllineitems}
\phantomsection\label{ref-manual/XrdTest:XrdTest.GitUtils.git_fetch}\pysiglinewithargsret{\code{XrdTest.GitUtils.}\bfcode{git\_fetch}}{\emph{cwd}}{}
Fetch objects and refs from a remote repository.

@param cwd: the working directory in which to execute.

\end{fulllineitems}

\index{git\_pull() (in module XrdTest.GitUtils)}

\begin{fulllineitems}
\phantomsection\label{ref-manual/XrdTest:XrdTest.GitUtils.git_pull}\pysiglinewithargsret{\code{XrdTest.GitUtils.}\bfcode{git\_pull}}{\emph{cwd}}{}
Fetch from and merge with a remote repository.

@param cwd: the working directory in which to execute.

\end{fulllineitems}

\index{sync\_remote\_git() (in module XrdTest.GitUtils)}

\begin{fulllineitems}
\phantomsection\label{ref-manual/XrdTest:XrdTest.GitUtils.sync_remote_git}\pysiglinewithargsret{\code{XrdTest.GitUtils.}\bfcode{sync\_remote\_git}}{\emph{repo}, \emph{config}}{}
Fetch the status of a remote git repository for new commits. If
new commits, pull the new changes.

Need key-based SSH authentication for this method to work. Also, on AFS
systems like lxplus, a valid kerberos ticket is needed.

@param repo: 
@param config: configuration file containing repository information

\end{fulllineitems}



\subsubsection{\texttt{Job} Module}
\label{ref-manual/XrdTest:job-module}\label{ref-manual/XrdTest:module-XrdTest.Job}\index{XrdTest.Job (module)}\index{Job (class in XrdTest.Job)}

\begin{fulllineitems}
\phantomsection\label{ref-manual/XrdTest:XrdTest.Job.Job}\pysiglinewithargsret{\strong{class }\code{XrdTest.Job.}\bfcode{Job}}{\emph{job}, \emph{groupId='`}, \emph{args=None}}{}
Bases: \code{object}

Keeps information about job, that is to be run. It's enqueued by scheduler
and dequeued if fore coming job was handled.
\index{FINALIZE\_TEST\_CASE (XrdTest.Job.Job attribute)}

\begin{fulllineitems}
\phantomsection\label{ref-manual/XrdTest:XrdTest.Job.Job.FINALIZE_TEST_CASE}\pysigline{\bfcode{FINALIZE\_TEST\_CASE}\strong{ = 5}}
\end{fulllineitems}

\index{FINALIZE\_TEST\_SUITE (XrdTest.Job.Job attribute)}

\begin{fulllineitems}
\phantomsection\label{ref-manual/XrdTest:XrdTest.Job.Job.FINALIZE_TEST_SUITE}\pysigline{\bfcode{FINALIZE\_TEST\_SUITE}\strong{ = 2}}
\end{fulllineitems}

\index{INITIALIZE\_TEST\_CASE (XrdTest.Job.Job attribute)}

\begin{fulllineitems}
\phantomsection\label{ref-manual/XrdTest:XrdTest.Job.Job.INITIALIZE_TEST_CASE}\pysigline{\bfcode{INITIALIZE\_TEST\_CASE}\strong{ = 3}}
\end{fulllineitems}

\index{INITIALIZE\_TEST\_SUITE (XrdTest.Job.Job attribute)}

\begin{fulllineitems}
\phantomsection\label{ref-manual/XrdTest:XrdTest.Job.Job.INITIALIZE_TEST_SUITE}\pysigline{\bfcode{INITIALIZE\_TEST\_SUITE}\strong{ = 1}}
\end{fulllineitems}

\index{RUN\_TEST\_CASE (XrdTest.Job.Job attribute)}

\begin{fulllineitems}
\phantomsection\label{ref-manual/XrdTest:XrdTest.Job.Job.RUN_TEST_CASE}\pysigline{\bfcode{RUN\_TEST\_CASE}\strong{ = 4}}
\end{fulllineitems}

\index{START\_CLUSTER (XrdTest.Job.Job attribute)}

\begin{fulllineitems}
\phantomsection\label{ref-manual/XrdTest:XrdTest.Job.Job.START_CLUSTER}\pysigline{\bfcode{START\_CLUSTER}\strong{ = 6}}
\end{fulllineitems}

\index{STOP\_CLUSTER (XrdTest.Job.Job attribute)}

\begin{fulllineitems}
\phantomsection\label{ref-manual/XrdTest:XrdTest.Job.Job.STOP_CLUSTER}\pysigline{\bfcode{STOP\_CLUSTER}\strong{ = 7}}
\end{fulllineitems}

\index{S\_ADDED (XrdTest.Job.Job attribute)}

\begin{fulllineitems}
\phantomsection\label{ref-manual/XrdTest:XrdTest.Job.Job.S_ADDED}\pysigline{\bfcode{S\_ADDED}\strong{ = (0, `Job added to jobs list.')}}
\end{fulllineitems}

\index{S\_STARTED (XrdTest.Job.Job attribute)}

\begin{fulllineitems}
\phantomsection\label{ref-manual/XrdTest:XrdTest.Job.Job.S_STARTED}\pysigline{\bfcode{S\_STARTED}\strong{ = (1, `Job started. In progress.')}}
\end{fulllineitems}

\index{TEST\_JOB (XrdTest.Job.Job attribute)}

\begin{fulllineitems}
\phantomsection\label{ref-manual/XrdTest:XrdTest.Job.Job.TEST_JOB}\pysigline{\bfcode{TEST\_JOB}\strong{ = 0}}
\end{fulllineitems}

\index{genJobGroupId() (XrdTest.Job.Job static method)}

\begin{fulllineitems}
\phantomsection\label{ref-manual/XrdTest:XrdTest.Job.Job.genJobGroupId}\pysiglinewithargsret{\strong{static }\bfcode{genJobGroupId}}{\emph{suite\_name}}{}
Utility function to create unique name for group of jobs.
@param suite\_name:

\end{fulllineitems}


\end{fulllineitems}



\subsubsection{\texttt{SocketUtils} Module}
\label{ref-manual/XrdTest:socketutils-module}\label{ref-manual/XrdTest:module-XrdTest.SocketUtils}\index{XrdTest.SocketUtils (module)}\index{FixedSockStream (class in XrdTest.SocketUtils)}

\begin{fulllineitems}
\phantomsection\label{ref-manual/XrdTest:XrdTest.SocketUtils.FixedSockStream}\pysiglinewithargsret{\strong{class }\code{XrdTest.SocketUtils.}\bfcode{FixedSockStream}}{\emph{sock}}{}
Bases: \code{object}

Wrapper for socket to ensure correct behaviour of send and recv.
\index{close() (XrdTest.SocketUtils.FixedSockStream method)}

\begin{fulllineitems}
\phantomsection\label{ref-manual/XrdTest:XrdTest.SocketUtils.FixedSockStream.close}\pysiglinewithargsret{\bfcode{close}}{}{}
\end{fulllineitems}

\index{recv() (XrdTest.SocketUtils.FixedSockStream method)}

\begin{fulllineitems}
\phantomsection\label{ref-manual/XrdTest:XrdTest.SocketUtils.FixedSockStream.recv}\pysiglinewithargsret{\bfcode{recv}}{\emph{recvRaw=False}}{}
\end{fulllineitems}

\index{recvBounded() (XrdTest.SocketUtils.FixedSockStream method)}

\begin{fulllineitems}
\phantomsection\label{ref-manual/XrdTest:XrdTest.SocketUtils.FixedSockStream.recvBounded}\pysiglinewithargsret{\bfcode{recvBounded}}{\emph{toRecvLen}}{}
\end{fulllineitems}

\index{send() (XrdTest.SocketUtils.FixedSockStream method)}

\begin{fulllineitems}
\phantomsection\label{ref-manual/XrdTest:XrdTest.SocketUtils.FixedSockStream.send}\pysiglinewithargsret{\bfcode{send}}{\emph{obj}, \emph{sendRaw=False}}{}
\end{fulllineitems}

\index{sendBounded() (XrdTest.SocketUtils.FixedSockStream method)}

\begin{fulllineitems}
\phantomsection\label{ref-manual/XrdTest:XrdTest.SocketUtils.FixedSockStream.sendBounded}\pysiglinewithargsret{\bfcode{sendBounded}}{\emph{msg}, \emph{toSendLen}}{}
\end{fulllineitems}


\end{fulllineitems}

\index{PriorityBlockingQueue (class in XrdTest.SocketUtils)}

\begin{fulllineitems}
\phantomsection\label{ref-manual/XrdTest:XrdTest.SocketUtils.PriorityBlockingQueue}\pysigline{\strong{class }\code{XrdTest.SocketUtils.}\bfcode{PriorityBlockingQueue}}
Bases: \code{object}

Synchronized priority queue.
Pattern for entries is a tuple in the form: (priority\_number, data).
Lowest valued entries are retrieved first.
\index{get() (XrdTest.SocketUtils.PriorityBlockingQueue method)}

\begin{fulllineitems}
\phantomsection\label{ref-manual/XrdTest:XrdTest.SocketUtils.PriorityBlockingQueue.get}\pysiglinewithargsret{\bfcode{get}}{}{}
Retrieves data of an element from (priority, data)
with the lowest priority from the queue.

\end{fulllineitems}

\index{put() (XrdTest.SocketUtils.PriorityBlockingQueue method)}

\begin{fulllineitems}
\phantomsection\label{ref-manual/XrdTest:XrdTest.SocketUtils.PriorityBlockingQueue.put}\pysiglinewithargsret{\bfcode{put}}{\emph{elem}}{}
Puts element to the queue.
@param elem: a tuple in the form: (priority\_number{[}int{]}, data).

\end{fulllineitems}

\index{rawGet() (XrdTest.SocketUtils.PriorityBlockingQueue method)}

\begin{fulllineitems}
\phantomsection\label{ref-manual/XrdTest:XrdTest.SocketUtils.PriorityBlockingQueue.rawGet}\pysiglinewithargsret{\bfcode{rawGet}}{}{}
Retrieves tuple element (priority, data)
with the lowest priority from the queue.

\end{fulllineitems}


\end{fulllineitems}

\index{SocketDisconnectedError}

\begin{fulllineitems}
\phantomsection\label{ref-manual/XrdTest:XrdTest.SocketUtils.SocketDisconnectedError}\pysiglinewithargsret{\strong{exception }\code{XrdTest.SocketUtils.}\bfcode{SocketDisconnectedError}}{\emph{desc}}{}
Bases: \code{exceptions.Exception}

TODO

\end{fulllineitems}

\index{XrdMessage (class in XrdTest.SocketUtils)}

\begin{fulllineitems}
\phantomsection\label{ref-manual/XrdTest:XrdTest.SocketUtils.XrdMessage}\pysiglinewithargsret{\strong{class }\code{XrdTest.SocketUtils.}\bfcode{XrdMessage}}{\emph{name}, \emph{msg\_sender=None}}{}
Bases: \code{object}

Network message passed between Xrd Testing Framework nodes.
\index{M\_CLUSTER\_STATE (XrdTest.SocketUtils.XrdMessage attribute)}

\begin{fulllineitems}
\phantomsection\label{ref-manual/XrdTest:XrdTest.SocketUtils.XrdMessage.M_CLUSTER_STATE}\pysigline{\bfcode{M\_CLUSTER\_STATE}\strong{ = `cluster\_state'}}
\end{fulllineitems}

\index{M\_DISCONNECT (XrdTest.SocketUtils.XrdMessage attribute)}

\begin{fulllineitems}
\phantomsection\label{ref-manual/XrdTest:XrdTest.SocketUtils.XrdMessage.M_DISCONNECT}\pysigline{\bfcode{M\_DISCONNECT}\strong{ = `disconnect'}}
\end{fulllineitems}

\index{M\_HELLO (XrdTest.SocketUtils.XrdMessage attribute)}

\begin{fulllineitems}
\phantomsection\label{ref-manual/XrdTest:XrdTest.SocketUtils.XrdMessage.M_HELLO}\pysigline{\bfcode{M\_HELLO}\strong{ = `hello'}}
\end{fulllineitems}

\index{M\_HYPERVISOR\_STATE (XrdTest.SocketUtils.XrdMessage attribute)}

\begin{fulllineitems}
\phantomsection\label{ref-manual/XrdTest:XrdTest.SocketUtils.XrdMessage.M_HYPERVISOR_STATE}\pysigline{\bfcode{M\_HYPERVISOR\_STATE}\strong{ = `hypervisor\_state'}}
\end{fulllineitems}

\index{M\_START\_CLUSTER (XrdTest.SocketUtils.XrdMessage attribute)}

\begin{fulllineitems}
\phantomsection\label{ref-manual/XrdTest:XrdTest.SocketUtils.XrdMessage.M_START_CLUSTER}\pysigline{\bfcode{M\_START\_CLUSTER}\strong{ = `start\_cluster'}}
\end{fulllineitems}

\index{M\_STOP\_CLUSTER (XrdTest.SocketUtils.XrdMessage attribute)}

\begin{fulllineitems}
\phantomsection\label{ref-manual/XrdTest:XrdTest.SocketUtils.XrdMessage.M_STOP_CLUSTER}\pysigline{\bfcode{M\_STOP\_CLUSTER}\strong{ = `stop\_cluster'}}
\end{fulllineitems}

\index{M\_TAG\_REPLY (XrdTest.SocketUtils.XrdMessage attribute)}

\begin{fulllineitems}
\phantomsection\label{ref-manual/XrdTest:XrdTest.SocketUtils.XrdMessage.M_TAG_REPLY}\pysigline{\bfcode{M\_TAG\_REPLY}\strong{ = `tag\_reply'}}
\end{fulllineitems}

\index{M\_TAG\_REQUEST (XrdTest.SocketUtils.XrdMessage attribute)}

\begin{fulllineitems}
\phantomsection\label{ref-manual/XrdTest:XrdTest.SocketUtils.XrdMessage.M_TAG_REQUEST}\pysigline{\bfcode{M\_TAG\_REQUEST}\strong{ = `tag\_request'}}
\end{fulllineitems}

\index{M\_TESTCASE\_FINALIZE (XrdTest.SocketUtils.XrdMessage attribute)}

\begin{fulllineitems}
\phantomsection\label{ref-manual/XrdTest:XrdTest.SocketUtils.XrdMessage.M_TESTCASE_FINALIZE}\pysigline{\bfcode{M\_TESTCASE\_FINALIZE}\strong{ = `test\_case\_finalize'}}
\end{fulllineitems}

\index{M\_TESTCASE\_INIT (XrdTest.SocketUtils.XrdMessage attribute)}

\begin{fulllineitems}
\phantomsection\label{ref-manual/XrdTest:XrdTest.SocketUtils.XrdMessage.M_TESTCASE_INIT}\pysigline{\bfcode{M\_TESTCASE\_INIT}\strong{ = `test\_case\_init'}}
\end{fulllineitems}

\index{M\_TESTCASE\_RUN (XrdTest.SocketUtils.XrdMessage attribute)}

\begin{fulllineitems}
\phantomsection\label{ref-manual/XrdTest:XrdTest.SocketUtils.XrdMessage.M_TESTCASE_RUN}\pysigline{\bfcode{M\_TESTCASE\_RUN}\strong{ = `test\_case\_run'}}
\end{fulllineitems}

\index{M\_TESTCASE\_STAGE\_RESULT (XrdTest.SocketUtils.XrdMessage attribute)}

\begin{fulllineitems}
\phantomsection\label{ref-manual/XrdTest:XrdTest.SocketUtils.XrdMessage.M_TESTCASE_STAGE_RESULT}\pysigline{\bfcode{M\_TESTCASE\_STAGE\_RESULT}\strong{ = `test\_case\_stage\_result'}}
\end{fulllineitems}

\index{M\_TESTSUITE\_FINALIZE (XrdTest.SocketUtils.XrdMessage attribute)}

\begin{fulllineitems}
\phantomsection\label{ref-manual/XrdTest:XrdTest.SocketUtils.XrdMessage.M_TESTSUITE_FINALIZE}\pysigline{\bfcode{M\_TESTSUITE\_FINALIZE}\strong{ = `test\_suite\_finalize'}}
\end{fulllineitems}

\index{M\_TESTSUITE\_INIT (XrdTest.SocketUtils.XrdMessage attribute)}

\begin{fulllineitems}
\phantomsection\label{ref-manual/XrdTest:XrdTest.SocketUtils.XrdMessage.M_TESTSUITE_INIT}\pysigline{\bfcode{M\_TESTSUITE\_INIT}\strong{ = `test\_suite\_init'}}
\end{fulllineitems}

\index{M\_TESTSUITE\_STATE (XrdTest.SocketUtils.XrdMessage attribute)}

\begin{fulllineitems}
\phantomsection\label{ref-manual/XrdTest:XrdTest.SocketUtils.XrdMessage.M_TESTSUITE_STATE}\pysigline{\bfcode{M\_TESTSUITE\_STATE}\strong{ = `test\_case\_state'}}
\end{fulllineitems}

\index{M\_UNKNOWN (XrdTest.SocketUtils.XrdMessage attribute)}

\begin{fulllineitems}
\phantomsection\label{ref-manual/XrdTest:XrdTest.SocketUtils.XrdMessage.M_UNKNOWN}\pysigline{\bfcode{M\_UNKNOWN}\strong{ = `unknown'}}
\end{fulllineitems}

\index{name (XrdTest.SocketUtils.XrdMessage attribute)}

\begin{fulllineitems}
\phantomsection\label{ref-manual/XrdTest:XrdTest.SocketUtils.XrdMessage.name}\pysigline{\bfcode{name}\strong{ = `unknown'}}
\end{fulllineitems}

\index{sender (XrdTest.SocketUtils.XrdMessage attribute)}

\begin{fulllineitems}
\phantomsection\label{ref-manual/XrdTest:XrdTest.SocketUtils.XrdMessage.sender}\pysigline{\bfcode{sender}\strong{ = None}}
\end{fulllineitems}


\end{fulllineitems}



\subsubsection{\texttt{TCPClient} Module}
\label{ref-manual/XrdTest:tcpclient-module}\label{ref-manual/XrdTest:module-XrdTest.TCPClient}\index{XrdTest.TCPClient (module)}\index{Hypervisor (class in XrdTest.TCPClient)}

\begin{fulllineitems}
\phantomsection\label{ref-manual/XrdTest:XrdTest.TCPClient.Hypervisor}\pysiglinewithargsret{\strong{class }\code{XrdTest.TCPClient.}\bfcode{Hypervisor}}{\emph{socket}, \emph{hostname}, \emph{address}, \emph{state}}{}
Bases: {\hyperref[ref-manual/XrdTest:XrdTest.TCPClient.TCPClient]{\code{XrdTest.TCPClient.TCPClient}}}

Wrapper for any hypervisor connection established.

\end{fulllineitems}

\index{Slave (class in XrdTest.TCPClient)}

\begin{fulllineitems}
\phantomsection\label{ref-manual/XrdTest:XrdTest.TCPClient.Slave}\pysiglinewithargsret{\strong{class }\code{XrdTest.TCPClient.}\bfcode{Slave}}{\emph{socket}, \emph{hostname}, \emph{address}, \emph{state}}{}
Bases: {\hyperref[ref-manual/XrdTest:XrdTest.TCPClient.TCPClient]{\code{XrdTest.TCPClient.TCPClient}}}

Wrapper for any slave connection established.
\index{S\_SUITE\_FINALIZE\_SENT (XrdTest.TCPClient.Slave attribute)}

\begin{fulllineitems}
\phantomsection\label{ref-manual/XrdTest:XrdTest.TCPClient.Slave.S_SUITE_FINALIZE_SENT}\pysigline{\bfcode{S\_SUITE\_FINALIZE\_SENT}\strong{ = (12, `Test suite finalize sent to slave')}}
\end{fulllineitems}

\index{S\_SUITE\_INITIALIZED (XrdTest.TCPClient.Slave attribute)}

\begin{fulllineitems}
\phantomsection\label{ref-manual/XrdTest:XrdTest.TCPClient.Slave.S_SUITE_INITIALIZED}\pysigline{\bfcode{S\_SUITE\_INITIALIZED}\strong{ = (11, `Test suite initialized')}}
\end{fulllineitems}

\index{S\_SUITE\_INIT\_SENT (XrdTest.TCPClient.Slave attribute)}

\begin{fulllineitems}
\phantomsection\label{ref-manual/XrdTest:XrdTest.TCPClient.Slave.S_SUITE_INIT_SENT}\pysigline{\bfcode{S\_SUITE\_INIT\_SENT}\strong{ = (10, `Test suite init sent to slave')}}
\end{fulllineitems}

\index{S\_TEST\_FINALIZED (XrdTest.TCPClient.Slave attribute)}

\begin{fulllineitems}
\phantomsection\label{ref-manual/XrdTest:XrdTest.TCPClient.Slave.S_TEST_FINALIZED}\pysigline{\bfcode{S\_TEST\_FINALIZED}\strong{ = (26, `Test case finalized')}}
\end{fulllineitems}

\index{S\_TEST\_FINALIZE\_SENT (XrdTest.TCPClient.Slave attribute)}

\begin{fulllineitems}
\phantomsection\label{ref-manual/XrdTest:XrdTest.TCPClient.Slave.S_TEST_FINALIZE_SENT}\pysigline{\bfcode{S\_TEST\_FINALIZE\_SENT}\strong{ = (25, `Sent test case finalize to slave')}}
\end{fulllineitems}

\index{S\_TEST\_INITIALIZED (XrdTest.TCPClient.Slave attribute)}

\begin{fulllineitems}
\phantomsection\label{ref-manual/XrdTest:XrdTest.TCPClient.Slave.S_TEST_INITIALIZED}\pysigline{\bfcode{S\_TEST\_INITIALIZED}\strong{ = (22, `Test case initialized')}}
\end{fulllineitems}

\index{S\_TEST\_INIT\_SENT (XrdTest.TCPClient.Slave attribute)}

\begin{fulllineitems}
\phantomsection\label{ref-manual/XrdTest:XrdTest.TCPClient.Slave.S_TEST_INIT_SENT}\pysigline{\bfcode{S\_TEST\_INIT\_SENT}\strong{ = (21, `Sent test case init to slave')}}
\end{fulllineitems}

\index{S\_TEST\_RUN\_FINISHED (XrdTest.TCPClient.Slave attribute)}

\begin{fulllineitems}
\phantomsection\label{ref-manual/XrdTest:XrdTest.TCPClient.Slave.S_TEST_RUN_FINISHED}\pysigline{\bfcode{S\_TEST\_RUN\_FINISHED}\strong{ = (24, `Test case run finished')}}
\end{fulllineitems}

\index{S\_TEST\_RUN\_SENT (XrdTest.TCPClient.Slave attribute)}

\begin{fulllineitems}
\phantomsection\label{ref-manual/XrdTest:XrdTest.TCPClient.Slave.S_TEST_RUN_SENT}\pysigline{\bfcode{S\_TEST\_RUN\_SENT}\strong{ = (23, `Sent test case run to slave')}}
\end{fulllineitems}


\end{fulllineitems}

\index{TCPClient (class in XrdTest.TCPClient)}

\begin{fulllineitems}
\phantomsection\label{ref-manual/XrdTest:XrdTest.TCPClient.TCPClient}\pysiglinewithargsret{\strong{class }\code{XrdTest.TCPClient.}\bfcode{TCPClient}}{\emph{socket}, \emph{hostname}, \emph{address}, \emph{state}}{}
Bases: {\hyperref[ref-manual/XrdTest:XrdTest.Utils.Stateful]{\code{XrdTest.Utils.Stateful}}}

Represents any type of TCP client that connects to XrdTestMaster. Base
class for Hypervisor and Slave.
\index{S\_CONNECTED\_IDLE (XrdTest.TCPClient.TCPClient attribute)}

\begin{fulllineitems}
\phantomsection\label{ref-manual/XrdTest:XrdTest.TCPClient.TCPClient.S_CONNECTED_IDLE}\pysigline{\bfcode{S\_CONNECTED\_IDLE}\strong{ = (1, `Connected')}}
\end{fulllineitems}

\index{S\_NOT\_CONNECTED (XrdTest.TCPClient.TCPClient attribute)}

\begin{fulllineitems}
\phantomsection\label{ref-manual/XrdTest:XrdTest.TCPClient.TCPClient.S_NOT_CONNECTED}\pysigline{\bfcode{S\_NOT\_CONNECTED}\strong{ = (2, `Not connected')}}
\end{fulllineitems}

\index{send() (XrdTest.TCPClient.TCPClient method)}

\begin{fulllineitems}
\phantomsection\label{ref-manual/XrdTest:XrdTest.TCPClient.TCPClient.send}\pysiglinewithargsret{\bfcode{send}}{\emph{msg}}{}
\end{fulllineitems}


\end{fulllineitems}

\index{TCPReceiveThread (class in XrdTest.TCPClient)}

\begin{fulllineitems}
\phantomsection\label{ref-manual/XrdTest:XrdTest.TCPClient.TCPReceiveThread}\pysiglinewithargsret{\strong{class }\code{XrdTest.TCPClient.}\bfcode{TCPReceiveThread}}{\emph{sock}, \emph{recvQueue}}{}
Bases: \code{object}

TODO:
\index{close() (XrdTest.TCPClient.TCPReceiveThread method)}

\begin{fulllineitems}
\phantomsection\label{ref-manual/XrdTest:XrdTest.TCPClient.TCPReceiveThread.close}\pysiglinewithargsret{\bfcode{close}}{}{}
TODO:

\end{fulllineitems}

\index{run() (XrdTest.TCPClient.TCPReceiveThread method)}

\begin{fulllineitems}
\phantomsection\label{ref-manual/XrdTest:XrdTest.TCPClient.TCPReceiveThread.run}\pysiglinewithargsret{\bfcode{run}}{}{}
TODO:

\end{fulllineitems}


\end{fulllineitems}



\subsubsection{\texttt{TCPServer} Module}
\label{ref-manual/XrdTest:module-XrdTest.TCPServer}\label{ref-manual/XrdTest:tcpserver-module}\index{XrdTest.TCPServer (module)}\index{MasterEvent (class in XrdTest.TCPServer)}

\begin{fulllineitems}
\phantomsection\label{ref-manual/XrdTest:XrdTest.TCPServer.MasterEvent}\pysiglinewithargsret{\strong{class }\code{XrdTest.TCPServer.}\bfcode{MasterEvent}}{\emph{e\_type}, \emph{e\_data}, \emph{msg\_sender\_addr=None}}{}
Bases: \code{object}

Wrapper for all events that comes to XrdTestMaster. MasterEvent can
be message from slave or hypervisor, system event like socket disconnection,
cluster or test suite definition file change or scheduler job initiation.
It has priorities. PRIO\_IMPORTANT is processed before PRIO\_NORMAL.
\index{M\_CLIENT\_CONNECTED (XrdTest.TCPServer.MasterEvent attribute)}

\begin{fulllineitems}
\phantomsection\label{ref-manual/XrdTest:XrdTest.TCPServer.MasterEvent.M_CLIENT_CONNECTED}\pysigline{\bfcode{M\_CLIENT\_CONNECTED}\strong{ = 2}}
\end{fulllineitems}

\index{M\_CLIENT\_DISCONNECTED (XrdTest.TCPServer.MasterEvent attribute)}

\begin{fulllineitems}
\phantomsection\label{ref-manual/XrdTest:XrdTest.TCPServer.MasterEvent.M_CLIENT_DISCONNECTED}\pysigline{\bfcode{M\_CLIENT\_DISCONNECTED}\strong{ = 3}}
\end{fulllineitems}

\index{M\_HYPERV\_MSG (XrdTest.TCPServer.MasterEvent attribute)}

\begin{fulllineitems}
\phantomsection\label{ref-manual/XrdTest:XrdTest.TCPServer.MasterEvent.M_HYPERV_MSG}\pysigline{\bfcode{M\_HYPERV\_MSG}\strong{ = 4}}
\end{fulllineitems}

\index{M\_JOB\_ENQUEUE (XrdTest.TCPServer.MasterEvent attribute)}

\begin{fulllineitems}
\phantomsection\label{ref-manual/XrdTest:XrdTest.TCPServer.MasterEvent.M_JOB_ENQUEUE}\pysigline{\bfcode{M\_JOB\_ENQUEUE}\strong{ = 6}}
\end{fulllineitems}

\index{M\_RELOAD\_CLUSTER\_DEF (XrdTest.TCPServer.MasterEvent attribute)}

\begin{fulllineitems}
\phantomsection\label{ref-manual/XrdTest:XrdTest.TCPServer.MasterEvent.M_RELOAD_CLUSTER_DEF}\pysigline{\bfcode{M\_RELOAD\_CLUSTER\_DEF}\strong{ = 7}}
\end{fulllineitems}

\index{M\_RELOAD\_SUITE\_DEF (XrdTest.TCPServer.MasterEvent attribute)}

\begin{fulllineitems}
\phantomsection\label{ref-manual/XrdTest:XrdTest.TCPServer.MasterEvent.M_RELOAD_SUITE_DEF}\pysigline{\bfcode{M\_RELOAD\_SUITE\_DEF}\strong{ = 8}}
\end{fulllineitems}

\index{M\_SLAVE\_MSG (XrdTest.TCPServer.MasterEvent attribute)}

\begin{fulllineitems}
\phantomsection\label{ref-manual/XrdTest:XrdTest.TCPServer.MasterEvent.M_SLAVE_MSG}\pysigline{\bfcode{M\_SLAVE\_MSG}\strong{ = 5}}
\end{fulllineitems}

\index{M\_UNKNOWN (XrdTest.TCPServer.MasterEvent attribute)}

\begin{fulllineitems}
\phantomsection\label{ref-manual/XrdTest:XrdTest.TCPServer.MasterEvent.M_UNKNOWN}\pysigline{\bfcode{M\_UNKNOWN}\strong{ = 1}}
\end{fulllineitems}

\index{PRIO\_IMPORTANT (XrdTest.TCPServer.MasterEvent attribute)}

\begin{fulllineitems}
\phantomsection\label{ref-manual/XrdTest:XrdTest.TCPServer.MasterEvent.PRIO_IMPORTANT}\pysigline{\bfcode{PRIO\_IMPORTANT}\strong{ = 1}}
\end{fulllineitems}

\index{PRIO\_NORMAL (XrdTest.TCPServer.MasterEvent attribute)}

\begin{fulllineitems}
\phantomsection\label{ref-manual/XrdTest:XrdTest.TCPServer.MasterEvent.PRIO_NORMAL}\pysigline{\bfcode{PRIO\_NORMAL}\strong{ = 9}}
\end{fulllineitems}


\end{fulllineitems}

\index{ThreadedTCPRequestHandler (class in XrdTest.TCPServer)}

\begin{fulllineitems}
\phantomsection\label{ref-manual/XrdTest:XrdTest.TCPServer.ThreadedTCPRequestHandler}\pysiglinewithargsret{\strong{class }\code{XrdTest.TCPServer.}\bfcode{ThreadedTCPRequestHandler}}{\emph{request}, \emph{client\_address}, \emph{server}}{}
Bases: \code{SocketServer.BaseRequestHandler}

Client's TCP request handler.
\index{C\_HYPERV (XrdTest.TCPServer.ThreadedTCPRequestHandler attribute)}

\begin{fulllineitems}
\phantomsection\label{ref-manual/XrdTest:XrdTest.TCPServer.ThreadedTCPRequestHandler.C_HYPERV}\pysigline{\bfcode{C\_HYPERV}\strong{ = `hypervisor'}}
\end{fulllineitems}

\index{C\_SLAVE (XrdTest.TCPServer.ThreadedTCPRequestHandler attribute)}

\begin{fulllineitems}
\phantomsection\label{ref-manual/XrdTest:XrdTest.TCPServer.ThreadedTCPRequestHandler.C_SLAVE}\pysigline{\bfcode{C\_SLAVE}\strong{ = `slave'}}
\end{fulllineitems}

\index{authClient() (XrdTest.TCPServer.ThreadedTCPRequestHandler method)}

\begin{fulllineitems}
\phantomsection\label{ref-manual/XrdTest:XrdTest.TCPServer.ThreadedTCPRequestHandler.authClient}\pysiglinewithargsret{\bfcode{authClient}}{\emph{clientType}}{}
Check if hypervisor is authentic. It will provide the connection password.

\end{fulllineitems}

\index{handle() (XrdTest.TCPServer.ThreadedTCPRequestHandler method)}

\begin{fulllineitems}
\phantomsection\label{ref-manual/XrdTest:XrdTest.TCPServer.ThreadedTCPRequestHandler.handle}\pysiglinewithargsret{\bfcode{handle}}{}{}
Handle new incoming connection and keep it to receive messages.

\end{fulllineitems}

\index{setup() (XrdTest.TCPServer.ThreadedTCPRequestHandler method)}

\begin{fulllineitems}
\phantomsection\label{ref-manual/XrdTest:XrdTest.TCPServer.ThreadedTCPRequestHandler.setup}\pysiglinewithargsret{\bfcode{setup}}{}{}
Initiate class properties

\end{fulllineitems}


\end{fulllineitems}

\index{ThreadedTCPServer (class in XrdTest.TCPServer)}

\begin{fulllineitems}
\phantomsection\label{ref-manual/XrdTest:XrdTest.TCPServer.ThreadedTCPServer}\pysiglinewithargsret{\strong{class }\code{XrdTest.TCPServer.}\bfcode{ThreadedTCPServer}}{\emph{server\_address}, \emph{RequestHandlerClass}, \emph{bind\_and\_activate=True}}{}
Bases: \code{SocketServer.ThreadingMixIn}, {\hyperref[ref-manual/XrdTest:XrdTest.TCPServer.XrdTCPServer]{\code{XrdTest.TCPServer.XrdTCPServer}}}

Wrapper to create threaded TCP Server.

\end{fulllineitems}

\index{XrdTCPServer (class in XrdTest.TCPServer)}

\begin{fulllineitems}
\phantomsection\label{ref-manual/XrdTest:XrdTest.TCPServer.XrdTCPServer}\pysiglinewithargsret{\strong{class }\code{XrdTest.TCPServer.}\bfcode{XrdTCPServer}}{\emph{server\_address}, \emph{RequestHandlerClass}, \emph{bind\_and\_activate=True}}{}
Bases: \code{SocketServer.TCPServer}

Wrapper for SocketServer.TCPServer, to enable setting beneath params.
\index{allow\_reuse\_address (XrdTest.TCPServer.XrdTCPServer attribute)}

\begin{fulllineitems}
\phantomsection\label{ref-manual/XrdTest:XrdTest.TCPServer.XrdTCPServer.allow_reuse_address}\pysigline{\bfcode{allow\_reuse\_address}\strong{ = True}}
\end{fulllineitems}


\end{fulllineitems}



\subsubsection{\texttt{TestUtils} Module}
\label{ref-manual/XrdTest:module-XrdTest.TestUtils}\label{ref-manual/XrdTest:testutils-module}\index{XrdTest.TestUtils (module)}\index{TestCase (class in XrdTest.TestUtils)}

\begin{fulllineitems}
\phantomsection\label{ref-manual/XrdTest:XrdTest.TestUtils.TestCase}\pysigline{\strong{class }\code{XrdTest.TestUtils.}\bfcode{TestCase}}
Represents a single test case object.
\index{validateStatic() (XrdTest.TestUtils.TestCase method)}

\begin{fulllineitems}
\phantomsection\label{ref-manual/XrdTest:XrdTest.TestUtils.TestCase.validateStatic}\pysiglinewithargsret{\bfcode{validateStatic}}{}{}
Return whether or not definition (e.g given names) is statically correct.

\end{fulllineitems}


\end{fulllineitems}

\index{TestSuite (class in XrdTest.TestUtils)}

\begin{fulllineitems}
\phantomsection\label{ref-manual/XrdTest:XrdTest.TestUtils.TestSuite}\pysigline{\strong{class }\code{XrdTest.TestUtils.}\bfcode{TestSuite}}
Bases: {\hyperref[ref-manual/XrdTest:XrdTest.Utils.Stateful]{\code{XrdTest.Utils.Stateful}}}

Represents a single test suite object.
\index{S\_ALL\_FINALIZED (XrdTest.TestUtils.TestSuite attribute)}

\begin{fulllineitems}
\phantomsection\label{ref-manual/XrdTest:XrdTest.TestUtils.TestSuite.S_ALL_FINALIZED}\pysigline{\bfcode{S\_ALL\_FINALIZED}\strong{ = (32, `All machines finalized')}}
\end{fulllineitems}

\index{S\_ALL\_INITIALIZED (XrdTest.TestUtils.TestSuite attribute)}

\begin{fulllineitems}
\phantomsection\label{ref-manual/XrdTest:XrdTest.TestUtils.TestSuite.S_ALL_INITIALIZED}\pysigline{\bfcode{S\_ALL\_INITIALIZED}\strong{ = (22, `All machines initialized')}}
\end{fulllineitems}

\index{S\_ALL\_TEST\_FINALIZED (XrdTest.TestUtils.TestSuite attribute)}

\begin{fulllineitems}
\phantomsection\label{ref-manual/XrdTest:XrdTest.TestUtils.TestSuite.S_ALL_TEST_FINALIZED}\pysigline{\bfcode{S\_ALL\_TEST\_FINALIZED}\strong{ = (48, `Test case finalized on all slaves')}}
\end{fulllineitems}

\index{S\_ALL\_TEST\_INITIALIZED (XrdTest.TestUtils.TestSuite attribute)}

\begin{fulllineitems}
\phantomsection\label{ref-manual/XrdTest:XrdTest.TestUtils.TestSuite.S_ALL_TEST_INITIALIZED}\pysigline{\bfcode{S\_ALL\_TEST\_INITIALIZED}\strong{ = (42, `Test case initialized on all slaves')}}
\end{fulllineitems}

\index{S\_ALL\_TEST\_RUN\_FINISHED (XrdTest.TestUtils.TestSuite attribute)}

\begin{fulllineitems}
\phantomsection\label{ref-manual/XrdTest:XrdTest.TestUtils.TestSuite.S_ALL_TEST_RUN_FINISHED}\pysigline{\bfcode{S\_ALL\_TEST\_RUN\_FINISHED}\strong{ = (45, `Test case run finished on all slaves')}}
\end{fulllineitems}

\index{S\_DEF\_OK (XrdTest.TestUtils.TestSuite attribute)}

\begin{fulllineitems}
\phantomsection\label{ref-manual/XrdTest:XrdTest.TestUtils.TestSuite.S_DEF_OK}\pysigline{\bfcode{S\_DEF\_OK}\strong{ = (1, `Test suite definition complete')}}
\end{fulllineitems}

\index{S\_IDLE (XrdTest.TestUtils.TestSuite attribute)}

\begin{fulllineitems}
\phantomsection\label{ref-manual/XrdTest:XrdTest.TestUtils.TestSuite.S_IDLE}\pysigline{\bfcode{S\_IDLE}\strong{ = (10, `Waiting for cluster to activate')}}
\end{fulllineitems}

\index{S\_INIT\_ERROR (XrdTest.TestUtils.TestSuite attribute)}

\begin{fulllineitems}
\phantomsection\label{ref-manual/XrdTest:XrdTest.TestUtils.TestSuite.S_INIT_ERROR}\pysigline{\bfcode{S\_INIT\_ERROR}\strong{ = (-22, `Test suite initialization error')}}
\end{fulllineitems}

\index{S\_SLAVE\_FINALIZED (XrdTest.TestUtils.TestSuite attribute)}

\begin{fulllineitems}
\phantomsection\label{ref-manual/XrdTest:XrdTest.TestUtils.TestSuite.S_SLAVE_FINALIZED}\pysigline{\bfcode{S\_SLAVE\_FINALIZED}\strong{ = (31, `Slave finalized')}}
\end{fulllineitems}

\index{S\_SLAVE\_INITIALIZED (XrdTest.TestUtils.TestSuite attribute)}

\begin{fulllineitems}
\phantomsection\label{ref-manual/XrdTest:XrdTest.TestUtils.TestSuite.S_SLAVE_INITIALIZED}\pysigline{\bfcode{S\_SLAVE\_INITIALIZED}\strong{ = (21, `Slave initialized')}}
\end{fulllineitems}

\index{S\_SLAVE\_TEST\_FINALIZED (XrdTest.TestUtils.TestSuite attribute)}

\begin{fulllineitems}
\phantomsection\label{ref-manual/XrdTest:XrdTest.TestUtils.TestSuite.S_SLAVE_TEST_FINALIZED}\pysigline{\bfcode{S\_SLAVE\_TEST\_FINALIZED}\strong{ = (47, `Test case finalized on a slave')}}
\end{fulllineitems}

\index{S\_SLAVE\_TEST\_INITIALIZED (XrdTest.TestUtils.TestSuite attribute)}

\begin{fulllineitems}
\phantomsection\label{ref-manual/XrdTest:XrdTest.TestUtils.TestSuite.S_SLAVE_TEST_INITIALIZED}\pysigline{\bfcode{S\_SLAVE\_TEST\_INITIALIZED}\strong{ = (41, `Test case initialized on a slave')}}
\end{fulllineitems}

\index{S\_SLAVE\_TEST\_RUN\_FINISHED (XrdTest.TestUtils.TestSuite attribute)}

\begin{fulllineitems}
\phantomsection\label{ref-manual/XrdTest:XrdTest.TestUtils.TestSuite.S_SLAVE_TEST_RUN_FINISHED}\pysigline{\bfcode{S\_SLAVE\_TEST\_RUN\_FINISHED}\strong{ = (44, `Test case run finished on a slave')}}
\end{fulllineitems}

\index{S\_WAIT\_4\_FINALIZE (XrdTest.TestUtils.TestSuite attribute)}

\begin{fulllineitems}
\phantomsection\label{ref-manual/XrdTest:XrdTest.TestUtils.TestSuite.S_WAIT_4_FINALIZE}\pysigline{\bfcode{S\_WAIT\_4\_FINALIZE}\strong{ = (30, `Finalizing test suite')}}
\end{fulllineitems}

\index{S\_WAIT\_4\_INIT (XrdTest.TestUtils.TestSuite attribute)}

\begin{fulllineitems}
\phantomsection\label{ref-manual/XrdTest:XrdTest.TestUtils.TestSuite.S_WAIT_4_INIT}\pysigline{\bfcode{S\_WAIT\_4\_INIT}\strong{ = (20, `Initializing test suite')}}
\end{fulllineitems}

\index{S\_WAIT\_4\_TEST\_FINALIZE (XrdTest.TestUtils.TestSuite attribute)}

\begin{fulllineitems}
\phantomsection\label{ref-manual/XrdTest:XrdTest.TestUtils.TestSuite.S_WAIT_4_TEST_FINALIZE}\pysigline{\bfcode{S\_WAIT\_4\_TEST\_FINALIZE}\strong{ = (46, `Finalizing test case')}}
\end{fulllineitems}

\index{S\_WAIT\_4\_TEST\_INIT (XrdTest.TestUtils.TestSuite attribute)}

\begin{fulllineitems}
\phantomsection\label{ref-manual/XrdTest:XrdTest.TestUtils.TestSuite.S_WAIT_4_TEST_INIT}\pysigline{\bfcode{S\_WAIT\_4\_TEST\_INIT}\strong{ = (40, `Initializing test case')}}
\end{fulllineitems}

\index{S\_WAIT\_4\_TEST\_RUN (XrdTest.TestUtils.TestSuite attribute)}

\begin{fulllineitems}
\phantomsection\label{ref-manual/XrdTest:XrdTest.TestUtils.TestSuite.S_WAIT_4_TEST_RUN}\pysigline{\bfcode{S\_WAIT\_4\_TEST\_RUN}\strong{ = (43, `Running test case')}}
\end{fulllineitems}

\index{checkIfDefComplete() (XrdTest.TestUtils.TestSuite method)}

\begin{fulllineitems}
\phantomsection\label{ref-manual/XrdTest:XrdTest.TestUtils.TestSuite.checkIfDefComplete}\pysiglinewithargsret{\bfcode{checkIfDefComplete}}{\emph{clusters}}{}
Makes sure all cluster definitions are complete.

@param clusters: all currently defined clusters.

\end{fulllineitems}

\index{getNextRunTime() (XrdTest.TestUtils.TestSuite method)}

\begin{fulllineitems}
\phantomsection\label{ref-manual/XrdTest:XrdTest.TestUtils.TestSuite.getNextRunTime}\pysiglinewithargsret{\bfcode{getNextRunTime}}{}{}
Get the next scheduled run time for this suite.

\end{fulllineitems}

\index{has\_failure() (XrdTest.TestUtils.TestSuite method)}

\begin{fulllineitems}
\phantomsection\label{ref-manual/XrdTest:XrdTest.TestUtils.TestSuite.has_failure}\pysiglinewithargsret{\bfcode{has\_failure}}{}{}
\end{fulllineitems}

\index{validateStatic() (XrdTest.TestUtils.TestSuite method)}

\begin{fulllineitems}
\phantomsection\label{ref-manual/XrdTest:XrdTest.TestUtils.TestSuite.validateStatic}\pysiglinewithargsret{\bfcode{validateStatic}}{}{}
Checks if definition (e.g given names) is statically correct.

\end{fulllineitems}


\end{fulllineitems}

\index{TestSuiteException}

\begin{fulllineitems}
\phantomsection\label{ref-manual/XrdTest:XrdTest.TestUtils.TestSuiteException}\pysiglinewithargsret{\strong{exception }\code{XrdTest.TestUtils.}\bfcode{TestSuiteException}}{\emph{desc}, \emph{typeFlag=1}}{}
Bases: \code{exceptions.Exception}

General exception raised by module.
\index{ERR\_CRITICAL (XrdTest.TestUtils.TestSuiteException attribute)}

\begin{fulllineitems}
\phantomsection\label{ref-manual/XrdTest:XrdTest.TestUtils.TestSuiteException.ERR_CRITICAL}\pysigline{\bfcode{ERR\_CRITICAL}\strong{ = 2}}
\end{fulllineitems}

\index{ERR\_UNKNOWN (XrdTest.TestUtils.TestSuiteException attribute)}

\begin{fulllineitems}
\phantomsection\label{ref-manual/XrdTest:XrdTest.TestUtils.TestSuiteException.ERR_UNKNOWN}\pysigline{\bfcode{ERR\_UNKNOWN}\strong{ = 1}}
\end{fulllineitems}


\end{fulllineitems}

\index{TestSuiteSession (class in XrdTest.TestUtils)}

\begin{fulllineitems}
\phantomsection\label{ref-manual/XrdTest:XrdTest.TestUtils.TestSuiteSession}\pysiglinewithargsret{\strong{class }\code{XrdTest.TestUtils.}\bfcode{TestSuiteSession}}{\emph{suiteDef}}{}
Bases: {\hyperref[ref-manual/XrdTest:XrdTest.Utils.Stateful]{\code{XrdTest.Utils.Stateful}}}

Represents run of Test Suite from the moment of its initialization.
It stores all information required for test suite to be run as well as
results of test stages. It has unique id (uid parameter) for recognition,
because there will be for sure many test suites with the same name.
\index{addCaseRun() (XrdTest.TestUtils.TestSuiteSession method)}

\begin{fulllineitems}
\phantomsection\label{ref-manual/XrdTest:XrdTest.TestUtils.TestSuiteSession.addCaseRun}\pysiglinewithargsret{\bfcode{addCaseRun}}{\emph{tc}}{}
Registers run of test case. Gives unique id (uid) for started
test case, because one test case can be run many time within test
suite session.
@param tc: TestCase definition object

\end{fulllineitems}

\index{addStageResult() (XrdTest.TestUtils.TestSuiteSession method)}

\begin{fulllineitems}
\phantomsection\label{ref-manual/XrdTest:XrdTest.TestUtils.TestSuiteSession.addStageResult}\pysiglinewithargsret{\bfcode{addStageResult}}{\emph{state}, \emph{result}, \emph{uid=None}, \emph{slave\_name=None}}{}
Adds all information about stage that has finished to test suite session
object. Stage are e.g.: initialize suite on some slave, run test case
on some slave etc.
@param state: state that happened
@param result: result of test run (code, stdout, stderr, custom logs)
@param uid: uid of test case or test suite init/finalize
@param slave\_name: where stage ended

\end{fulllineitems}

\index{getTestCaseStages() (XrdTest.TestUtils.TestSuiteSession method)}

\begin{fulllineitems}
\phantomsection\label{ref-manual/XrdTest:XrdTest.TestUtils.TestSuiteSession.getTestCaseStages}\pysiglinewithargsret{\bfcode{getTestCaseStages}}{\emph{test\_case\_uid}}{}
Retrieve test case stages for given test case unique id.
@param test\_case\_uid:

\end{fulllineitems}

\index{sendEmailAlert() (XrdTest.TestUtils.TestSuiteSession method)}

\begin{fulllineitems}
\phantomsection\label{ref-manual/XrdTest:XrdTest.TestUtils.TestSuiteSession.sendEmailAlert}\pysiglinewithargsret{\bfcode{sendEmailAlert}}{\emph{failure}, \emph{state}, \emph{result=None}, \emph{slave\_name=None}, \emph{test\_case=None}, \emph{timeout=False}}{}
\end{fulllineitems}


\end{fulllineitems}

\index{extractSuiteName() (in module XrdTest.TestUtils)}

\begin{fulllineitems}
\phantomsection\label{ref-manual/XrdTest:XrdTest.TestUtils.extractSuiteName}\pysiglinewithargsret{\code{XrdTest.TestUtils.}\bfcode{extractSuiteName}}{\emph{path}}{}
Return the suite name from the given path.

\end{fulllineitems}

\index{loadTestCasesDefs() (in module XrdTest.TestUtils)}

\begin{fulllineitems}
\phantomsection\label{ref-manual/XrdTest:XrdTest.TestUtils.loadTestCasesDefs}\pysiglinewithargsret{\code{XrdTest.TestUtils.}\bfcode{loadTestCasesDefs}}{\emph{path}, \emph{tests}}{}
Loads TestCase definitions from .py file. Search for getTestCases function
in the file and expects list of testCases to be returned.

@param path: path for .py files, storing cluster definitions

\end{fulllineitems}

\index{loadTestSuiteDef() (in module XrdTest.TestUtils)}

\begin{fulllineitems}
\phantomsection\label{ref-manual/XrdTest:XrdTest.TestUtils.loadTestSuiteDef}\pysiglinewithargsret{\code{XrdTest.TestUtils.}\bfcode{loadTestSuiteDef}}{\emph{path}}{}
Load a single test suite definition.

@param path: path to the suite definition to be loaded.

\end{fulllineitems}

\index{loadTestSuiteDefs() (in module XrdTest.TestUtils)}

\begin{fulllineitems}
\phantomsection\label{ref-manual/XrdTest:XrdTest.TestUtils.loadTestSuiteDefs}\pysiglinewithargsret{\code{XrdTest.TestUtils.}\bfcode{loadTestSuiteDefs}}{\emph{path}}{}
Loads TestSuite and TestCase definitions from .py files
stored in path directory.

@param path: path for .py files, storing cluster definitions

\end{fulllineitems}

\index{readFile() (in module XrdTest.TestUtils)}

\begin{fulllineitems}
\phantomsection\label{ref-manual/XrdTest:XrdTest.TestUtils.readFile}\pysiglinewithargsret{\code{XrdTest.TestUtils.}\bfcode{readFile}}{\emph{path}}{}
\end{fulllineitems}

\index{resolveScript() (in module XrdTest.TestUtils)}

\begin{fulllineitems}
\phantomsection\label{ref-manual/XrdTest:XrdTest.TestUtils.resolveScript}\pysiglinewithargsret{\code{XrdTest.TestUtils.}\bfcode{resolveScript}}{\emph{definition}, \emph{root\_path}}{}
Grabs a script from some arbitrary path and appends a set of util functions
to it.

\end{fulllineitems}



\subsubsection{\texttt{Utils} Module}
\label{ref-manual/XrdTest:module-XrdTest.Utils}\label{ref-manual/XrdTest:utils-module}\index{XrdTest.Utils (module)}\index{Command (class in XrdTest.Utils)}

\begin{fulllineitems}
\phantomsection\label{ref-manual/XrdTest:XrdTest.Utils.Command}\pysiglinewithargsret{\strong{class }\code{XrdTest.Utils.}\bfcode{Command}}{\emph{cmd}, \emph{cwd}}{}
Bases: \code{object}

Execute a subprocess command.
\index{execute() (XrdTest.Utils.Command method)}

\begin{fulllineitems}
\phantomsection\label{ref-manual/XrdTest:XrdTest.Utils.Command.execute}\pysiglinewithargsret{\bfcode{execute}}{}{}
\end{fulllineitems}


\end{fulllineitems}

\index{Logger (class in XrdTest.Utils)}

\begin{fulllineitems}
\phantomsection\label{ref-manual/XrdTest:XrdTest.Utils.Logger}\pysiglinewithargsret{\strong{class }\code{XrdTest.Utils.}\bfcode{Logger}}{\emph{filename}}{}
Bases: \code{object}

Generic logging class
\index{setup() (XrdTest.Utils.Logger method)}

\begin{fulllineitems}
\phantomsection\label{ref-manual/XrdTest:XrdTest.Utils.Logger.setup}\pysiglinewithargsret{\bfcode{setup}}{}{}
\end{fulllineitems}


\end{fulllineitems}

\index{SafeCounter (class in XrdTest.Utils)}

\begin{fulllineitems}
\phantomsection\label{ref-manual/XrdTest:XrdTest.Utils.SafeCounter}\pysigline{\strong{class }\code{XrdTest.Utils.}\bfcode{SafeCounter}}
Bases: \code{object}

TODO:
\index{get() (XrdTest.Utils.SafeCounter method)}

\begin{fulllineitems}
\phantomsection\label{ref-manual/XrdTest:XrdTest.Utils.SafeCounter.get}\pysiglinewithargsret{\bfcode{get}}{}{}
\end{fulllineitems}

\index{inc() (XrdTest.Utils.SafeCounter method)}

\begin{fulllineitems}
\phantomsection\label{ref-manual/XrdTest:XrdTest.Utils.SafeCounter.inc}\pysiglinewithargsret{\bfcode{inc}}{}{}
\end{fulllineitems}


\end{fulllineitems}

\index{State (class in XrdTest.Utils)}

\begin{fulllineitems}
\phantomsection\label{ref-manual/XrdTest:XrdTest.Utils.State}\pysiglinewithargsret{\strong{class }\code{XrdTest.Utils.}\bfcode{State}}{\emph{status\_tuple}, \emph{additDesc='`}}{}
Bases: \code{object}

Represents current state of some entity.
\index{addDesc() (XrdTest.Utils.State method)}

\begin{fulllineitems}
\phantomsection\label{ref-manual/XrdTest:XrdTest.Utils.State.addDesc}\pysiglinewithargsret{\bfcode{addDesc}}{\emph{anyStr}}{}
\end{fulllineitems}

\index{isError() (XrdTest.Utils.State method)}

\begin{fulllineitems}
\phantomsection\label{ref-manual/XrdTest:XrdTest.Utils.State.isError}\pysiglinewithargsret{\bfcode{isError}}{}{}
\end{fulllineitems}


\end{fulllineitems}

\index{Stateful (class in XrdTest.Utils)}

\begin{fulllineitems}
\phantomsection\label{ref-manual/XrdTest:XrdTest.Utils.Stateful}\pysigline{\strong{class }\code{XrdTest.Utils.}\bfcode{Stateful}}
Bases: \code{object}

Represents stateful entity and remember its previous states.
\index{getState() (XrdTest.Utils.Stateful method)}

\begin{fulllineitems}
\phantomsection\label{ref-manual/XrdTest:XrdTest.Utils.Stateful.getState}\pysiglinewithargsret{\bfcode{getState}}{}{}
\end{fulllineitems}

\index{setState() (XrdTest.Utils.Stateful method)}

\begin{fulllineitems}
\phantomsection\label{ref-manual/XrdTest:XrdTest.Utils.Stateful.setState}\pysiglinewithargsret{\bfcode{setState}}{\emph{state}}{}
\end{fulllineitems}

\index{state (XrdTest.Utils.Stateful attribute)}

\begin{fulllineitems}
\phantomsection\label{ref-manual/XrdTest:XrdTest.Utils.Stateful.state}\pysigline{\bfcode{state}}
\end{fulllineitems}


\end{fulllineitems}

\index{redirectOutput() (in module XrdTest.Utils)}

\begin{fulllineitems}
\phantomsection\label{ref-manual/XrdTest:XrdTest.Utils.redirectOutput}\pysiglinewithargsret{\code{XrdTest.Utils.}\bfcode{redirectOutput}}{\emph{logFile}}{}
Redirect the stderr and stdout to a file

\end{fulllineitems}



\subsubsection{\texttt{WebInterface} Module}
\label{ref-manual/XrdTest:webinterface-module}\label{ref-manual/XrdTest:module-XrdTest.WebInterface}\index{XrdTest.WebInterface (module)}\index{WebInterface (class in XrdTest.WebInterface)}

\begin{fulllineitems}
\phantomsection\label{ref-manual/XrdTest:XrdTest.WebInterface.WebInterface}\pysiglinewithargsret{\strong{class }\code{XrdTest.WebInterface.}\bfcode{WebInterface}}{\emph{config}, \emph{test\_master\_ref}}{}
All pages and files available via Web Interface,
defined as methods of this class.
\index{action() (XrdTest.WebInterface.WebInterface method)}

\begin{fulllineitems}
\phantomsection\label{ref-manual/XrdTest:XrdTest.WebInterface.WebInterface.action}\pysiglinewithargsret{\bfcode{action}}{\emph{type=None}, \emph{testsuite=None}, \emph{cluster=None}, \emph{location=None}}{}
\end{fulllineitems}

\index{auth() (XrdTest.WebInterface.WebInterface method)}

\begin{fulllineitems}
\phantomsection\label{ref-manual/XrdTest:XrdTest.WebInterface.WebInterface.auth}\pysiglinewithargsret{\bfcode{auth}}{\emph{password=None}, \emph{testsuite=None}, \emph{cluster=None}, \emph{type=None}}{}
\end{fulllineitems}

\index{clusters() (XrdTest.WebInterface.WebInterface method)}

\begin{fulllineitems}
\phantomsection\label{ref-manual/XrdTest:XrdTest.WebInterface.WebInterface.clusters}\pysiglinewithargsret{\bfcode{clusters}}{}{}
\end{fulllineitems}

\index{disp() (XrdTest.WebInterface.WebInterface method)}

\begin{fulllineitems}
\phantomsection\label{ref-manual/XrdTest:XrdTest.WebInterface.WebInterface.disp}\pysiglinewithargsret{\bfcode{disp}}{\emph{body}, \emph{tvars}}{}
Utility method for displying a Cheetah template file with the
supplied variables.

@param body: to be displayed as HTML page
@param tvars: variables to be used in HTML page, Cheetah style

\end{fulllineitems}

\index{documentation() (XrdTest.WebInterface.WebInterface method)}

\begin{fulllineitems}
\phantomsection\label{ref-manual/XrdTest:XrdTest.WebInterface.WebInterface.documentation}\pysiglinewithargsret{\bfcode{documentation}}{}{}
\end{fulllineitems}

\index{downloadScript() (XrdTest.WebInterface.WebInterface method)}

\begin{fulllineitems}
\phantomsection\label{ref-manual/XrdTest:XrdTest.WebInterface.WebInterface.downloadScript}\pysiglinewithargsret{\bfcode{downloadScript}}{\emph{*script\_name}}{}
Enable slave to download some script as a regular file from master and
run it.

@param script\_name:

\end{fulllineitems}

\index{getSSSKeytable() (XrdTest.WebInterface.WebInterface method)}

\begin{fulllineitems}
\phantomsection\label{ref-manual/XrdTest:XrdTest.WebInterface.WebInterface.getSSSKeytable}\pysiglinewithargsret{\bfcode{getSSSKeytable}}{}{}
\end{fulllineitems}

\index{getSignedCertificate() (XrdTest.WebInterface.WebInterface method)}

\begin{fulllineitems}
\phantomsection\label{ref-manual/XrdTest:XrdTest.WebInterface.WebInterface.getSignedCertificate}\pysiglinewithargsret{\bfcode{getSignedCertificate}}{\emph{**kwargs}}{}
\end{fulllineitems}

\index{getTrustedCACertificate() (XrdTest.WebInterface.WebInterface method)}

\begin{fulllineitems}
\phantomsection\label{ref-manual/XrdTest:XrdTest.WebInterface.WebInterface.getTrustedCACertificate}\pysiglinewithargsret{\bfcode{getTrustedCACertificate}}{\emph{**kwargs}}{}
\end{fulllineitems}

\index{handleCherrypyError() (XrdTest.WebInterface.WebInterface method)}

\begin{fulllineitems}
\phantomsection\label{ref-manual/XrdTest:XrdTest.WebInterface.WebInterface.handleCherrypyError}\pysiglinewithargsret{\bfcode{handleCherrypyError}}{}{}
\end{fulllineitems}

\index{http\_methods\_allowed() (XrdTest.WebInterface.WebInterface method)}

\begin{fulllineitems}
\phantomsection\label{ref-manual/XrdTest:XrdTest.WebInterface.WebInterface.http_methods_allowed}\pysiglinewithargsret{\bfcode{http\_methods\_allowed}}{\emph{methods={[}'GET', `POST'{]}}}{}
\end{fulllineitems}

\index{hypervisors() (XrdTest.WebInterface.WebInterface method)}

\begin{fulllineitems}
\phantomsection\label{ref-manual/XrdTest:XrdTest.WebInterface.WebInterface.hypervisors}\pysiglinewithargsret{\bfcode{hypervisors}}{}{}
\end{fulllineitems}

\index{index() (XrdTest.WebInterface.WebInterface method)}

\begin{fulllineitems}
\phantomsection\label{ref-manual/XrdTest:XrdTest.WebInterface.WebInterface.index}\pysiglinewithargsret{\bfcode{index}}{}{}
\end{fulllineitems}

\index{indexRedirect() (XrdTest.WebInterface.WebInterface method)}

\begin{fulllineitems}
\phantomsection\label{ref-manual/XrdTest:XrdTest.WebInterface.WebInterface.indexRedirect}\pysiglinewithargsret{\bfcode{indexRedirect}}{}{}
Page that at once redirects user to index. Used to clear URL parameters.

\end{fulllineitems}

\index{showScript() (XrdTest.WebInterface.WebInterface method)}

\begin{fulllineitems}
\phantomsection\label{ref-manual/XrdTest:XrdTest.WebInterface.WebInterface.showScript}\pysiglinewithargsret{\bfcode{showScript}}{\emph{*script\_name}}{}
Enable slave to view some script as text from master and
run it.

@param script\_name:

\end{fulllineitems}

\index{testsuites() (XrdTest.WebInterface.WebInterface method)}

\begin{fulllineitems}
\phantomsection\label{ref-manual/XrdTest:XrdTest.WebInterface.WebInterface.testsuites}\pysiglinewithargsret{\bfcode{testsuites}}{\emph{ts\_name=None}}{}
\end{fulllineitems}

\index{ts\_vars() (XrdTest.WebInterface.WebInterface method)}

\begin{fulllineitems}
\phantomsection\label{ref-manual/XrdTest:XrdTest.WebInterface.WebInterface.ts_vars}\pysiglinewithargsret{\bfcode{ts\_vars}}{\emph{ts\_name}}{}
\end{fulllineitems}

\index{unsupported() (XrdTest.WebInterface.WebInterface method)}

\begin{fulllineitems}
\phantomsection\label{ref-manual/XrdTest:XrdTest.WebInterface.WebInterface.unsupported}\pysiglinewithargsret{\bfcode{unsupported}}{}{}
\end{fulllineitems}

\index{update() (XrdTest.WebInterface.WebInterface method)}

\begin{fulllineitems}
\phantomsection\label{ref-manual/XrdTest:XrdTest.WebInterface.WebInterface.update}\pysiglinewithargsret{\bfcode{update}}{\emph{path}}{}
\end{fulllineitems}

\index{vars() (XrdTest.WebInterface.WebInterface method)}

\begin{fulllineitems}
\phantomsection\label{ref-manual/XrdTest:XrdTest.WebInterface.WebInterface.vars}\pysiglinewithargsret{\bfcode{vars}}{}{}
Return the variables necessary for a webpage template.

\end{fulllineitems}


\end{fulllineitems}

\index{XrdWebInterfaceException}

\begin{fulllineitems}
\phantomsection\label{ref-manual/XrdTest:XrdTest.WebInterface.XrdWebInterfaceException}\pysiglinewithargsret{\strong{exception }\code{XrdTest.WebInterface.}\bfcode{XrdWebInterfaceException}}{\emph{desc}}{}
Bases: \code{exceptions.Exception}

General Exception raised by WebInterface.

\end{fulllineitems}



\subsection{XrdTestMaster Module}
\label{ref-manual/XrdTestMaster:xrdtestmaster-module}\label{ref-manual/XrdTestMaster:module-XrdTestMaster}\label{ref-manual/XrdTestMaster::doc}\index{XrdTestMaster (module)}\index{XrdTestMaster (class in XrdTestMaster)}

\begin{fulllineitems}
\phantomsection\label{ref-manual/XrdTestMaster:XrdTestMaster.XrdTestMaster}\pysiglinewithargsret{\strong{class }\code{XrdTestMaster.}\bfcode{XrdTestMaster}}{\emph{configFile}, \emph{backgroundMode}}{}
Bases: {\hyperref[ref-manual/XrdTest:XrdTest.Daemon.Runnable]{\code{XrdTest.Daemon.Runnable}}}

Main class of module, only one instance can exist in the system,
it's runnable as a daemon.
\index{activateCluster() (XrdTestMaster.XrdTestMaster method)}

\begin{fulllineitems}
\phantomsection\label{ref-manual/XrdTestMaster:XrdTestMaster.XrdTestMaster.activateCluster}\pysiglinewithargsret{\bfcode{activateCluster}}{\emph{cluster}}{}
Start a cluster without attaching it to a particular test suite.

\end{fulllineitems}

\index{archiveSuiteSessions() (XrdTestMaster.XrdTestMaster method)}

\begin{fulllineitems}
\phantomsection\label{ref-manual/XrdTestMaster:XrdTestMaster.XrdTestMaster.archiveSuiteSessions}\pysiglinewithargsret{\bfcode{archiveSuiteSessions}}{}{}
\end{fulllineitems}

\index{cancelTestSuite() (XrdTestMaster.XrdTestMaster method)}

\begin{fulllineitems}
\phantomsection\label{ref-manual/XrdTestMaster:XrdTestMaster.XrdTestMaster.cancelTestSuite}\pysiglinewithargsret{\bfcode{cancelTestSuite}}{\emph{test\_suite\_name}, \emph{timeout=False}}{}
Cancel a running test suite

\end{fulllineitems}

\index{checkIfSuitsDefsComplete() (XrdTestMaster.XrdTestMaster method)}

\begin{fulllineitems}
\phantomsection\label{ref-manual/XrdTestMaster:XrdTestMaster.XrdTestMaster.checkIfSuitsDefsComplete}\pysiglinewithargsret{\bfcode{checkIfSuitsDefsComplete}}{}{}
Search for incompletness in test suits' definitions, that may be caused
by e.g. lack of test case definition.
@param dirEvent:

\end{fulllineitems}

\index{createCA() (XrdTestMaster.XrdTestMaster method)}

\begin{fulllineitems}
\phantomsection\label{ref-manual/XrdTestMaster:XrdTestMaster.XrdTestMaster.createCA}\pysiglinewithargsret{\bfcode{createCA}}{}{}
Generate CA key/cert suitable for signing slave CSRs.

\end{fulllineitems}

\index{enqueueJob() (XrdTestMaster.XrdTestMaster method)}

\begin{fulllineitems}
\phantomsection\label{ref-manual/XrdTestMaster:XrdTestMaster.XrdTestMaster.enqueueJob}\pysiglinewithargsret{\bfcode{enqueueJob}}{\emph{test\_suite\_name}}{}
Add job to list of jobs to run immediately after foregoing
jobs are finished.

@param test\_suite\_name:

\end{fulllineitems}

\index{executeJob() (XrdTestMaster.XrdTestMaster method)}

\begin{fulllineitems}
\phantomsection\label{ref-manual/XrdTestMaster:XrdTestMaster.XrdTestMaster.executeJob}\pysiglinewithargsret{\bfcode{executeJob}}{\emph{test\_suite\_name}}{}
Closure to pass the contexts of method self.fireEnqueueJobEvent:
argument the test\_suite\_name.

@param test\_suite\_name: name of test suite
@return: lambda method with no argument

\end{fulllineitems}

\index{finalizeTestCase() (XrdTestMaster.XrdTestMaster method)}

\begin{fulllineitems}
\phantomsection\label{ref-manual/XrdTestMaster:XrdTestMaster.XrdTestMaster.finalizeTestCase}\pysiglinewithargsret{\bfcode{finalizeTestCase}}{\emph{test\_suite\_name}, \emph{test\_name}}{}
Sends runTest message to slaves.

@param test\_suite\_name:
@param test\_name:
@return: True/False in case of Success/Failure in sending messages

\end{fulllineitems}

\index{finalizeTestSuite() (XrdTestMaster.XrdTestMaster method)}

\begin{fulllineitems}
\phantomsection\label{ref-manual/XrdTestMaster:XrdTestMaster.XrdTestMaster.finalizeTestSuite}\pysiglinewithargsret{\bfcode{finalizeTestSuite}}{\emph{test\_suite\_name}}{}
Sends finalization message to slaves and destroys TestSuiteSession.

@param test\_suite\_name:
@return: True/False in case of Success/Failure in sending messages

\end{fulllineitems}

\index{fireEnqueueJobEvent() (XrdTestMaster.XrdTestMaster method)}

\begin{fulllineitems}
\phantomsection\label{ref-manual/XrdTestMaster:XrdTestMaster.XrdTestMaster.fireEnqueueJobEvent}\pysiglinewithargsret{\bfcode{fireEnqueueJobEvent}}{\emph{test\_suite\_name}}{}
Add the Run Job event to main events queue of controll thread.
Method to be used by different thread.

@param test\_suite\_name:
@return: None

\end{fulllineitems}

\index{fireReloadDefinitionsEvent() (XrdTestMaster.XrdTestMaster method)}

\begin{fulllineitems}
\phantomsection\label{ref-manual/XrdTestMaster:XrdTestMaster.XrdTestMaster.fireReloadDefinitionsEvent}\pysiglinewithargsret{\bfcode{fireReloadDefinitionsEvent}}{\emph{type}, \emph{dirEvent=None}}{}
Any time something is changed in the directory with config files,
it puts proper event into main events queue.
@param type:
@param dirEvent:

\end{fulllineitems}

\index{getSuiteSlaves() (XrdTestMaster.XrdTestMaster method)}

\begin{fulllineitems}
\phantomsection\label{ref-manual/XrdTestMaster:XrdTestMaster.XrdTestMaster.getSuiteSlaves}\pysiglinewithargsret{\bfcode{getSuiteSlaves}}{\emph{test\_suite}, \emph{slave\_state=None}, \emph{test\_case=None}}{}
Gets reference to slaves' objects representing slaves currently
connected. Optionally return only slaves associated with the given
test\_suite or test\_case
or being in given slave\_state. All given parameters has to accord.
@param test\_suite: test suite definition
@param slave\_state: required slave state
@param test\_case: test case defintion

\end{fulllineitems}

\index{handleClientConnected() (XrdTestMaster.XrdTestMaster method)}

\begin{fulllineitems}
\phantomsection\label{ref-manual/XrdTestMaster:XrdTestMaster.XrdTestMaster.handleClientConnected}\pysiglinewithargsret{\bfcode{handleClientConnected}}{\emph{client\_type}, \emph{client\_addr}, \emph{sock\_obj}, \emph{client\_hostname}}{}
Do the logic of client incoming connection.

@param client\_type:
@param client\_addr:
@param client\_hostname:
@return: None

\end{fulllineitems}

\index{handleClientDisconnected() (XrdTestMaster.XrdTestMaster method)}

\begin{fulllineitems}
\phantomsection\label{ref-manual/XrdTestMaster:XrdTestMaster.XrdTestMaster.handleClientDisconnected}\pysiglinewithargsret{\bfcode{handleClientDisconnected}}{\emph{client\_type}, \emph{client\_addr}}{}
Do the logic of client disconnection.

@param client\_type:
@param client\_addr:
@return: None

\end{fulllineitems}

\index{handleClusterDefinitionChanged() (XrdTestMaster.XrdTestMaster method)}

\begin{fulllineitems}
\phantomsection\label{ref-manual/XrdTestMaster:XrdTestMaster.XrdTestMaster.handleClusterDefinitionChanged}\pysiglinewithargsret{\bfcode{handleClusterDefinitionChanged}}{\emph{dirEvent}}{}
Handle event created any time definition of cluster changes.
@param dirEvent:

\end{fulllineitems}

\index{handleSuiteDefinitionChanged() (XrdTestMaster.XrdTestMaster method)}

\begin{fulllineitems}
\phantomsection\label{ref-manual/XrdTestMaster:XrdTestMaster.XrdTestMaster.handleSuiteDefinitionChanged}\pysiglinewithargsret{\bfcode{handleSuiteDefinitionChanged}}{\emph{dirEvent}}{}
Handle event created any time definition of test suite changes.
@param dirEvent:

\end{fulllineitems}

\index{handleTagRequest() (XrdTestMaster.XrdTestMaster method)}

\begin{fulllineitems}
\phantomsection\label{ref-manual/XrdTestMaster:XrdTestMaster.XrdTestMaster.handleTagRequest}\pysiglinewithargsret{\bfcode{handleTagRequest}}{\emph{slavename}}{}
TODO:

\end{fulllineitems}

\index{initializeTestCase() (XrdTestMaster.XrdTestMaster method)}

\begin{fulllineitems}
\phantomsection\label{ref-manual/XrdTestMaster:XrdTestMaster.XrdTestMaster.initializeTestCase}\pysiglinewithargsret{\bfcode{initializeTestCase}}{\emph{test\_suite\_name}, \emph{test\_name}, \emph{jobGroupId}}{}
Sends initTest message to slaves.

@param test\_suite\_name:
@param test\_name:
@param jobGroupId:
@return: True/False in case of Success/Failure in sending messages

\end{fulllineitems}

\index{initializeTestSuite() (XrdTestMaster.XrdTestMaster method)}

\begin{fulllineitems}
\phantomsection\label{ref-manual/XrdTestMaster:XrdTestMaster.XrdTestMaster.initializeTestSuite}\pysiglinewithargsret{\bfcode{initializeTestSuite}}{\emph{test\_suite\_name}, \emph{jobGroupId}}{}
Sends initialize message to slaves, creates TestSuite Session
and stores it in python shelve.

@param test\_suite\_name:
@param jobGroupId:
@return: True/False in case of Success/Failure in sending messages

\end{fulllineitems}

\index{isJobValid() (XrdTestMaster.XrdTestMaster method)}

\begin{fulllineitems}
\phantomsection\label{ref-manual/XrdTestMaster:XrdTestMaster.XrdTestMaster.isJobValid}\pysiglinewithargsret{\bfcode{isJobValid}}{\emph{job}}{}
Check if job is still executable e.g. if required definitions
are complete.

@param job:
@return: True/False

\end{fulllineitems}

\index{loadDefinitions() (XrdTestMaster.XrdTestMaster method)}

\begin{fulllineitems}
\phantomsection\label{ref-manual/XrdTestMaster:XrdTestMaster.XrdTestMaster.loadDefinitions}\pysiglinewithargsret{\bfcode{loadDefinitions}}{}{}
Load all definitions of example clusters and test suites at once.
If any definitions are invalid, raise exceptions.

\end{fulllineitems}

\index{loadSuiteSessions() (XrdTestMaster.XrdTestMaster method)}

\begin{fulllineitems}
\phantomsection\label{ref-manual/XrdTestMaster:XrdTestMaster.XrdTestMaster.loadSuiteSessions}\pysiglinewithargsret{\bfcode{loadSuiteSessions}}{}{}
\end{fulllineitems}

\index{procEvents() (XrdTestMaster.XrdTestMaster method)}

\begin{fulllineitems}
\phantomsection\label{ref-manual/XrdTestMaster:XrdTestMaster.XrdTestMaster.procEvents}\pysiglinewithargsret{\bfcode{procEvents}}{}{}
Main loop processing incoming MasterEvents from main events queue:
self.recvQueue. MasterEvents with higher priority are handled first.

\end{fulllineitems}

\index{procSlaveMsg() (XrdTestMaster.XrdTestMaster method)}

\begin{fulllineitems}
\phantomsection\label{ref-manual/XrdTestMaster:XrdTestMaster.XrdTestMaster.procSlaveMsg}\pysiglinewithargsret{\bfcode{procSlaveMsg}}{\emph{msg}}{}
Process incoming messages from a slave.

@param msg:

\end{fulllineitems}

\index{readConfig() (XrdTestMaster.XrdTestMaster method)}

\begin{fulllineitems}
\phantomsection\label{ref-manual/XrdTestMaster:XrdTestMaster.XrdTestMaster.readConfig}\pysiglinewithargsret{\bfcode{readConfig}}{\emph{confFile}}{}
Reads configuration from given file or from default if None given.
@param confFile: file with configuration

\end{fulllineitems}

\index{removeJob() (XrdTestMaster.XrdTestMaster method)}

\begin{fulllineitems}
\phantomsection\label{ref-manual/XrdTestMaster:XrdTestMaster.XrdTestMaster.removeJob}\pysiglinewithargsret{\bfcode{removeJob}}{\emph{remove\_job}}{}
Look through queue of jobs and remove one, which satisfy conditions
defined by parameters of pattern job remove\_job.

@param remove\_job: pattern of a job to be removed

\end{fulllineitems}

\index{removeJobs() (XrdTestMaster.XrdTestMaster method)}

\begin{fulllineitems}
\phantomsection\label{ref-manual/XrdTestMaster:XrdTestMaster.XrdTestMaster.removeJobs}\pysiglinewithargsret{\bfcode{removeJobs}}{\emph{groupId}, \emph{jobType=6}, \emph{testName=None}}{}
Remove multiple jobs from enqueued jobs list. Depending of what kind of
job is removed, different parameters are used and different number of
jobs is removed.
@param groupId: used for all kind of deleted jobs
@param jobType: determines type of job that begins the chain of
jobs to be removed
@param testName: used if removed jobs concerns particular test case
@return: None

\end{fulllineitems}

\index{retrieveAllSuiteSessions() (XrdTestMaster.XrdTestMaster method)}

\begin{fulllineitems}
\phantomsection\label{ref-manual/XrdTestMaster:XrdTestMaster.XrdTestMaster.retrieveAllSuiteSessions}\pysiglinewithargsret{\bfcode{retrieveAllSuiteSessions}}{}{}
\end{fulllineitems}

\index{retrieveSuiteSession() (XrdTestMaster.XrdTestMaster method)}

\begin{fulllineitems}
\phantomsection\label{ref-manual/XrdTestMaster:XrdTestMaster.XrdTestMaster.retrieveSuiteSession}\pysiglinewithargsret{\bfcode{retrieveSuiteSession}}{\emph{suite\_name}}{}
Retrieve test suite session from shelve self.suiteSessions
@param suite\_name:

\end{fulllineitems}

\index{run() (XrdTestMaster.XrdTestMaster method)}

\begin{fulllineitems}
\phantomsection\label{ref-manual/XrdTestMaster:XrdTestMaster.XrdTestMaster.run}\pysiglinewithargsret{\bfcode{run}}{}{}
Main method of a programme. Initializes all serving threads and starts
main loop receiving MasterEvents.

\end{fulllineitems}

\index{runTestCase() (XrdTestMaster.XrdTestMaster method)}

\begin{fulllineitems}
\phantomsection\label{ref-manual/XrdTestMaster:XrdTestMaster.XrdTestMaster.runTestCase}\pysiglinewithargsret{\bfcode{runTestCase}}{\emph{test\_suite\_name}, \emph{test\_name}}{}
Sends runTest message to slaves.

@param test\_suite\_name:
@param test\_name:
@return: True/False in case of Success/Failure in sending messages

\end{fulllineitems}

\index{runTestSuite() (XrdTestMaster.XrdTestMaster method)}

\begin{fulllineitems}
\phantomsection\label{ref-manual/XrdTestMaster:XrdTestMaster.XrdTestMaster.runTestSuite}\pysiglinewithargsret{\bfcode{runTestSuite}}{\emph{test\_suite\_name}}{}
Run a particular test suite

\end{fulllineitems}

\index{selectHypervisor() (XrdTestMaster.XrdTestMaster method)}

\begin{fulllineitems}
\phantomsection\label{ref-manual/XrdTestMaster:XrdTestMaster.XrdTestMaster.selectHypervisor}\pysiglinewithargsret{\bfcode{selectHypervisor}}{\emph{hypervisor=None}}{}
\end{fulllineitems}

\index{slaveState() (XrdTestMaster.XrdTestMaster method)}

\begin{fulllineitems}
\phantomsection\label{ref-manual/XrdTestMaster:XrdTestMaster.XrdTestMaster.slaveState}\pysiglinewithargsret{\bfcode{slaveState}}{\emph{slave\_name}}{}
Get state of a slave by its name, even if it's not connected.
@param slave\_name: equal to fully qualified hostname

\end{fulllineitems}

\index{startCluster() (XrdTestMaster.XrdTestMaster method)}

\begin{fulllineitems}
\phantomsection\label{ref-manual/XrdTestMaster:XrdTestMaster.XrdTestMaster.startCluster}\pysiglinewithargsret{\bfcode{startCluster}}{\emph{clusterName}, \emph{suiteName}, \emph{jobGroupId}}{}
Sends message to hypervisor to start the cluster.

@param clusterName:
@param suiteName:
@param jobGroupId:
@return: True/False in case of Success/Failure in sending messages

\end{fulllineitems}

\index{startNextJob() (XrdTestMaster.XrdTestMaster method)}

\begin{fulllineitems}
\phantomsection\label{ref-manual/XrdTestMaster:XrdTestMaster.XrdTestMaster.startNextJob}\pysiglinewithargsret{\bfcode{startNextJob}}{}{}
Start next possible job enqueued in pendingJobs list or continue
without doing anything. Check if first job on a list is not already
started, then check if it is valid and if it is, start it and change
it's status to started.

@param test\_suite\_name:
@return: None

\end{fulllineitems}

\index{startTCPServer() (XrdTestMaster.XrdTestMaster method)}

\begin{fulllineitems}
\phantomsection\label{ref-manual/XrdTestMaster:XrdTestMaster.XrdTestMaster.startTCPServer}\pysiglinewithargsret{\bfcode{startTCPServer}}{}{}
TODO:

\end{fulllineitems}

\index{startWebInterface() (XrdTestMaster.XrdTestMaster method)}

\begin{fulllineitems}
\phantomsection\label{ref-manual/XrdTestMaster:XrdTestMaster.XrdTestMaster.startWebInterface}\pysiglinewithargsret{\bfcode{startWebInterface}}{}{}
TODO:

\end{fulllineitems}

\index{stopCluster() (XrdTestMaster.XrdTestMaster method)}

\begin{fulllineitems}
\phantomsection\label{ref-manual/XrdTestMaster:XrdTestMaster.XrdTestMaster.stopCluster}\pysiglinewithargsret{\bfcode{stopCluster}}{\emph{clusterName}}{}
Sends message to hypervisor to stop the cluster.

@param clusterName:
@return: True/False in case of Success/Failure in sending messages

\end{fulllineitems}

\index{storeSuiteSession() (XrdTestMaster.XrdTestMaster method)}

\begin{fulllineitems}
\phantomsection\label{ref-manual/XrdTestMaster:XrdTestMaster.XrdTestMaster.storeSuiteSession}\pysiglinewithargsret{\bfcode{storeSuiteSession}}{\emph{test\_suite\_session}}{}
Save test suite session to python shelve self.suiteSessions
@param test\_suite\_session:

\end{fulllineitems}

\index{watchDirectories() (XrdTestMaster.XrdTestMaster method)}

\begin{fulllineitems}
\phantomsection\label{ref-manual/XrdTestMaster:XrdTestMaster.XrdTestMaster.watchDirectories}\pysiglinewithargsret{\bfcode{watchDirectories}}{}{}
TODO:

\end{fulllineitems}


\end{fulllineitems}

\index{XrdTestMasterException}

\begin{fulllineitems}
\phantomsection\label{ref-manual/XrdTestMaster:XrdTestMaster.XrdTestMasterException}\pysiglinewithargsret{\strong{exception }\code{XrdTestMaster.}\bfcode{XrdTestMasterException}}{\emph{desc}}{}
Bases: \code{exceptions.Exception}

General Exception raised by XrdTestMaster.

\end{fulllineitems}

\index{main() (in module XrdTestMaster)}

\begin{fulllineitems}
\phantomsection\label{ref-manual/XrdTestMaster:XrdTestMaster.main}\pysiglinewithargsret{\code{XrdTestMaster.}\bfcode{main}}{}{}
Program begins here.

\end{fulllineitems}



\subsection{XrdTestHypervisor Module}
\label{ref-manual/XrdTestHypervisor:xrdtesthypervisor-module}\label{ref-manual/XrdTestHypervisor::doc}\label{ref-manual/XrdTestHypervisor:module-XrdTestHypervisor}\index{XrdTestHypervisor (module)}\index{XrdTestHypervisor (class in XrdTestHypervisor)}

\begin{fulllineitems}
\phantomsection\label{ref-manual/XrdTestHypervisor:XrdTestHypervisor.XrdTestHypervisor}\pysiglinewithargsret{\strong{class }\code{XrdTestHypervisor.}\bfcode{XrdTestHypervisor}}{\emph{configFile}, \emph{backgroundMode}}{}
Bases: {\hyperref[ref-manual/XrdTest:XrdTest.Daemon.Runnable]{\code{XrdTest.Daemon.Runnable}}}

Test Hypervisor main executable class.
\index{connectMaster() (XrdTestHypervisor.XrdTestHypervisor method)}

\begin{fulllineitems}
\phantomsection\label{ref-manual/XrdTestHypervisor:XrdTestHypervisor.XrdTestHypervisor.connectMaster}\pysiglinewithargsret{\bfcode{connectMaster}}{\emph{masterName}, \emph{masterPort}}{}
Try to establish the connection with the test master.

@param masterName:
@param masterPort:

\end{fulllineitems}

\index{handleStartCluster() (XrdTestHypervisor.XrdTestHypervisor method)}

\begin{fulllineitems}
\phantomsection\label{ref-manual/XrdTestHypervisor:XrdTestHypervisor.XrdTestHypervisor.handleStartCluster}\pysiglinewithargsret{\bfcode{handleStartCluster}}{\emph{msg}}{}
Handle start cluster message from a master - start a cluster.

\end{fulllineitems}

\index{handleStopCluster() (XrdTestHypervisor.XrdTestHypervisor method)}

\begin{fulllineitems}
\phantomsection\label{ref-manual/XrdTestHypervisor:XrdTestHypervisor.XrdTestHypervisor.handleStopCluster}\pysiglinewithargsret{\bfcode{handleStopCluster}}{\emph{msg}}{}
Handle stop cluster message from a master - stop a running cluster.

\end{fulllineitems}

\index{readConfig() (XrdTestHypervisor.XrdTestHypervisor method)}

\begin{fulllineitems}
\phantomsection\label{ref-manual/XrdTestHypervisor:XrdTestHypervisor.XrdTestHypervisor.readConfig}\pysiglinewithargsret{\bfcode{readConfig}}{\emph{confFile}}{}
Reads configuration from given file or from default if None given.
@param confFile: file with configuration

\end{fulllineitems}

\index{recvLoop() (XrdTestHypervisor.XrdTestHypervisor method)}

\begin{fulllineitems}
\phantomsection\label{ref-manual/XrdTestHypervisor:XrdTestHypervisor.XrdTestHypervisor.recvLoop}\pysiglinewithargsret{\bfcode{recvLoop}}{}{}
Main loop processing messages from master. It take out jobs
from blocking queue of received messages, runs appropriate and
return answer message.

\end{fulllineitems}

\index{run() (XrdTestHypervisor.XrdTestHypervisor method)}

\begin{fulllineitems}
\phantomsection\label{ref-manual/XrdTestHypervisor:XrdTestHypervisor.XrdTestHypervisor.run}\pysiglinewithargsret{\bfcode{run}}{}{}
Main thread. Initialize TCP threads and run recvLoop().

\end{fulllineitems}

\index{tryConnect() (XrdTestHypervisor.XrdTestHypervisor method)}

\begin{fulllineitems}
\phantomsection\label{ref-manual/XrdTestHypervisor:XrdTestHypervisor.XrdTestHypervisor.tryConnect}\pysiglinewithargsret{\bfcode{tryConnect}}{}{}
Attempt to connect to the master. Retry every 5 seconds, up to a 
maximum of 500 times.

\end{fulllineitems}

\index{updateState() (XrdTestHypervisor.XrdTestHypervisor method)}

\begin{fulllineitems}
\phantomsection\label{ref-manual/XrdTestHypervisor:XrdTestHypervisor.XrdTestHypervisor.updateState}\pysiglinewithargsret{\bfcode{updateState}}{\emph{state}, \emph{clusterName}}{}
Send a progress update message to the master.

\end{fulllineitems}


\end{fulllineitems}

\index{XrdTestHypervisorException}

\begin{fulllineitems}
\phantomsection\label{ref-manual/XrdTestHypervisor:XrdTestHypervisor.XrdTestHypervisorException}\pysiglinewithargsret{\strong{exception }\code{XrdTestHypervisor.}\bfcode{XrdTestHypervisorException}}{\emph{desc}}{}
Bases: \code{exceptions.Exception}

General Exception raised by XrdTestHypervisor.

\end{fulllineitems}

\index{main() (in module XrdTestHypervisor)}

\begin{fulllineitems}
\phantomsection\label{ref-manual/XrdTestHypervisor:XrdTestHypervisor.main}\pysiglinewithargsret{\code{XrdTestHypervisor.}\bfcode{main}}{}{}
Program begins here.

\end{fulllineitems}



\subsection{XrdTestSlave Module}
\label{ref-manual/XrdTestSlave:module-XrdTestSlave}\label{ref-manual/XrdTestSlave:xrdtestslave-module}\label{ref-manual/XrdTestSlave::doc}\index{XrdTestSlave (module)}\index{XrdTestSlave (class in XrdTestSlave)}

\begin{fulllineitems}
\phantomsection\label{ref-manual/XrdTestSlave:XrdTestSlave.XrdTestSlave}\pysiglinewithargsret{\strong{class }\code{XrdTestSlave.}\bfcode{XrdTestSlave}}{\emph{configFile}, \emph{backgroundMode}}{}
Bases: {\hyperref[ref-manual/XrdTest:XrdTest.Daemon.Runnable]{\code{XrdTest.Daemon.Runnable}}}

Test Slave main executable class.
\index{connectMaster() (XrdTestSlave.XrdTestSlave method)}

\begin{fulllineitems}
\phantomsection\label{ref-manual/XrdTestSlave:XrdTestSlave.XrdTestSlave.connectMaster}\pysiglinewithargsret{\bfcode{connectMaster}}{\emph{masterName}, \emph{masterPort}}{}
TODO:

\end{fulllineitems}

\index{executeSh() (XrdTestSlave.XrdTestSlave method)}

\begin{fulllineitems}
\phantomsection\label{ref-manual/XrdTestSlave:XrdTestSlave.XrdTestSlave.executeSh}\pysiglinewithargsret{\bfcode{executeSh}}{\emph{cmd}}{}
@param cmd:

\end{fulllineitems}

\index{handleTestCaseFinalize() (XrdTestSlave.XrdTestSlave method)}

\begin{fulllineitems}
\phantomsection\label{ref-manual/XrdTestSlave:XrdTestSlave.XrdTestSlave.handleTestCaseFinalize}\pysiglinewithargsret{\bfcode{handleTestCaseFinalize}}{\emph{msg}}{}
TODO:

\end{fulllineitems}

\index{handleTestCaseInitialize() (XrdTestSlave.XrdTestSlave method)}

\begin{fulllineitems}
\phantomsection\label{ref-manual/XrdTestSlave:XrdTestSlave.XrdTestSlave.handleTestCaseInitialize}\pysiglinewithargsret{\bfcode{handleTestCaseInitialize}}{\emph{msg}}{}
TODO:

\end{fulllineitems}

\index{handleTestCaseRun() (XrdTestSlave.XrdTestSlave method)}

\begin{fulllineitems}
\phantomsection\label{ref-manual/XrdTestSlave:XrdTestSlave.XrdTestSlave.handleTestCaseRun}\pysiglinewithargsret{\bfcode{handleTestCaseRun}}{\emph{msg}}{}
TODO:

\end{fulllineitems}

\index{handleTestSuiteFinalize() (XrdTestSlave.XrdTestSlave method)}

\begin{fulllineitems}
\phantomsection\label{ref-manual/XrdTestSlave:XrdTestSlave.XrdTestSlave.handleTestSuiteFinalize}\pysiglinewithargsret{\bfcode{handleTestSuiteFinalize}}{\emph{msg}}{}
TODO:

\end{fulllineitems}

\index{handleTestSuiteInitialize() (XrdTestSlave.XrdTestSlave method)}

\begin{fulllineitems}
\phantomsection\label{ref-manual/XrdTestSlave:XrdTestSlave.XrdTestSlave.handleTestSuiteInitialize}\pysiglinewithargsret{\bfcode{handleTestSuiteInitialize}}{\emph{msg}}{}
TODO:

\end{fulllineitems}

\index{parseTags() (XrdTestSlave.XrdTestSlave method)}

\begin{fulllineitems}
\phantomsection\label{ref-manual/XrdTestSlave:XrdTestSlave.XrdTestSlave.parseTags}\pysiglinewithargsret{\bfcode{parseTags}}{\emph{command}}{}
TODO:

\end{fulllineitems}

\index{readConfig() (XrdTestSlave.XrdTestSlave method)}

\begin{fulllineitems}
\phantomsection\label{ref-manual/XrdTestSlave:XrdTestSlave.XrdTestSlave.readConfig}\pysiglinewithargsret{\bfcode{readConfig}}{\emph{confFile}}{}
Reads configuration from given file or from default if None given.
@param confFile: file with configuration

\end{fulllineitems}

\index{recvLoop() (XrdTestSlave.XrdTestSlave method)}

\begin{fulllineitems}
\phantomsection\label{ref-manual/XrdTestSlave:XrdTestSlave.XrdTestSlave.recvLoop}\pysiglinewithargsret{\bfcode{recvLoop}}{}{}
TODO:

\end{fulllineitems}

\index{requestTags() (XrdTestSlave.XrdTestSlave method)}

\begin{fulllineitems}
\phantomsection\label{ref-manual/XrdTestSlave:XrdTestSlave.XrdTestSlave.requestTags}\pysiglinewithargsret{\bfcode{requestTags}}{\emph{hostname}}{}
TODO:

\end{fulllineitems}

\index{run() (XrdTestSlave.XrdTestSlave method)}

\begin{fulllineitems}
\phantomsection\label{ref-manual/XrdTestSlave:XrdTestSlave.XrdTestSlave.run}\pysiglinewithargsret{\bfcode{run}}{}{}
TODO:

\end{fulllineitems}


\end{fulllineitems}

\index{XrdTestSlaveException}

\begin{fulllineitems}
\phantomsection\label{ref-manual/XrdTestSlave:XrdTestSlave.XrdTestSlaveException}\pysiglinewithargsret{\strong{exception }\code{XrdTestSlave.}\bfcode{XrdTestSlaveException}}{\emph{desc}}{}
Bases: \code{exceptions.Exception}

General Exception raised by XrdTestSlave.

\end{fulllineitems}

\index{main() (in module XrdTestSlave)}

\begin{fulllineitems}
\phantomsection\label{ref-manual/XrdTestSlave:XrdTestSlave.main}\pysiglinewithargsret{\code{XrdTestSlave.}\bfcode{main}}{}{}
Program begins here.

\end{fulllineitems}



\section{Appendix}
\label{index:appendix}\begin{itemize}
\item {} 
\emph{genindex}

\item {} 
\emph{modindex}

\item {} 
\emph{search}

\end{itemize}


\renewcommand{\indexname}{Python Module Index}
\begin{theindex}
\def\bigletter#1{{\Large\sffamily#1}\nopagebreak\vspace{1mm}}
\bigletter{c}
\item {\texttt{ClusterManager}} \emph{(Linux)}, \pageref{ref-manual/XrdTest:module-ClusterManager}
\indexspace
\bigletter{x}
\item {\texttt{XrdTest}}, \pageref{ref-manual/XrdTest:module-XrdTest}
\item {\texttt{XrdTest.ClusterManager}}, \pageref{ref-manual/XrdTest:module-XrdTest.ClusterManager}
\item {\texttt{XrdTest.ClusterUtils}}, \pageref{ref-manual/XrdTest:module-XrdTest.ClusterUtils}
\item {\texttt{XrdTest.Daemon}}, \pageref{ref-manual/XrdTest:module-XrdTest.Daemon}
\item {\texttt{XrdTest.DirectoryWatch}}, \pageref{ref-manual/XrdTest:module-XrdTest.DirectoryWatch}
\item {\texttt{XrdTest.GitUtils}}, \pageref{ref-manual/XrdTest:module-XrdTest.GitUtils}
\item {\texttt{XrdTest.Job}}, \pageref{ref-manual/XrdTest:module-XrdTest.Job}
\item {\texttt{XrdTest.SocketUtils}}, \pageref{ref-manual/XrdTest:module-XrdTest.SocketUtils}
\item {\texttt{XrdTest.TCPClient}}, \pageref{ref-manual/XrdTest:module-XrdTest.TCPClient}
\item {\texttt{XrdTest.TCPServer}}, \pageref{ref-manual/XrdTest:module-XrdTest.TCPServer}
\item {\texttt{XrdTest.TestUtils}}, \pageref{ref-manual/XrdTest:module-XrdTest.TestUtils}
\item {\texttt{XrdTest.Utils}}, \pageref{ref-manual/XrdTest:module-XrdTest.Utils}
\item {\texttt{XrdTest.WebInterface}}, \pageref{ref-manual/XrdTest:module-XrdTest.WebInterface}
\item {\texttt{XrdTestHypervisor}}, \pageref{ref-manual/XrdTestHypervisor:module-XrdTestHypervisor}
\item {\texttt{XrdTestMaster}}, \pageref{ref-manual/XrdTestMaster:module-XrdTestMaster}
\item {\texttt{XrdTestSlave}}, \pageref{ref-manual/XrdTestSlave:module-XrdTestSlave}
\end{theindex}

\renewcommand{\indexname}{Index}
\printindex
\end{document}
